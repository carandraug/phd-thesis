\section{Results}
  \subsection{Development of MaryI}
  
  
%About the short resume of the technique here's what I was doing at the
%end. Basically I was using 10 iterations at 10\% laser power for PA-GFP
%(on tandem with mRuby, what I call MaryI ) and 30 iterations at 1%
%laser power for mEos2. The pixel-dwell for the activation was between
%11-13µs (I can't check them now, see bottom of mail why). On the
%bleaching settings there's a box to use a pixel dweel different than
%the one for imaging. I was doing that. It was also only a small square
%that I was using for activation. In your directory of data there's a
%directory with my name. In there there is one called FancyFRAP. Look
%at the images of the last 3 days for the best settings. By the way, I
%did this with MaryI-CENP-S, MaryI-CENP-W and mEos2-CENP-S. It's all on
%the names of the directory.

%About optimizing the laser settings, I wrote a script for octave which
%is at https://github.com/af-lab/scripts/blob/master/microscopy/optimal-laser-settings.m
%On the top of that file says how to use it. The settings used will be
%different for other microscopes, you will need something like an
%H3-PA-GFP plasmid (Christian said he already has one) to optimize this
%and do it in vivo since fixed samples should behave differently. For
%my case, I didn't found that much difference between 8 and 14
%iterations at 10\% laser power. For mEos2 I also didn't saw great
%difference between 24 and 32 iterations at 1\% power so it's not that
%sensitive to the settings you use just much lower than what I tried at
%the start. I never tried to change the pixel dweel since that would be
%1 more variable I just always used the same settings for that (but the
%code I wrote is able to handle it, it's just more difficult to
%interpret and I don't think it was necessary anyway).

