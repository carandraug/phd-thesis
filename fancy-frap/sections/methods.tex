\section{Materials preparation}

  \subsection{Plasmid construction}

    The following plasmids were provided by Chelly van Vuuren in
    Prof.~Kevin Sullivan laboratory at the National University of
    Ireland Galway, and confirmed by DNA sequencing:

    \begin{itemize}
      \item mCherry--CENP-S
      \item mCherry--CENP-T
      \item mCherry--CENP-W
      \item mCherry--CENP-X
      \item mEos2--CENP-S
      \item mEos2--CENP-W
      \item mEos2--CENP-X
      \item mRFP--PCNA
      \item mEos2--C1
      \item mCerulean3--C1
      \item mRuby--C1
      %% No one seems to know how the following were created.  They didn't
      %% even knew on what terminal the PAGFP was.  Chelly's best guess was
      %% that PAGFP from PAGFP-N1 was cloned into a gateway vector and from
      %% there into pIC113 from Cheeseman lab:
      %%    http://jura.wi.mit.edu/cheeseman/Plasmids.php
      %% before having the CENPs cloned into it.
      %% However, sequencing results show none of the TEV and S tag from
      %% pIC113, which are the important parts of that plasmids.  There is
      %% only a tiny bit of sequence before the CENPs and after PAGFP that
      %% matches pIC113 but should not have any purpose (see lab notebook
      %% dates 07/08/2012 for details).
      \item CENP-S--PAGFP
      \item CENP-T--PAGFP Ala269Ser
      \item CENP-W--PAGFP
      %% This one was specially problematic.  It looks ok in a gel, but
      %% sequencing with EGFP-N1-R and EGFP-N1-F does not work.  The
      %% sequencing looks good until it reaches the linker sequence at
      %% which point stops abruptly.  Contacted Agowa who don't know why
      %% it's not working (see notes on lab notebook between
      %% 19/08/2012 and 23/08/2012).  In the end, the CENP-X and PAGFP
      %% sequence look good but we have no idea about the linker.  Kevin
      %% said Nadine had troubles with this plasmid.  For all he knows,
      %% this might have never worked since it's not even on her thesis.
      %% Would have been nice if they had told me at the start...
      \item CENP-X--PAGFP
    \end{itemize}

    The plasmids with centromeric proteins tagged with mCherry, PAGFP, and
    mEos2 were cloned on Prof.~Kevin Sullivan's laboratory.
    The mRFP--PCNA plasmid was originally from Prof.~Ciaran Morrisson
    laboratory at the National University of Ireland, Galway
    \citep{cloning-mRFP-PCNA-by-Helen}.
    The mCerulean3 and mRuby plasmids originated from Prof.~Richard Day
    from Indiana University, School of Medicine.

    \paragraph{CENP-T--PAGFP}
      Created by correcting the Ala269Ser mutant.  Done by PCR mutagenesis
      with the oligonucleotides AFG544 and AFG545.

    %% These were created as control for MaryI or an as alternative to MaryI
    %% by co-transfection.  We didn't use mCherry because it was weak and
    %% mRuby had reported much better.
    \paragraph{mRuby--CENP-S, mRuby--CENP-T, mRuby--CENP-W, mRuby--CENP-X}
      Plasmids enconding CENP-S, T, W, and X tagged with mRuby at the
      N-terminal were created.

      the cenps were extracted from the mCherry plasmids and inserted into
      mRuby--C1.  This adds the Kozak consensus sequence (which had been
      removed not sure why on mcherry), adds a SGL at the start of the linker

      digest with SmaI and HindIII-HF, extract from gel and ligate.



    \paragraph{mCerulean3--PCNA}
      Created by subcloning PCNA from mRFP--PCNA into into mCerulean--C1.
      Both plasmids were digested with BamHI-HF and HindIII-HF, the fragments
      extracted from an agarose gel, and ligated.

    \paragraph{pBOS--H2B--mEos2}
      The plasmid was created by replacing the GFP gene in pBOS--H2B--GFP
      (\Sref{sec:kill-frap:cloning}) with an insert from pmEos2--C1.
      The insert was prepared by PCR amplification with the oligonucleotides
      AFG510 and AFG511, and then digested by NotI-HF and BamHI.
      The vector was prepared by digestion with the same restriction
      enzymes followed by extraction from an agarose gel.

    \paragraph{pmEos2--CENP-T}
      The plasmid was created by replacing mCherry in pmCherry--CENP-T
      with mEos2 from pmEos2--C1.  Both were digested with AgeI and EcoRI,
      the bands extracted from an agarose gel, and then ligated.


  \subsection{Cell culture}

    We used Hep2 and U2-OS cells in Jena (first time).  Second time in Jena
    we got HeLa CCL2 from a another lab in Jena.  Also HeLa in Galway.

    \subsubsection{Transfection by electroporation}
      Transfection was always performed by electroporation, Amaxa program 0-17.


    \subsubsection{Synchronization of cells by double thymidine block}


  \subsection{Microscopy}

    \subsubsection{Widefield microscopy}
      Widefield fluorescence microscopy was performed in a DeltaVision Core
      microscope from Applied Precision; fitted with the Quantifiable Laser
      Module~(QLM) hardware module, a Xenon lamp for light source,
      and a CoolSNAP\textsubscript{HQ\textsuperscript{2}} camera.

      Cells were cultured in \SI{35}{\mm} glass-bottom MatTek dishes
      for live-cell imaging.  Temperature and CO\textsubscript{2} levels
      for imaging environment were controlled within an acrylic
      environmental chamber that fully enclosed the microscope stage
      plate and objectives.

    \subsubsection{Confocal microscopy}
      Confocal microscopy was performed with a LSM~710 from Zeiss on
      an inverted stand Axio Observer.Z1 fitted with the Zeiss Definite Focus.

      For live cell imaging, cells were cultured on top of round
      \SI{30}{\mm} diameter glass coverslips within wider
      plastic petri dishes.
      Before imaging, coverslips were washed with PBS, and mounted on
      a POC-R2 chamber system, again with PBS.
      Temperature was controlled via a heated stage, an objective ring,
      and a small incubator enclosing the heated staged.
      %% Note that the incubator was only for controlling temperature,
      %% there was no CO2 levels control.

      For fixed samples, cells were cultured on top of round \SI{15}{\mm}
      diameter glass coverslips within wider plastic petri dishes.
      Coverslips were washed with \dc{37} PBS, and mounted on glass
      slides with ProLong Gold Antifade embedding media from Invitrogen.
      Embedding media was solidified overnight at room temperature within
      a dark box.

  \subsection{Image analysis}
