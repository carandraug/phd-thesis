%% We need to follow NUIG structure and style in the thesis.  They
%% tell us how to do it but they don't really give a template or latex
%% document class to use.  I guess being able to format a document is
%% part.  Follows their documentation retrieved from ``University
%% Guidelines for Research Degree Programmes --- For Research
%% Students, Supervisors and Staff'', July 2016 edition, retrieved on
%% Mon 6 Feb 23:05:05 GMT 2017 from
%% HTTP://www.nuigalway.ie/media/graduatestudies/files/university_guidelines_for_research_degree_programmes.pdf
%%
%% From Section 6 - The PhD Examination Process
%%
%% 6.2.3 Directions on Format, Layout and Presentation
%%
%% The PhD thesis should not normally exceed 80,000 words, inclusive
%% of appendices, footnotes, tables and bibliography.  It is
%% university policy that the practice of engaging professional
%% editorial services to assist in writing the thesis is not
%% permitted.  There must be a title page which shall contain the
%% following information:
%%
%%  a. The full title (and subtitle, if any)
%%  b. The volume number and total number of volumes, if more than one
%%  c. The full name of the candidate, followed, if desired, by any
%%     degree and/or professional qualification(s)
%%  d. The name(s) of the supervisor(s), School(s), component
%%     Discipline(s), Institution
%%  e. The month and year of submission.
%%
%% Format and Layout
%%
%% The 'Table of Contents', which should not be over-detailed, shall
%% immediately follow the title page.  The text must be printed on good
%% quality (110g/m2) A4 size paper.  Line-spacing should be a maximum
%% of one-and-half; text must be left justified with a left-hand
%% margin of 4 cm and may be right justified.  An easily-readable
%% layout and double-sided printing are recommended for the body text.
%% For double sided printing ensure that the right hand margin is also
%% adequate for binding (i.e. a margin of 4 cm).  More compact formats,
%% with smaller font sizes, are usually appropriate for certain
%% sections, such as reference lists, bibliographies and some kinds of
%% appendices.  Pages must be numbered consecutively, with page numbers
%% located centrally at the bottom, and chapter headers at the top, of
%% each page.  Diagrams, graphs, photographs and tables should be
%% properly numbered and located in relation to the text.  The copies
%% of the thesis presented initially for examination must be spiral or
%% gum-bound.
%%
%% This is pretty much repeated in Appendix 1 --- Regulations for
%% Higher Research Degrees, Section 10 Submission of the Thesis.

\chapterstyle{veelo}

\OnehalfSpacing

\makepagestyle{NUIG}
\makeevenfoot{NUIG}{}{\thepage}{} % page numbers in the centre
\makeoddfoot{NUIG}{}{\thepage}{} % page numbers at the centre
\makeevenhead{NUIG}{\leftmark}{}{} % page marks in the edge
\makeoddhead{NUIG}{}{}{\rightmark} % page marks in the edge
\pagestyle{NUIG}

%% NUIG style requires 4cm for the spine margine (actually, it
%% requires 4 cm on the left margin, and 4 cm on the right margin if
%% it is to be double sided so that is is ``adequate for binding''.
%% Sounds like what matters is the spine margin and not right or left
%% margin).  Anyway, for an A4 page with font size 10pt (the default),
%% the memoir class default to ~3.5cm on the spine, and ~5.5cm on the
%% other side.  I kinda like the default format and text width so if
%% we add .5cm on one side, I'm taking it back from the other side.
%% Those default margins are computed based on the font size so the
%% text ends with roughly 66 characters per line.  Check the margin
%% sizes again if we have to change the font size.
\setlrmarginsandblock{4cm}{5cm}{*}
\checkandfixthelayout

%% remove colorlinks option when ready for print
\usepackage[final,hyperindex,hyperfootnotes,bookmarksnumbered,colorlinks]{hyperref}

\usepackage[T1]{fontenc}
\usepackage[utf8]{inputenc}
\usepackage{textcomp}

\usepackage{palatino}
\usepackage[euler]{textgreek}

\maxtocdepth{subsection}

\usepackage[final]{graphicx}

%% Input files that input others relative to themselves instead of
%% relative to the initial tex file.  Handy for each chapter but an
%% absolute requirement to include the pdf_tex figures from inkscape
%% which call includegraphics with the filename only.
\usepackage{import}

\usepackage{amsmath}

\usepackage[textsize=footnotesize]{todonotes}
  %% new command for box about missing references
  \newcommand{\addref}[1]{\todo[color=red!40,size=\tiny]{Add reference: #1}}

\usepackage{enumitem}       % so we can use the unboxed style when item names are too long
\usepackage{longtable}      % because memoir's ctabular does not work well with eqparbox
\usepackage{eqparbox}       % for adjusting size of table column (specially on appendices)
  %% Checks what LaTeX thinks is best for a column and save that value.
  %% It will then use it to calculate what's left of \textwidth, and
  %% use it for the other columns. See http://tex.stackexchange.com/questions/95397
  \newsavebox{\SolutionNameBox}
  \newcolumntype{\SolutionNameCol}{
    >{\begin{lrbox}{\SolutionNameBox}}l<{\end{lrbox}
    \eqmakebox[SolutionNameBox][l]{\unhcopy\SolutionNameBox}}
  }

\usepackage{tikz}

\newsubfloat{figure}        % subfigures with LaTeX

\usepackage{rotating}       % for sideways tables and figures
  \newcommand{\crows}[1]{\multicolumn{2}{c}{#1}}

%% Use agu style (American Geophysical Union) which only uses author
%% forenames and after too many authors, uses et. al.  All this helps
%% saves a lot of paper.
\usepackage[round]{natbib}
\bibliographystyle{agu}

\usepackage{seqsplit}
\usepackage{dnaseq}

\usepackage{siunitx}
  \DeclareSIUnit{\gn}{\textit{g$_n$}}   % standard gravity
  \DeclareSIUnit{\bp}{bp}               % base pairs
  \DeclareSIUnit{\cfu}{cfu}             % colony forming unit
  \DeclareSIUnit{\Molar}{\textsc{m}}
  \DeclareSIUnit{\mm}{\si{\milli}\si{\meter}}
  \DeclareSIUnit{\mM}{\si{\milli}\si{\Molar}}
  \DeclareSIUnit{\uM}{\si{\micro}\si{\Molar}}
  \DeclareSIUnit{\X}{\times}
  \newcommand{\dc}[1]{\SI{#1}{\degreeCelsius}}
  \newcommand{\pcent}[1]{\SI{#1}{\percent}}

\newcommand{\captionIntro}[2]{\caption[#1]{\textbf{#1} #2}}

%% Just like we have cite and citep to cite in text and between parentheses,
%% have the same for fref, tref, etc...
\newcommand{\frefp}[1]{(\fref{#1})}
\newcommand{\trefp}[1]{(\tref{#1})}
\newcommand{\Crefp}[1]{(\Cref{#1})}
\newcommand{\Srefp}[1]{(\Sref{#1})}
\newcommand{\Arefp}[1]{(\Aref{#1})}


\newcommand{\species}[1]{\textit{#1}}

%% NCBI Style Guide, Chapter 5 "Style Points and Conventions", recommends
%% italic for gene names (except in long list of genes), and roman for
%% protein names.
\newcommand{\gene}[1]{\textit{#1}}
\newcommand{\protein}[1]{#1}

\newcommand{\Kon}{$K_{on}$}
\newcommand{\Koff}{$K_{off}$}

\newcommand{\G}[1]{G$_#1$}  % for G0, G1, and G2 phases

\renewcommand{\abstractname}{Summary}

%% make it easy to center any dedication
\newcommand{\dedication}[1]{
{\clearpage\mbox{}\vfill\centering #1 \par\vfill\clearpage}}

\usepackage{makecell}
\usepackage[UKenglish,abbreviations]{foreign}

\input{methods/results/software_versions}
