%% "University Guidelines for Research Degree Programmes - For Research Students,
%% Supervisors and Staff"
%% http://www.nuigalway.ie/graduatestudies/documents/university_research_guidelines.pdf
%% Retrieved on Wed 19 Dec 2012 12:42:43 WET
%%
%% Excerpt from Section 6 - The Examination Process
%%
%%  6.3 The Thesis
%%
%%  Norms as to the formats of research master’s and PhD theses vary. In general,
%%  thesis formats, structures and layouts that are consistent with readability and
%%  compactness are required. Acceptable formats and other attributes of Master
%%  and PhD theses are outlined in the University Regulations (Appendices 3 and 1,
%%  respectively) and may be specified in more detail in local guidelines.
%%  It is university policy that the practice of engaging professional editorial services
%%  is not permitted.
%%
%%  6.3.1 Length of Thesis
%%
%%  [...] The PhD thesis should not
%%  normally exceed 80,000 words, inclusive of appendices, footnotes, tables and
%%  bibliography.
%%
%% Excerpt from Appendix 1 - Regulations for Higher Research Degrees
%%
%%  Submission of the Thesis
%%
%%  10. The candidate must follow the directions on format, layout and presentation
%%      of a thesis, as described below. Two copies of the PhD thesis, spiral or gum-
%%      bound, must be lodged with the Examinations Office (unless otherwise
%%      stated). Each copy of the thesis must be accompanied by:
%%        a) a ‘Summary of the Contents’, not exceeding 300 words in length
%%        b) a copy of the completed form EOG 020.
%%        c) Library Submission Form, EOG 051 signed by the candidate
%%
%%      Directions on Format, Layout and Presentation
%%
%%      There must be a title page which shall contain the following information:
%%        a) The full title (and subtitle, if any)
%%        b) The volume number and total number of volumes, if more than one
%%        c) The full name of the candidate, followed, if desired, by any degree and/or
%%           professional qualification(s)
%%        d) The name(s) of the supervisor(s), school(s), component discipline(s),
%%           institution
%%        e) The month and year of submission.
%%
%%      Table of Contents
%%
%%      The ‘Table of Contents’, which should not be over-detailed, shall immediately
%%      follow the title page.
%%
%%      Format and Layout
%%
%%      The text must be printed on good quality (110g/m2) A4 size paper with a
%%      left-hand margin of 4 cm. A maximum of one-and-half line-spacing, left
%%      and right justified, an easily readable layout and double sided printing are
%%      recommended for the body text. More compact formats, with smaller font
%%      sizes, are usually appropriate for certain sections, such as reference lists,
%%      bibliographies and some kinds of appendices. Pages must be numbered
%%      consecutively, with page numbers located centrally at the bottom, and chapter
%%      headers at the top, of each page. Diagrams, graphs, photographs and tables
%%      should be properly numbered and located in relation to the text.
%%
%% No local guidelines found at
%% http://www.nuigalway.ie/graduatestudies/Current_Students_2/local_guidelines.html
%% only " To be added shortly" was displayed on the page

\chapterstyle{veelo}

\OnehalfSpacing
%% remove colorlinks option when ready for print
\usepackage[final,hyperindex,hyperfootnotes,bookmarksnumbered,colorlinks]{hyperref}

\usepackage[T1]{fontenc}
\usepackage{palatino}
\usepackage[euler]{textgreek}

\maxtocdepth{subsection}

\usepackage{graphicx}
  \graphicspath{{./figs/}{./results/}}

\usepackage{amsmath}

\usepackage[textsize=footnotesize]{todonotes}
  %% new command for box about missing references
  \newcommand{\addref}[1]{\todo[color=red!40,size=\tiny]{Add reference: #1}}

\usepackage{enumitem}       % so we can use the unboxed style when item names are too long
\usepackage{longtable}      % because memoir's ctabular does not work well with eqparbox
\usepackage{eqparbox}       % for adjusting size of table column (specially on appendices)
  %% Checks what LaTeX thinks is best for a column and save that value.
  %% It will then use it to calculate what's left of \textwidth, and
  %% use it for the other columns. See http://tex.stackexchange.com/questions/95397
  \newsavebox{\SolutionNameBox}
  \newcolumntype{\SolutionNameCol}{
    >{\begin{lrbox}{\SolutionNameBox}}l<{\end{lrbox}
    \eqmakebox[SolutionNameBox][l]{\unhcopy\SolutionNameBox}}
  }


\usepackage{tikz}

\newsubfloat{figure}        % subfigures with LaTeX

\usepackage{rotating}       % for sideways tables and figures
  \newcommand{\crows}[1]{\multicolumn{2}{c}{#1}}

\usepackage[round]{natbib}

\usepackage{siunitx}
  \DeclareSIUnit{\gn}{\textit{g$_n$}}   % standard gravity
  \DeclareSIUnit{\bp}{bp}               % base pairs
  \DeclareSIUnit{\cfu}{cfu}             % colony forming unit
  \DeclareSIUnit{\Molar}{\textsc{m}}
  \DeclareSIUnit{\mM}{\si{\milli}\si{\Molar}}
  \DeclareSIUnit{\uM}{\si{\micro}\si{\Molar}}
  \DeclareSIUnit{\X}{\times}
  \newcommand{\dc}[1]{\SI{#1}{\degreeCelsius}}
  \newcommand{\pcent}[1]{\SI{#1}{\percent}}

\newcommand{\captionIntro}[2]{\caption[#1]{\textbf{#1} #2}}

\newcommand{\species}[1]{\textit{#1}}

\newcommand{\Kon}{$K_{on}$}
\newcommand{\Koff}{$K_{off}$}

\newcommand{\G}[1]{G$_#1$}  % for G0, G1, and G2 phases

\renewcommand{\abstractname}{Summary}

%% make it easy to center any dedication
\newcommand{\dedication}[1]{
{\clearpage\mbox{}\vfill\centering #1 \par\vfill\clearpage}}

