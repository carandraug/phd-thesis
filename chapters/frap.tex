\chapter{FRAP with histones}
\label{ch:frap}

%% FIXME this epigraph may not be acceptable...
\epigraph{Agora desenmerda-te.}{Portuguese ``saying''}

%% chapter concept: in this chapter goes the whole kill FRAP project. I started
%% positive that this should work and so should the text. We start by studying
%% the technique and list the assumptions it requires. We do one experiment and
%% face some problems. Each problem has its own section that ends in a solution.
%% When the solution was not done, we could offer one. The last one is the cause
%% that this is not possible and has no solution. Conclusion lists all problems
%% and their solutions again.

%% FIXME 2nd year report
%%My overall objective is to measure \textit{in vivo} the role of different nucleosome regions
%%in chromatin dynamics. Based upon the workflow established in my first year, I aim to obtain
%%the mathematical analysis knowledge that would enable examination of FRAP data and obtain
%%accurate quantitative data. This defines a series of objectives:
%%
%%\begin{enumerate}
%%  \item Create stable cell lines expressing a FP-tagged Sin$^{-}$ mutant;
%%  \item Track cells moving during long time FRAP experiments;
%%  \item Obtain recovery curves from FRAP experiments;
%%  \item Measure kinetic properties from the recovery curves.
%%\end{enumerate}

\noindent
The building block of eukaryotic chromatin structure is the nucleosome, comprising
\SI{147}{\bp} of DNA wrapped around an octamer of two copies of core histones H2A,
H2B, H3 and H4. They are arranged in a linear chain separated by DNA linkers, and
can be further compacted into higher order chromatin structures. But chromatin extends
well beyond DNA compaction, it is a dynamic complex that controls access to genetic
information by undergoing reconfiguration of its structure.

Since nucleosomes are chromatin's basic structural unit, it is the interactions
within these structures that determine its configuration. As such, these local
reconfigurations can be accomplished by changing nucleosome structure or altering
its composition by post-translational modifications or incorporation of variant
histones.

These changes, and through them access to genetic information, are usually performed
by protein complexes. Mutations of the histones can relieve the need of these complexes
by affecting the stability of the nucleosomes so it is possible to gain insight on
the mechanism of these complexes by studying the kinetic alterations these mutants create.

Studies of this mutations have all been done \textit{in vivo} or yeast. We are trying
to validate to measure their effects in the context of chromatin by using live cells.

\section{The FRAP model}

  Fluorescence recovery after photobleaching (FRAP) is an optical technique
  that reveals the dynamics of fluorescently tagged molecules within live cells.
  The tagged molecules inside a small region are irreversibly photobleached by
  action of a high-power focused laser beam and the recovery rate of fluorescence
  is measured. The recovery rate is interpreted as unbleached molecules,
  which were outside of the region at the time of photobleaching, moving into
  the bleached area, replacing the bleached molecules. It is assumed that the
  the fluorescence recovery reflects the protein natural movement.

  This technique has been extensively used to obtain qualitative and quantitative
  insight on the kinetic properties of proteins. Development of this technique has
  led to complex models that are both more precise and accurate than simple models
  based on inverse exponential decay. These take into account important parameters
  that were previously discarded but have since been shown as important, such
  as diffusion and the bleach spot profile shape.

  However, despite their sophistication, these models of recovery make certain important
  assumptions that are hard to maintain for long experimental observation times:

  \begin{itemize}
    \item the biological system has reached equilibrium before photobleaching;
    \item total amount of both Fluorescent Protein (FP) fusion protein and its
          binding sites remain constant over the time course of the recovery;
    %% FIXME we should probably group the 2 before into one
    \item distribution of tagged molecule mimics the endogenous protein;
    \item the binding sites are part of a large, relatively immobile complex, at
          least on the time and length scale of the recovery.
  \end{itemize}

  FRAP has been successfully used before to show that different core histones have
  different kinetics and populations. We expected to be able to use the same approach
  for \textit{in vivo} validation of previously identified differences between
  wild type and mutant histones.

\todo[inline]{where do we mention FRAPINATOR? That goes in the Octave chapter but we still need to
              reference it from here somewhere.}


\section{Materials preparation}
  %% TODO
  \todo[inline]{shouldn't I have a section where I describe the cloning of all the plasmids
                I made for this? There's Sin- mutants, histone variants, different colors...
                
                Also, how I created the stable cell lines, first by picking colonies and
                then using FACS to sort them. There's those nice plots from the sorting.
                
                Maybe I should split the methods into the methods chapters and materials
                chapeter.
                
                And the kill curve for selection of antbiotc concentration}

\section{The required time frame}
  
  While the FRAP model we are using works best when complete recovery is observed, it could be
  possible to observe some change on a fast exchanging population at the start of the recovery.
  
  Since the histone mutant which had shown the highest \textit{in vitro} effect on nucleosome
  was H4 R45H, we used it as starting point for testing this. We performed a 90 seconds FRAP
  experiment with it, H3 T45A, and H4 WT but at were unable to find a noticeable difference
  (\fref{fig:90-sec-frap}).
  
  \begin{figure}
    \centering
    \missingfigure{plots comparing H3 T45A and H4 and H4 R45H \Kon and \Koff}
    \captionIntro{\Kon and \Koff values estimated from 90 seconds FRAP experiment.}
                 {}
    \label{fig:90-sec-frap}
  \end{figure}
  
  We then tried to simulate how a recovery curve for a single population of proteins with
  residence times of 9 and 7 hours would look like (\fref{fig:simulated-frap}). We assumed
  that these were reasonable values given Kimura and Cook paper. This had shown us that indeed
  the initial 90 seconds would not be enough to measure a difference on the microscope.
  
  \begin{figure}
    \centering
    \missingfigure{comparison of recovery of 2 proteins with residence time of 9 and 7 hours}
    \captionIntro{Simulated recovery curves of proteins with long residence times.}
                 {}
    \label{fig:simulated-frap}
  \end{figure}
  
  Considering the typical fluorescence intensity we observed on our cells (around 40), and the simulated curve,
  it would require at least 1 hour of recovery until the difference in intensity between the two was of 1, so
  we could see with our microscope camera. Of course, this is the minimum, and considering the noise, we need
  much much more. At least 4 hours would be required to obtain data to validate any difference, and even then,
  more should have been used.

\section{Cell movement}

  While most FRAP experiments are done on a short time scale where the cell movement
  is not noticeable, the low recovery rates of histones requires collection of images
  over several hours. \addref[Kimura and Cook] reported that the recovering was not
  complete, even after 8h for H2B--EGFP, the time frame we used for our initial experiments.
  On time series experiments with of such long
  times, cell movement becomes a issue (\fref{fig:cell-movement}) thus requiring tracking
  of the cell nuclei. This is assuming that we want to automate the analysis of course.
  Would not be a problem if we had trained monkeys drawing regions by hand.
  
  \begin{figure}
    \centering
    \missingfigure{HeLa cells moving around}
    \captionIntro{Long time series of HeLa cells expressing H2B--EGFP.}
                 {Cells were transfected with pBOS--H2B-EGFP and imaged for 8
                  hours, with intervals of 20 minutes.}
    \label{fig:cell-movement}
  \end{figure}
  
  We tried a couple of different approaches to solve this issue.
  
  \subsection{Higher cell density}
  
    Our first attempt to limit the movement of the cells was to perform the experiment with
    cells at higher cell density.  Contact between cells is a factor that usually inhibits
    their movement, a characteristics named ``contact inhibition''\addref[Density Dependent Inhibition of Cell Growth in Culture].
    
    However, HeLa cells did not seem to respect this, and kept moving around, even crawling
    over each other in occasions. This probably comes from the fact that HeLa cells are fucked
    up cancer cells with little respect for anything other than I want to multiply.
    
    We did notice a decrease on movement, probably due to the fact that the cells did have less
    space to move around, the dish was more crowded. Still, this was not enough

  \subsection{Primary cell lines}
  \label{sec:frap-movement-horse}
  
    The issue with HeLa cells is that they are cancer cell lines where all the checks that typical
    healthy cells are ignored. As such, we attempted the same approach with a primary cell line.
    
    We used a horse fibroblast cell line as these are supposed to stop growth once they have reached
    confluence. They are supposed to create a tissue layer of healthy cells and not move. Their main
    disadvantage, aside the more difficult handling and limited passage number, is the much greater
    difficult in transfection, with a much lower transfection efficiency. Also, these cells are not
    so standard which would make taking comparisons from other studies less conclusive.
    
    We made an initial check that the horse cells would indeed stay healthy after reaching confluence.
    We let them reach confluence on a dish and left them alone for 2 weeks, only changing the growth
    media every 3 days. After this time the cells had formed a tissue layer and looked healthy (\fref{fig:healthy-horse}).
    
    \begin{figure}
      \centering
      \missingfigure{Horse nucleus cells looking healthy after 2 weeks}
      %% TODO we need to redo this experiment, this time with glass bottom for imaging after.
      %%      Problem may be that when doing the staining, some cells will be washed away and
      %%      will look fake. We can take an image before the washing to explain that.
      %%      something better than just DAPI to show that the cells are actually healthy?
      %%
      %%      Just some viability test using trypan blue, and compare against the cells kept growing
      \captionIntro{Effect of contact inhibition on primary horse cell line.}
                   {The cells were left on the dish for 2 weeks after reaching confluence. The
                    image on the left is the bright field image before staining and fixation.
                    The cells on the right are the DAPI staining for better view. The missing
                    areas is due to the washing steps involved with the staining.}
      \label{fig:healthy-horse}
    \end{figure}
    
    Happy with this, I transfected the cells with H2B--EGFP and assess the reduction in movement.
    The transfection efficiency was much lower than when using HeLa as expected (this was checked 2 days
    after transfection). We imaged the cells over 8 hours but cells moved anyway (\fref{fig:horse-moving}).
    
    \begin{figure}
      \centering
      \missingfigure{Time series of horse cells expressing H2B--EGFP moving after 8 hours}
      \captionIntro{Movement of primary horse fibroblast despite contact inhibition.}
                   {The cells were transfected with H2B--EGFPat high cell density and
                    imaged over 6 hours. They still moved, and even spiralled.}
      \label{fig:horse-moving}
    \end{figure}

    %% We never tried serum starvation


  \subsection{CropReg}
  
    To solve this problem, we developed a computer program named CropReg\footnote{The name CropReg
    comes from the program flow, a loop between image cropping and image registration.}. This
    program implements template matching\todo{image registration vs template matching} by
    finding the maximum in the normalized cross-correlation matrix of a
    box enclosing the nucleus of interest (the template) and the following frame (\fref{fig:normxcorr2}). The
    coordinates of the maximum correspond to the coordinates where the template matches the ``best'' on the
    image. This new coordinates are then used to create a new template which is used to search for the nucleus
    on the following time point.
    
    For performance reasons, rather than using the whole image of the following time point, the area surrounding
    the one on the previous time is used.
    
    \begin{figure}
      \centering
      \missingfigure{Explaining normalized cross correlation. 3 images. First is $f_n$ with a box
                     around the cell that will be used as template $t$. Second image is $f_n$ with the
                     cells moved a bit. Third image is the normalized xcorr2 output as a 3D landscape
                     with an arrow pointing to the max peak.}
      \captionIntro{Using normalized cross-correlation for tracking of cell nucleus.}
                   {The left image, $f_n$
                    The box around the nucleus of interest, $t$, will be the template for the cross-correlation.
                    The center image, $f_{n+1}$, is the image on the time point after $f_n$. The right image is
                    the surface mesh plot of the normalized $(t \star f_{n+1})$.}
      \label{fig:normxcorr2}
    \end{figure}
    
    %% TODO since there's more than one way to actually do the normalization, it might be
    %% a good idea to writedown the actual math formula
    
    %% FIXME we could use the listings package which will look nicer and let us have the actual
    %%      code on a separate file to not break the indentation
    
    \begin{verbatim}
## a is the image
## b is template
c = conv2 (a, conj (b (rows(b):-1:1, columns(b):-1:1))); 
a = conv2 (a.^2, ones (size (b)));
b = sumsq (b(:));
c(:,:) = c(:,:) ./ sqrt (a(:,:) * b);
\end{verbatim}
    
    This is the code for the normalized cross-correlation which is at the core of CropReg. Rather than having
    it for ourselves only, it was implemented as a new option for the xcorr2 function and released under GPLv3+ with
    the Octave Forge signal package version 1.2.0. The function normxcorr2 was released as a wrapper for
    this option, with the image package version 2.0.0.
    
    This algorithm is able to track cell nucleus efficiently, as long as cell nuclei of different cell do not
    overlap\todo{And when they touch... some morphology to count number of nuclei and if two, check with using the
    center of the template.}(\fref{fig:cropreg}).
    
    \begin{figure}
      \centering
      \missingfigure{}
      \captionIntro{CropReg in action.}
                   {This will actually be 3 subfigures. First is finding the nucleus of intereste by image subtraction.
                    Second is just CropReg. Thirs is accounting for image rotation.}
      \label{fig:cropreg}
    \end{figure}

    Two problems are left to solve: nucleus rotation (rotational movement of the nucleus on the
    Z axis) and finding the template for the first image. The first
    was solved by using the StackReg plugin for ImageJ\todo{we should write it in Octave and add it to cropreg}
    \addref[we need to reference it if we do use it]. The second was using by subtracting the post bleach to the
    pre-bleach image, threshold the image, remove noise and use it as marker to select the right nucleus.


\section{Cell cycle --- System equilibrium}
\label{sec:frap-equilibrium}

%% There's no chemical equilibrium in S phase

  Another problem that stems from long time experiments is that the experiment will start to span
  multiple cell cycle phases. I was using HeLa cells which have an average doubling time of 24h.
  Since the experiment are at least eight hours long, and the experiment can not cross mitosis,
  it is problematic to perform it so that it does not cover part of S~phase (\fref{fig:cell-cycle-watch}), the problematic cell
  cycle phase.
  
%  \begin{wrapfigure}{o}{0.5\textwidth}
  \begin{figure}
    \centering
    \missingfigure{a small figure, probably should be a float with wrapping}
    \captionIntro{Cell cycle watch.}
                 {Too short cell cycle and too long experiment.}
    \label{fig:cell-cycle-watch}
  \end{figure}
%  \end{wrapfigure}
  
  The histone expression profile is atypical in that it is mainly expressed during S~phase, suffering
  a 35-fold increase for this phase. This is a consequence of the requirement for large amounts of
  canonical histones during S~phase to package the newly duplicated genome\todo{reference section
  on histone catalogue or introduction where I talk about this}.
  
  This breaks the assumptions of the FRAP model that the system remains in equilibrium. If one
  considers this as a simple chemical reaction, then it boils down to:
    
  \begin{displaymath}
    F + S \overset{K_{on}}{\underset{K_{off}}{\rightleftharpoons}} C
  \end{displaymath}
  
  where $F$ represents the freely diffusing proteins, $S$ represents the vacant binding sites, and
  $C$ the $FC$ complex when the protein is bound to the binding site while we try to measure the
  values of \Kon and \Koff, the association and dissociation constants.
  
  The assumption is that this equilibrium has been reached when we start the FRAP experiment and
  continues so that \Kon and \Koff remain constant throughout the experiment. This does
  not hold for S~phase where there's an increase of $F$ and $S$, the new histones being expressed
  and the new duplicated genome. Adding more reactants to a system in equilibrium will change the equilibrium
  constants. This means that the fluorescence recovery will not be consistent\todo{this is not the right word}.
  
  \subsection{Contact inhibition}
  \label{sec:frap-cell-cycle-horse}
    
    Another effect of contact inhibition is that cells showing this behaviour should stop cell
    growth when reaching confluence. Once reaching a certain state, the cells should move from
    the G$_1$ into the G$_0$ phase where they are arrested until the inhibition stops.
    
    Similar to the approach we took for the cell movement, we used a primary horse fibroblast
    cell line that showed contact inhibition.
    
    \begin{figure}
      \centering
      \missingfigure{FACS of horse cells after 2 weeks on dish and full}
      \captionIntro{FACS of primary cells showing contact inhibition.}
                   {Control is the horses that have been kept growing for the 2 weeks,
                    being split every 3 days, while horse cells were left on dish for
                    2 weeks, changing media only every 3 days.}
      \label{fig:horse-facs}
    \end{figure}
    
    This approach showed success but the cells showed a much lower transfection efficiency. This
    would have been a good approach if the use of this cells had also solved the problem of cell
    movement. As it did not, picking cells was better.

  \subsection{Picking cells}
  
    If we want to avoid changing cell cycle, and avoid drugs for blocking cell growth, during a
    continuous time interval of eight hours, the only cell cycle phase that lasts at least that
    long is the G$_1$ phase (\fref{fig:cell-cycle-watch}). So the ideal cells to perform this
    experiment are cells that are at the beginning of G$_1$ phase.
    
    To be certain of the cell cycle phase at the start of the experiment, we started by looking for
    cells that were entering mitosis. We imaged these cells for 4 hours and used the daughter cells
    for the FRAP experiments (\fref{fig:picking-cells}).
    
    \begin{figure}
      \centering
      \missingfigure{Hela cells splitting}
      \captionIntro{Picking cells at early G$_1$.}
                   {We imaged cells that were entering mitosis and picked their
                    daughter cells for the FRAP experiments. Because HeLa cells lift
                    away from the dish during mitosis, the trick is in opening the
                    pinhole and set the Z-center in between the cell dividing plane
                    and dish bottom. Ends up nothing being properly in focus but we
                    can track things fine. Of course, some cells still floated away.}
      \label{fig:picking-cells}
    \end{figure}
    
    There is care to be taken however. Right after mitosis, there's an interval of a couple of
    hours for chromatin to decondense and reorganize. So we can't pick cells right after mitosis.
    Kimura has recently found fancy dynamics for DNA binding proteins \addref[some ring like shaped protein
    from what Tim was working on] on the first 3 hours of G$_1$.
    
    \todo[inline]{look for papers from Andrew Belmont. Search finite time potential 'spatial disequilibrium'.}

  \subsection{Drugs}
  
    Drugs should be avoided. Different drugs had different effects on Kimura and Cook paper.

    \begin{figure}
      \centering
      \missingfigure{tracking stable mRFP--PCNA cell lines for a long time after picking them after mitosis}
      \captionIntro{Measuring length of G1 phase}
                   {Control is the horses that have been kept growing for the 2 weeks,
                    being split every 3 days, while horse cells were left on dish for
                    2 weeks, changing media only every 3 days.}
    \end{figure}

  \subsection{Alternative cell lines}
  
    An alternative would be to pick cell lines with longer cell cycles (\tref{tab:alternative-cells}).
    
    \begin{table}
      \missingfigure{actually a tabular environment listing some alternative cell lines}
      \caption{Table of cell lines with cell cycles longer than HeLa.}
      \label{tab:alternative-cells}
    \end{table}


\section{Tagged protein distribution}
  An obvious assumption of the model is that the distribution of the tagged protein mimics
  the distribution of the endogenous protein. It was proposed by \addref[Kimura and Cook] that
  this does not hold, which explained the abnormal distribution of H2B--GFP on the cell nuclei.

  %% we could also write this by first mentioning the two types of regulation and explaining them
  %% and them showing how the pBOS plasmid fails with both
  The non-linear expression of the histone proteins previously mentioned (\Sref{sec:frap-equilibrium}),
  is controlled at two stages. On a first level, there's an increase of transcription during S~phase
  but this only accounts for part of the the increase. At a second level, the stability of the histone
  transcripts also increases at S~phase \todo{reference to the introduction where I mention this}.
  
  Unlike all other eukaryotic genes, histone transcripts lack a poly(A) tail, encoding instead a
  stem--loop and a purine-rich Histone Downstream Element (HDE) downstream of the stop codon.
  These two structures, which are not part of the coding sequence, are involved in the
  mRNA stabilization and further responsible for the expression profile of histones.
  
  The pBOS--H2B--EGFP plasmid does not respect any of these. The promoter of the EF-1 \textalpha gene, used in the
  pBOS plasmid, is unregulated leading to constitutive expression of the tagged protein. This lead to different
  distribution of tagged vs endogenous proteins on the free pool over time. Since the duplication
  of genome is not equal, there's more heterochromatin being duplicated at the start of S~phase, while
  euchromatin is duplicated at the end, the distribution of the tagged histones is skewed so that there's
  more of it on the heterochromatin, since it's incorporated when the proportion of tagged histones is
  higher (\ref{fig:mess-histone-expression}).
  
  \begin{figure}
    \centering
    \missingfigure{a scheme like the one in Himura and Cook paper.}
    \captionIntro{Distribution of tagged histones on free pool over time}
                 {Profile of free pool size on typical case vs our experiment. Use
                  different colors under the curve to show the different proportions.
                  Separate plot for the actual expression}.
    \label{fig:mess-histone-expression}
  \end{figure}
  
  Of course, no one actually tested this, can be quite tricky to this with proper controls. And we also
  don't know that if in the case of the overexpression, the cell would at translation or protein degradation
  level. There's chaperones specialized in degrading histones. Kevin seems to believe so. This would
  also explain Kimura and Cook results with the weird distribution if the difference on dynamics
  between H2B and H3 also holds.
  
  I actually tried to perform subcellular fractionation with a protocol from Ronan which is on the wiki.
  The signal of the tagged histone, in relation to the endogenous, was quite high. But the western
  looked like crap, many non-specific bands. Couldn't see with the H3 antibody but something went
  wrong with that lane (it was the one that turned yellow during the boiling of the sample). Now that
  I have the samples, I could try an H3 antibody for a H3 PTM which didn't work before.
  
  While getting the promoter right is a bit tricker, cloning the stem loop and removing the poly-A tail
  is much more straightforward, and is a care that should be taken when doing this. It has been reported
  that the downstream elements have a bigger impact on the expression than the promoter. Luckily, this
  is also the part that should be easier to fix with cloning.
  
  We used a plasmids that Kevin used a long time ago, where he cloned an H3 gene from mouse on its
  entirety and only added a tag to the H3 gene, keeping the rest intact\addref[Kevin's paper
  with the plasmid used in mouse]. We started to work on it but did not had any plasmid map,
  or sequence, other than a very small picture from the paper where he published with very
  bad quality. We couldn't even use any standard sequencing primer. We ended up not doing
  any work on it since by the time we got it, we were already noticing all
  the other issues with the project.

  %% on the case of proteins with special control of expression, may make sense to look into it.
  %% Specially if the proteins will have very slow exchange rates this may affect its distribution.
  
  %% Solution may be to clone to whole gene. In some cell lines (DT40) it's very easy to knock in
  %% a gene and tag the endogenous. When zinc-finger technology becomes more accessible it may
  %% become possible to use it on more common human models.


\section{Chromatin movement}
  
  During some FRAP experiments, we observed major changes in the shape of the bleached area due to
  chromatin movement (\fref{fig:frap-chromatin-movement}). This is difficult to assess because of the
  non homogeneous nature of chromatin and the whole nucleus being stained green.
  
  \begin{figure}
    \centering
    \missingfigure{original H2B-EGFP FRAP experiment with bleached hole acting like a contortionist}
    \captionIntro{H2B--EGFP FRAP experiment showing chromatin movement.}
                 {mention time interval and whatever}
    \label{fig:frap-chromatin-movement}
  \end{figure}
  
  The assumption that the binding sites, in this case chromatin, remain immobile through the FRAP experiment is
  required in order to interpret the recovery of fluorescence as free unbleached molecules moving
  into the bleached area and associating with the binding sites, an effect of the association rates \Kon and
  \Koff. If chromatin itself reshapes and moves unbleached molecules from outside into the bleached area, then
  the recovery is also a function of the binding sites movement. Without a model for this movement, it is not
  possible to separate the two functions, and obtain the association rates.
  
  To correct for this movement, it would be necessary to visualize it. The standard method when doing such long
  FRAP experiments is to use histones for control, where they are usually perceived as being immobile. We are
  at the limit, on the time scale of the experiments we need to do, nothing in the chromatin can be assumed
  to be immobile. A possibility would be to stain the actual DNA but the molecules would still exchange faster
  than the histones.
  
  To have a better look at this, we performed inverse FRAP, where instead of bleaching a small region
  and measure the fluorescence recovery, the protein is tagged with PA--GFP, a small region is
  activated, and its fluorescence decay is measured. Because PA--GFP is invisible before the photoactivation,
  we co-transfected cells with mCherry--\textalpha--tubulin. When performing co-transfection, it's more likely
  that cells have incorporated both plasmids\todo{prove this where?}, so expression of
  mCherry--\textalpha--tubulin is a good indication that it is also expressing H2B--PA--GFP. The use of
  \textalpha--tubulin is also a good indicator of the nucleus boundaries which is a visual aid for this
  experiment (\fref{fig:ifrap-chromatin-movement}).
  
  \begin{figure}
    \centering
    \missingfigure{inverse FRAP experiment showing chromatin movement}
    \captionIntro{inverse FRAP experiment showing chromatin movement.}
                 {mention time interval and whatever. We have co-transfected cells with \textalpha--tubulin
                  tagged with mCherry to find cells that had been transfected and to visualize the borders
                  of the nucleus}
    \label{fig:ifrap-chromatin-movement}
  \end{figure}
  
  With this set up, we were able to notice that the movement of chromatin was indeed more frequent and of
  bigger impact that what the ``normal'' FRAP experiments originally suggested. We also note that even with
  iFRAP, it is not possible to be sure whether chromatin movement has occurred. If intensity in the activated
  region drops by 4 for example, and we can't see those 4 in another location (or locations) of the nucleus,
  it can still be
  that the chromatin that holds those 4 are spread over a larger region which is perfectly reasonable if the
  chromatin is movement and it is likely to be below the sensitivity of the camera.
  
  And the low intensity and greater sensitivity to bleaching of PA--GFP when compared to EGFP, just makes this
  even worse.
  
  We though that this problem would be related with cell movement and could be
  corrected by using a cell line with contact inhibition but this did not happen (see \Sref{sec:frap-cell-cycle-horse} and
  \Sref{sec:frap-movement-horse}).

  \todo[inline]{chromatin is not homogeneous. Different parts of the chromatin should have
                different recoveries. Where can I mention this?}

\section{Conclusions: why histone dynamics is not measurable by FRAP}
  
  I have attempted to establish FRAP as a method for measurement of histone variants. Such
  an experiment requires observation over an extremely long time interval, challenging several
  assumptions of a typical FRAP experiment.
  
  Workarounds were found to each of them, without the introduction of more factors or decreasing
  the fidelity of the FRAP technique. Except for one, movement of chromatin --- the binding sites.
  This is not easy to observe on a conventional wide-field microscope, and even on a confocal
  microscope is hard to spot. This makes it very easy to go unnoticed. I have performed an iFRAP
  experiment which makes it easier to detect.
  
  While the chromatin movement will not be a problem when performing FRAP on slow exchanging
  proteins, the other issues I found might.
  
  %% FRAP is not goof for very long time frames. Reference papers where it was used incorrectly?
  %% see Tim's paper
  
  Still, alternative techniques might be used to measure dynamics of histone variants in
  live cells. Namely, single molecule tracking would be an ideal candidate provided access
  to the required equipment\addref[has Davide published already about his ``new microscope''
  to do this?].
  
  \todo[inline]{search more for histone and single molecule tracking and imaging}
  
  Single-molecule imaging of histones for short period of times in live cells
  has recently been reported using super-resolution imaging\addref[nature methods 7(9):717-719,
  2010 and nature methods 8(1):7-9, 2011].
   
  Also, use of PA--GFP has been used to measure dynamics of H4 over \SI{90}{\ms} reporting
  differences between interphase chromatin and mitotic chromosomes\addref[Saera Hihara et al 2012].
  However, the difference between these two phases is the highest and might not be comparable to
  the difference between histones variants\todo{study this. Someone must have measured this}.
  
  %% did not mention if FRAP could have been used with H2A and H2B since these move faster after
  %% all. However, the ones really important on the nucleosome structure seem to be H3 and H4, and are
  %% the ones of more interest for us.
