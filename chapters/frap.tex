\chapter{FRAP with histones}
\label{ch:frap}

%% chapter concept: in this chapter goes the whole kill FRAP project. I started
%% positive that this should work and so should the text. We start by studying
%% the technique and list the assumptions it requires. We do one experiment and
%% face some problems. Each problem has its own section that ends in a solution.
%% When the solution was not done, we could offer one. The last one is the cause
%% that this is not possible and has no solution. Conclusion lists all problems
%% and their solutions again.

The objective is to study the structure of the histone proteins in their dynamics
in live cells. Many studies have been done but they are all in vitro. FRAP was reported
to work for histones and shown differences between H2B and H3. FRAP is a well established
technique, used a lot in many labs.

\section{The FRAP model}

  %% short overview of FRAP
  Idea behind FRAP is bleach a region and see how unbleached molecules move to that region.
  Speed on ow that happens shows dynamics of proteins.\todo{get some text from GRC reviews and 4th year project}
  %% picture of simple example on how reading FRAP data

  %%examples of when FRAP has been used
  FRAP has been established in 197? to study things moving in lipid membranes. It has been
  used to measure proteins together with GFP since late 90s. It's well established and a
  respectful technique.

  %% mention development of more sophisticated models
  

  %% important assumptions of FRAP model that will fail for long times
    %% system is in equilibrium
    %% binding sites are immobile
    %% distribution of tagged molecule mimics endogenous
    
    %% plot function for long recovery times to see minimum analysis time to see the difference
    %%it's usually necessary to see the whole recovery but for these that's just not possible

\section{Cell movement}
  %% That's the first problem, the cell is moving around. It's not a show killer, just more difficult.
  %% Fixes for these include simple scripts, ImageJ plugins. Some microscope already have software built
  %% in to track cells and keep them on focus. It didn't work for us because if a cell on the field view
  %% was in mitosis, it would mess up the algorithm. However, when the cell moves, the nucleus reshapes
  %% itself, it's not just dragged which makes

    %% \missingfigure{cell moving an reshaping the nucleus as it moves}

  %% A fix for this could have been use of primary cell lines that have contact inhibition (but they are
  %% a pain to transfect). Must try transfection by electroporation
  %% we did use horse cells in the end
    %% \missingfigure{table with common used cell lines, human and mouse, that still show contact inhibition}

\section{Cell cycle}
  %% If cell enters S phase, binding sites double but system must stay in equilibrium.

  %% Cell stays on G1 for longer than G2 so it's prefered.

  %% What's the maximum time that a cell stays on G1? I searched for cells in mitosis,
  %% tracked them for a few hours until they were in G1 and FRAP both of daughter cells.
  %% However, even on the first hours of G1, Kimura found fancy dynamics for some DNA binding proteins.
  
  %% Drugs should be avoided. Different drugs had different effects on Kimura and Cook paper.

  %% contact inhibition could also fix this. But would it still fix the problem of chromatin movement

\section{Expression levels}

     %% on the case of proteins with special control of expression, may make sense to look into it.
     %% Specially if the proteins will have very slow exchange rates this may affect its distribution.

%%      \missingfigure{qPCR showing expression of tagged and endogenous histones on different cell phases
%%      \missingfigure{western showing comparison of endogenous vs tagged
%%      \missingfigure{FRAP recovery, almost 20\% is free pool
%%      \missingfigure{western of chromatin bound vs unbound fractions. Subcellular fractionation

      %% Solution may be to clone to whole gene. In some cell lines (DT40) it's very easy to knock in
      %% a gene and tag the endogenous. When zinc-finger techonology becomes more accessible it may
      %% become possible to use it on more common human models.

    \subsection{protein distribution}
      %% Kimura and Cook mentioned on their original paper that the distribution of the tagged H3 was
      %% different from the endogenous and showed it by mixing distribution of DNA with H3--GFP. This
      %% shuld not be a problem if expression levels are the same.
      %%\missingfigure{similar image as Kimura and Cook (mix channels) + Plot of relative intensities of each pixel}


\section{Chromatin movement}

    %% The chromatin moves. It's not possible to correct for this movement on the model unless it's possible to see
    %% it. However, histones are usually the ones used to see it. This seems to be the limit.

%    \missingfigure{cells with H3-PA-GFP. Activate a few circles and see distribution 1h later}
%    \missingfigure{see effect of different drugs on this movement}

    %% not homogeneous. Different parts of the chromatin should have different recoveries.

\section{Discussion: why histone dynamics is not measurable}

    %% FRAP is not goof for very long time frames. Reference papers where it was used incorrectly?
    %% see Tim's paper

\section{Conclusions}



