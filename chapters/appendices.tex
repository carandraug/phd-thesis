\appendix
\chapter{Solutions}
  \label{app:solutions}
  
  %% an alternative method to make this lists is with description environment
  %%
  %% \begin{description}
  %%   \item[Freezing media] \hfill \\
  %%     91\% FCS;
  %%     10\% DMSO
  %%   \item[Growth medium (HeLa)] \hfill \\
  %%     87\% DMEM;
  %%     10\% FCS;
  %%     1\% NEAA;
  %%     ...
  %% \end{description}
  
  %% This requires the eqparbox package. It checks what LaTeX thinks is best for a column
  %% and save that value. It will then use it to calculate what's left of \textwidth, and
  %% use it for the other columns. See http://tex.stackexchange.com/questions/95397
  %% It doesn't work with memoir's ctabular so we are using longtable instead
  \newsavebox{\firstentrybox}
  \newcolumntype{N}{%
    >{\begin{lrbox}{\firstentrybox}}%
      l%
    <{\end{lrbox}%
    \eqmakebox[firstentry][l]{\unhcopy\firstentrybox}}}

  \begin{longtable}{>{\bfseries}N p{\dimexpr(\textwidth-\eqboxwidth{firstentry}-4\tabcolsep)}}
    \toprule
    Name & Recipe\\
    \midrule
    2YT broth               & \SI{16}{\g\per\l} tryptone;
                              \SI{10}{\g\per\l} yeast extract;
                              \SI{5}{\g\per\l}  NaCl.\\
    
    DMEM                    & \SI{4.5}{\g\per\l}   glucose;
                              \SI{110}{\mg\per\l}  L-glutamine;
                              \SI{584}{\ug\per\l}  sodium pyruvate;
                              \SI{15.9}{\mg\per\l} phenol red.\\
    
    DNA loading buffer (10$\times$) & \pcent{25} Ficoll ($w/v$);
                              \SI{100}{\mM}      Tris--HCl pH=\num{7.4};
                              \SI{100}{\mM}      EDTA.\\
    
    Freezing media          & \pcent{90} FCS ($v/v$);
                              \pcent{10} DMSO ($v/v$).\\
    
    Growth medium (HeLa)    & \pcent{89}            DMEM ($v/v$);   % 500ml
                              \pcent{9}             FCS ($v/v$);    % 50ml
                              \SI{1}{$\times$}      NEAA solution;  % 5.5mL
                              \SI{50}{units\per\ml} penicillin;     % 5.5mL (Pen/Strep solution)
                              \SI{50}{\ug\per\ml}   streptomycin.\\ % 5.5mL (Pen/Strep solution)
    
    Growth medium (horse)   & \pcent{81}            DMEM ($v/v$);   % 500ml
                              \pcent{16}            FCS ($v/v$);    % 100ml
                              \SI{2}{$\times$}      NEAA solution;  %  12mL
                              \SI{50}{units\per\ml} penicillin;     % 5.5mL (Pen/Strep solution)
                              \SI{50}{\ug\per\ml}   streptomycin.\\ % 5.5mL (Pen/Strep solution)
    
    LB agar                 & \SI{20}{\g\per\l}  LB broth powder;
                              \SI{7.5}{\g\per\l} agar.\\
    
    LB broth                & \SI{20}{\g\per\l} LB broth powder.\\
    
    Ponceau S solution      & \pcent{5} Ponceau S ($w/v$);
                              \pcent{5} Acetic acid ($v/v$).\\
    
    Running buffer          & \SI{1}{$\times$} TG;
                              \pcent{0.1}      SDS ($w/v$).\\
    
    SSC (Saline Sodim Citrate) & \SI{150}{\mM} NaCl;
                              \SI{15}{\mM}     trisodium citrate.\\
    
    TAE (Tris Acetate EDTA) & \SI{40}{\mM} Tris;
                              \SI{20}{\mM} acetic acid;
                              \SI{1}{\mM}  EDTA.\\
    
    TBE (Tris Borate EDTA)  & \SI{89}{\mM} Tris;
                              \SI{89}{\mM} boric acid;
                              \SI{2}{\mM}  EDTA.\\
    
    TBS (Tris Buffered Saline)  & \SI{50}{\mM} Tris--HCl pH=\num{7.5};
                              \SI{100}{\mM}    NaCl.\\
    
    PBS-T (TBS-Tween)       & \pcent{99.95} PBS ($v/v$);
                              \pcent{0.05}  Tween 20 ($v/v$).\\
    
    TG (Tris Glycine)       & \SI{25}{\mM}  Tris;
                              \SI{192}{\mM} glycine.\\
    
    Transfer buffer         & \SI{1}{$\times$} TG;
                              \pcent{15} methanol ($v/v$).\\
    \bottomrule
  \end{longtable}
    

\chapter{List of plasmids}
  \label{app:plasmids}
  %% generate this automatically from SQlite db
  %% should we print the maps on genbank format, just the genbank features
  %% table, or a draw of the map, poiting to the database for the sequence
  %% details?

\chapter{List of primers}
  \label{app:primers}
  %% generate this automatically from SQlite db

\chapter{Documentation}
  \label{app:pod-doc}
  %% this setlist commands require the enumitem package and avoids that the very long item entries
  %% on the description environment (as it happens on the examples section), do not run out of the
  %% paper margins. See http://tex.stackexchange.com/questions/96267
  \setlist[description]{style=unboxed}
  \input{results/bp_genbank_ref_extractor}
  \setlist[description]{style=standard}
