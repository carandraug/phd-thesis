\chapter{Materials and Methods}
\label{ch:methods}

\section{Chemicals, solutions and reagents used}
  The chemicals used were obtained from Sigma unless otherwise stated. Solutions were prepared
  according to \Aref{app:solutions} with dH$_2$O and autoclaved prior to use when appropriate.
  
  Restriction enzymes, T4 DNA ligase, and other DNA modifying enzymes, DNA ladders were obtained
  from New England BioLabs. Protein ladder was SeeBlue Plus2 from Invitrogen.
  
\section{DNA methods}
  \subsection{Preparation of competent bacteria}
    Competent \species{E.~coli} cells were prepared from a culture of Invitrogen's One Shot
    TOP10 Chemically Competent \species{E.~coli}. LB cultures of \SI{1}{\l} were set at
    \dc{37} until an OD$_{\SI{600}{\nm}}$ of \numrange{0.4}{0.5}. Further steps were carried
    at \dc{4} with previously chilled equipment and solutions.
    
    Cultures were centrifuged at \SI{6000}{\gn} for 10 minutes, the pellet resuspended in
    \SI{500}{\ml} of \SI{0.1}{\mM} CaCl$_2$, and incubated on ice for 30 minutes. The suspensions
    were centrifuged again at \SI{6000}{\gn} for 10 minutes, and the new pellet resuspended in
    \SI{100}{\ml} of CaCl$_2$ with \pcent{15} glycerol. Aliquots of competent cells were prepared
    and stored at \dc{-80}.
    
    Transformation efficiencies were measured after preparation of each batch and discarded
    if less than \SI{1d6}{\cfu\per\mg} of plasmid was obtained. Absence of antibiotic-resistant
    contaminations was assessed by streaking the cells on selective plates.

  \subsection{Transformation of competent cells}
    Competent cells were thawed on ice and split into aliquots of \SI{50}{\ul} to pre-chilled \SI{2}{\ml}
    tubes where \SI{1}{\ul} DNA was added. Cells were incubated on ice for 30 minutes, followed by a 60 seconds
    heat-shock at \dc{42}, and 5 more minutes on ice. \SI{300}{\ul} of non-selective LB was added to each tube
    and the cultures incubated at \dc{37} with vigorous shaking for 45 minutes. Samples from the cultures
    were plated onto the appropriate antibiotic containing plates, and incubated overnight at \dc{37}.
    
    %% FIXME add concentration of antibiotic used for both agar and broth
    
    For high concentrations of plasmid DNA (more than \SI{500}{\ng\per\ug}), smaller volumes of DNA were used,
    and the initial and final incubation steps were lowered to 10 minutes.

  \subsection{Plasmid DNA preparation}
    Plasmid DNA was prepared with kits from QIAGEN (QIAprep Spin Miniprep, QIAGEN Plasmid \textit{Plus} Midi,
    QIAquick Gel Extraction and QIAquick PCR purification) following the manufacturer's instructions. Once
    prepared, DNA was stored at \dc{-20}. DNA concentrations were measured with a spectrophotometer (NanoDrop
    1000 Spectrophotometer from Thermo Scientific).
    %% FIXME confirm nanodrop model

  \subsection{Ethanol precipitation}
    \label{sec:ethanol-precipitation}
    The DNA solution was mixed with \num{2.5} volumes of \pcent{100} ethanol and \num{1/10} volumes of
    Sodium Acetate (\SI{3}{\Molar}, pH=\num{5.2}), and incubated at \dc{4} for 15 minutes (overnight for
    low DNA plasmid concentrations). Solution was centrifuged at \SI{18000}{\gn} for 30 minutes at \dc{4},
    the supernatant discarded, and the pellet left to dry until all traces of solvent evaporated. DNA pellet
    was resuspended in the desired solvent (usually H$_2$O for transfection).

  \subsection{Agarose gels electrophoresis}
    Agarose gels with of concentrations ranging from \SIrange{0.6}{2.0}{\percent} were prepared with TAE buffer,
    and supplemented with ethidium bromide. Gels were ran in chambers with \SI{1}{$\times$}~TAE buffer at
    \SIrange{80}{120}{\volt} until the required separation was achieved. Gels were visualized at a UV
    transilluminator (ChemiImager 5500 from Alpha Innotech).

  \subsection{DNA sequencing and oligonucleotide preparation}
    DNA sequencing was performed by LGC Genomics after cloning for confirmation and avoid unexpected mutations.
    
    Oligonucleotides were ordered from Eurofins MWG operon in lyophilised format, dissolved in H$_2$O to a
    \SI{100}{\micro\Molar} concentration, and stored at \dc{-20}. A list of all designed oligonucleotides is
    produced at \Aref{app:primers}.

  \subsection{Polimerase Chain Reaction}
    \subsubsection{Colony PCR}
      When screening multiple clones for a specific plasmid and a set of appropriate oligonucleotides was available,
      PCRs were set directly from the bacteria colonies by adding them directly to the reaction mixture
      (\tref{tab:pcr-mixture}). Bacteria from individual colonies were used to simultaneously perform a PCR and
      start a small culture. Plasmid purification was performed on cultures whose sample was positive by PCR.

    \subsubsection{Gene amplification}
    \subsubsection{Mutagenesis}
    \subsubsection{Screening}
    
    \begin{table}
      %% FIXME I should be using subcaptions here
      \centering
      \captionIntro{PCR mixtures.}{}
      \label{tab:pcr-mixture}
      \todo[inline]{get the actual values used here}
      \begin{tabular}{l c c c c}
        \toprule
        \null                                 & colony  & amplification & mutagenesis & screening \\
        \midrule
        Template (\si{\ng})                   & 6       & 67            & 78          & 89    \\
        DMSO                                  & 5       & 78            & 78          & 89    \\
        Buffer                                & 6       &     67        & 78          & 89    \\
        MgSO$_4$                              & 67      &   78          & 78          & 89    \\
        dNTPs                                 & 8       &   78          & 78          & 89    \\
        Primer (forward) (\si{\micro\Molar})  & 8       &   78          & 78          & 89    \\
        Primer (reverse) (\si{\micro\Molar})  & 9       &   78          & 78          & 89    \\
        DNA polymerase                        & 9       &   78          & 78          & 89    \\
        \addlinespace
        Total volume (\si{\ul})               & 25      &   78          & 78          & 89    \\
        \bottomrule
      \end{tabular}

      \begin{threeparttable}
        \captionIntro{PCR conditions.}{}
        \label{tab:pcr-conditions}
        \begin{tabular}{l r<{\si{\second}}@{ at }l<{\si{\degreeCelsius}} r<{\si{\second}}@{ at }l<{\si{\degreeCelsius}}
                          r<{\si{\second}}@{ at }l<{\si{\degreeCelsius}} r<{\si{\second}}@{ at }l<{\si{\degreeCelsius}}}
          \toprule
          \null               & \multicolumn{2}{c}{colony}      & \multicolumn{2}{c}{amplification} &
                                \multicolumn{2}{c}{mutagenesis} & \multicolumn{2}{c}{screening} \\
          \midrule
          Initialization      & 180 & 94                   & 60  & 94                   & 60  & 94                   & 60  & 94                   \\
          Denaturation        & 15  & 94                   & 60  & 94                   & 60  & 94                   & 60  & 94                   \\
          Annealing \tnote{1} & 15  & 94                   & 60  & 94                   & 60  & 94                   & 60  & 94                   \\
          Extension \tnote{2} & 60  & 94                   & 60  & 94                   & 60  & 94                   & 60  & 94                   \\
          Number of cycles    & \multicolumn{2}{c}{30}     & \multicolumn{2}{c}{30}     & \multicolumn{2}{c}{30}     & \multicolumn{2}{c}{30}     \\
          Final extension     & 300 & 72                   & 60  & 94                   & 60  & 94                   & 60  & 94                   \\
          Final hold          & \multicolumn{2}{c}{\dc{4}} & \multicolumn{2}{c}{\dc{4}} & \multicolumn{2}{c}{\dc{4}} & \multicolumn{2}{c}{\dc{4}} \\
          \bottomrule
        \end{tabular}
        \begin{tablenotes}
          \item [1] Temperature is primer dependent. Displayed values correspond to the typical usage.
          \item [2] Time is dependent on sequence length. Displayed time is by \si{\kilo\bp}.
        \end{tablenotes}
      \end{threeparttable}
    \end{table}



\section{Protein methods}
  \subsection{Phenol:chloroform extraction}
    \label{sec:phenol-extraction}
    To extract proteins, an equal volume of phenol:chloroform was added and the
    mixture centrifuged at \SI{6000}{\gn} for 15 minutes. The top aqueous phase (chloroform)
    was pipetted to a new tube and the process repeated a total of 3 times.



  \subsection{Western blotting}
    \subsubsection{Protein concentration determination}
      Concentration of protein was measured with Bradford reagent. \SI{2}{\ul} of the
      sample after sonication (\Sref{sec:cell-extract}) was mixed with \SI{48}{\ul} of H$_2$O
      and \SI{50}{\ul} of NaOH and incubated at \dc{65} for 8 minutes before adding
      \SI{900}{\ul} of Bradford reagent from Pierce. The mixture was transferred to plastic
      cuvettes and the absorvance at \SI{595}{\nm} measured in a Shimadzu spectrophotometer. The
      values obtained were interpolated from a standard curve prepared using known concentrations
      of BSA.
    
    \subsubsection{SDS--PAGE}
      Resolving and stacking SDS--PAGE gels of \SIrange{15}{5}{\percent} respectively, both with a
      cross-linking ratio of \num{37.5}:1. The resolving gel was poured directly after addition
      of TEMED and it was covered with a layer of isopropanol during polymerisation
      to ensure a sharp interface between the resolving and stacking layers.
      Protein samples and markers were boiled at \dc{99} for 3 minutes and each
      was loaded twice, with volumes for \SI{3.3}{\ug} and \SI{16.5}{\ug} of protein. Gels ran at
      \SI{180}{\volt} for 1 hour in \SI{1}{$\times$} TG buffer.
    
    \subsubsection{Protein transfer}
      Protein transfer occurred through the wet transfer system. The SDS-PAGE gel
      was placed onto pre-cut nitrocellulose transfer membrane previously soaked in
      transfer buffer. It was then set between a pair of extra thick blotting paper and
      cushions before being placed inside a transfer apparatus. The transfer ran at
      \dc{4} for 60 minutes.
    
    \subsubsection{Probing of blot with antibody}
      Blocking of the membrane was performed with \SI{10}{\percent} non-fat dry milk in \SI{1}{$\times$} TBST
      at room temperature for 30 minutes. Blocking was followed by primary antibody
      incubation which occurred in \SI{5}{\percent} non-fat dry milk in \SI{1}{$\times$} TBST overnight at
      \dc{4}. Concentrations of antibody used were 1:500 and 1:20000 for anti-GFP
      (catalogue number 11 814 460 001 from Roche) and anti-H3 (code ab1791 from
      abcam). The membrane was then washed with \SI{1}{$\times$} TBST for 15 minutes 3 times
      before the secondary antibody incubation which occurred in \SI{5}{\percent} non-fat dry
      milk in \SI{1}{$\times$} TBST for 1 hour. The membrane was washed once more in the same
      conditions as before for the detection. All blocking, antibody incubation and
      washing steps occurred on a rocker.\todo{should list antibodies and their conditions on appendix}
      
      Detection was performed using the SuperSignal West Pico Chemiluminescent
      Substrate from Pierce, adding 1:1 of the solutions and allowing it to incubate
      with the membrane for 5 minutes. The membrane was exposed to x-ray films for
      10, 60, 5, 180 and 1800 seconds which were then developed.


\section{Tissue culture}
  \subsection{Cell culture}
    HeLa cells were supplied by Agnieszka Kaczmarczyk from National University
    of Ireland, Galway. They were maintained at \dc{37} and \pcent{5} CO$_2$ in \SI{10}{\cm}
    diameter plates with \SI{10}{\ml} of growth medium. Dulbecco's Phosphate Buffered
    Saline (DPBS) and trypsin--EDTA solutions were used, respectively, to wash
    and split the cells 1:10 once they reached a confluence of \SIrange{80}{90}{\percent}.

  \subsection{Cell stock storage}
    For long-term storage of HeLa cell lines, they were grown until they reached
    a confluence of \SIrange{80}{90}{\percent} and then trypsinized as usual. A volume of Freezing
    Media was added, equal to the volume of trypsin--EDTA, and \SI{2}{\ml} aliquots of
    cells transferred to cryotubes. Tubes were immediately wrapped in cotton and
    placed at \dc{-80}.

  \subsection{Whole cell extract}
    \label{sec:cell-extract}
    To obtain whole cell extracts, HeLa cells were trypsinized as usual. Growth
    medium added and the suspension was centrifuged at \SI{900}{\gn} for 10 minutes. Further
    steps were carried at \dc{4} and with previously chilled reagents. The supernatant
    was discarded and the pellet resuspended in \SI{500}{\ul} of chilled PBS before being
    sonicated 3 times at \pcent{40} amplitude for 10 seconds. Avoiding the foam formed at
    the top, \SI{300}{\ul} of suspension were transferred from the bottom of the tube to a
    new eppendorf and mixed with an equal volume of Laemmli buffer before being
    stored at \dc{-80}. \SI{2}{\ul} from the suspension was also transferred to a new tube
    for determination of protein concentration.

  \subsection{Genomic DNA extraction}
    To extract genomic DNA of HeLa cells, they were trypsinized as usual and
    growth medium was added in the end before counting with an hemocytometer.
    Cells were centrifuged at \SI{1500}{\gn} for 10 minutes at \dc{4}, the supernatant discarded
    and the pellet resuspended in TE buffer to achieve a desired concentration of
    \SI{4e7}{cells\per\ml}. 9 volumes of Genomic lysis buffer was added and the mixture
    incubated at \dc{37} for 90 minutes. Proteinase K was added to a final concentration of
    \SI{100}{\ug\per\ml} and the mixture incubated at 50 °C for 3 h and swirled every 20 min.
    DNA was then extracted by phenol:chloroform (\Sref{sec:phenol-extraction}) and purified by ethanol
    precipitation (\Sref{sec:ethanol-precipitation}).

  \subsection{Viable cells count}
    Trypan blue was used to assess the number of viable cells. After trypsination
    as usual, cells were diluted in growth medium, to a final concentration of \SI{2e5}{cells\per\ml}.
    To \SI{0.5}{\ml} of the cell suspension, was added \SI{0.1}{\ml} of \pcent{0.4} Trypan
    Blue Stain and the mixture left for 5 minutes at room temperature before counting
    the cells in an hemocytometer making a distinction between stained (non-viable)
    and non-stained (viable) cells.

  \subsection{Transfection of HeLa cells}
    Cells were transfected using Lipofectamine 2000, a cationic lipid reagent, from
    Invitrogen. Cells were trypsinized as usual on the day before transfection and
    replated on 6 well plates (surface area of \SI{9.5}{\square\cm\per well}) with \SI{2.5}{\ml} of growth
    medium so they would be \SI{90}{\percent} confluent on the following day. For each well,
    two tubes with \SI{250}{\ul} of transfection medium were prepared, one with \SI{7.5}{\ul}
    of Lipofectamine 2000 and another with \SI{3750}{\ng} of DNA (ethanol precipitation
    (\Sref{sec:ethanol-precipitation}) was used to increase the DNA concentration for values around \SI{500}{\ng\per\ul}
    before transfection). Both tubes were incubated at room temperature for 5 minutes,
    mixed together, and incubated again at room temperature for 20 minutes. Cells
    were washed with DPBS during this time and growth medium switched to \SI{2}{\ml}
    of transfection medium. The mixture was then added to the cells medium who
    were incubated at \dc{37} for 6 hours after which time it was switched back to
    \SI{0.5}{\ml} of growth medium.

  \subsection{Fixation and staining of HeLa cells}
    For microscopy visualization, cells grown directly on top of HCl washed coverslips since
    HeLa cells have difficulty attaching to glass. At least 24 hours passed
    between the plating and fixation. Growth medium was removed and the cells
    washed with PBS once before incubation with \SI{4}{\percent} formaldehyde in PBS
    for 4 minutes. The solution was removed and the cells washed with H$_2$O 2 more
    times, after which coverslips were removed from the wells and left to air dry.
    For each coverslip, \SI{2}{\ul} of SlowFade Light Antifade kit from Molecular Probes
    was used for mounting the coverslip on a microscope slide. DAPI was added
    to the mounting media when needed. Coverslips were then sealed with a 1:1
    mixture of clear nail polish and acetone and stored on a dark box at \dc{4}.


\section{Microscopy}
\section{Software used}
  %% we probably should use a script to get the actual version when building the document
  %% this versions numbers are the ones I'm using at the moment... 4 years ago, they didn't exist
  %% must redo the analysis and get the new version numbers.
  The European Molecular Biology Open Software Suite (EMBOSS) version 6.4.0
  was used for analysis of codon usage, RNA folding, sequence alignment, reading of abi
  files, search for restriction sites, prediction of molecular weight and
  other trivial tasks.

  The Perl language and the BioPerl module version 1.6.901 was also extensively used for automation of
  several tasks.
  
  GNU Octave version 3.6.2 and the image package version 2.0.0.
  
  ImageJ version 1.47h packaged though FIJI.

\section{Software developed}




