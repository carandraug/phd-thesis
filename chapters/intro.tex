\chapter{Introduction}
\label{ch:intro}

%% FIXME 2nd year report abstract
%%The nucleosome is the fundamental structural unit of chromatin and the foundation
%%of it dynamics. Certain mutants have been shown to have a large effect on the
%%stability of the nucleosome \textit{in vitro} and to mimic chromatin remodelling
%%complexes in yeast.
%%
%%Fluorescence Recovery After Photobleaching (FRAP) has been previously used to
%%measure kinetics of endogenous histones. We are attempting to use it to measure
%%differences between wild type histones and mutants showing unusual \textit{in vitro} dynamics.
%%
%%We have developed two programs on the GNU Octave programming language, CropReg and
%%FRAPINATOR which was released under GPL. CropReg is able to track the nuclei of moving
%%cells. FRAPINATOR is able to do all the image processing, data extraction and fitting
%%of our FRAP experiments. These have revealed significant challenges to the determination
%%of relevant kinetic parameters, principally due to movement of chromatin.







\section{Chromatin organization}
  Chromatin is a dynamic complex that controls access to genetic information by
  undergoing reconfiguration of its structure. Since nucleosomes are the chromatin
  basic structural unit, their structural properties are on the basis of such reconfigurations.

  \subsection{Chromatin as substrate}
  \subsection{Nucleosome structure}
  \subsection{Chromatin remodelling}

\section{Centromeres --- specialized chromatin}
  \subsection{Centromere and kinetochore structure}
    \subsubsection{Timing of assembly}
  \subsection{Histone Fold Domain}
    \subsubsection{CENP--T/W}
    \subsubsection{CENP--S/X}

\section{In vivo techniques for quantitative protein dynamics}
  \subsection{Fluorescent proteins}
  \subsection{FRAP}

    Fluorescence Recovery After Photobleaching has been extensively used to obtain
    qualitative and quantitative insight on the kinetic properties of proteins. Development of
    this technique has led to complex models that are both more precise and accurate than
    simple models based on inverse exponential decay. These take into account important
    parameters that were previously discarded but have since been shown as important, such as
    diffusion and the bleach spot profile shape.

    However, despite the their sophistication, these models of recovery make certain
    important assumptions:
    \begin{itemize}
      \item the biological system has reached equilibrium before photobleaching;
      \item total amount of both Fluorescent Protein (FP) fusion protein and its
            binding sites remain constant over the time course of the recovery;
      \item the binding sites are part of a large, relatively immobile complex, at
            least on the time- and length-scale of the recovery.
    \end{itemize}

    FRAP has been successfully used before to show that different core histones have
    different kinetics and populations. We expect to be able to use it for \textit{in vivo}
    validation of previously identified differences between wild type and mutant histones.

  \subsection{iFRAP}
  \label{sec:iFRAP}

  \subsection{FLAP}











