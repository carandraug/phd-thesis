The building block of eukaryotic chromatin structure is the nucleosome, comprising
\SI{147}{\bp} of DNA wrapped around an octamer of two copies of core histones H2A,
H2B, H3 and H4. Nucleosomes are arranged in a linear chain separated by DNA linkers, and
can be further compacted into higher order chromatin structures. But chromatin extends
well beyond DNA compaction, and is a dynamic complex that controls access to genetic
information by undergoing reconfiguration of its structure.

Since nucleosomes are the basic structural unit of chromatin, it is the interactions
within these structures that determine its configuration. Local
reconfigurations of chromatin can be accomplished by changing nucleosome structure or altering
its composition by post-translational modifications or incorporation of variant
histones.

These changes and their influence on access to genetic information are usually performed
by protein complexes\todo{???}. Mutations of the histones can relieve the need of these complexes
by affecting the stability of the nucleosomes so it is possible to gain insight on
the mechanism of these complexes by studying the kinetic alterations these mutants create.

Studies of this mutations have all been carried on \textit{in vivo}\todo{???}{} or yeast. We attempted
to validate these effects in the context of chromatin by using live cell imaging.

\section{The FRAP model}

  Fluorescence recovery after photobleaching (FRAP) is an optical technique
  that reveals the dynamics of fluorescently tagged molecules within live cells.
  The tagged molecules inside a small region are irreversibly photobleached by
  action of a high-power focused laser beam and the recovery rate of fluorescence
  is measured. The recovery rate is interpreted as unbleached molecules,
  which were outside of the region at the time of photobleaching, moving into
  the bleached area, replacing the bleached molecules. It is assumed that the
  the fluorescence recovery reflects the protein natural movement.

  This technique has been extensively used to obtain qualitative and quantitative
  insight on the kinetic properties of proteins. Development of this technique has
  led to complex models that are both more precise and accurate than simple models
  based on inverse exponential decay. These take into account important parameters
  that were previously discarded but have since been shown as important, such
  as diffusion and the bleach spot profile shape.

  However, despite their sophistication, these models of recovery require certain important
  assumptions that are difficult to maintain for long experimental observation times:

  \begin{itemize}
    \item the biological system has reached equilibrium before photobleaching;
    \item total amount of both Fluorescent Protein (FP) fusion protein and its
          binding sites remain constant over the time course of the recovery;
    %% FIXME we should probably group the 2 before into one
    \item distribution of tagged molecule mimics the endogenous protein;
    \item the binding sites are part of a large, relatively immobile complex, at
          least on the time and length scale of the recovery.
  \end{itemize}

  FRAP has been successfully used before to show that different core histones have
  different kinetics and populations. We expected to be able to use the same approach
  for \textit{in vivo} validation of previously identified differences between
  wild type and mutant histones.

\todo[inline]{where do we mention FRAPINATOR? That goes in the Octave chapter but we still need to
              reference it from here somewhere.}

