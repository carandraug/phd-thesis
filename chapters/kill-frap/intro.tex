\section{Introduction}

  \subsection{Chromatin remodelling and nucleosome structure}

    The building block of eukaryotic chromatin structure is the nucleosome, comprising
    \SI{147}{\bp} of DNA wrapped around an octamer of two copies of core histones H2A,
    H2B, H3, and H4. Nucleosomes are arranged in a linear chain separated by DNA linkers, and
    can be further compacted into higher order chromatin structures. But chromatin extends
    well beyond DNA compaction, it is a dynamic complex that controls access to genetic
    information by undergoing reconfiguration of its structure.

    As the basic structural unit of chromatin, intra-nucleosome interactions
    are the lowest possible level of chromatin configuration. Local reconfiguration
    of chromatin can be achieved by changing nucleosome structure or altering
    its composition. Either via post-translational modifications or incorporation
    of histone variants, nucleosomes may gain different DNA sequence preferences
    or recruit other proteins ultimately changing the chromatin structure.

    Alternatively, chaperones and ATP-dependent chromatin remodelling complexes
    act extrinsically in nucleosome position. One of these is the SWI/SNF~complex
    whose deficiency causes growth defects in yeast. While the exact details of
    its mechanism are not yet known, the contenders being twist defect and
    bulge diffusion, this complex destabilizes the interactions between the
    nucleosome and DNA, causing it to shift position in the DNA sequence.

  \subsection{SIN mutants}

    There is a set of mutations able to compensate the loss of the SWI/SNF
    complex. This set is collectively known as SIN mutations, so named because
    they provide SWI/SNF INdependence. A subset of these are single
    amino-acid changes in the histones sequence, thus suggesting a very
    attractive hypothesis that not only are these locations involved in the
    mechanism of SWI/SNF, but are also of major importance in the nucleosome
    structure. Indeed, this hypothesis has been tested \textit{in vitro}
    where SIN mutant nucleosomes display higher thermal mobility.

  \subsection{FRAP}

    Fluorescence Recovery After Photobleaching (FRAP) is an optical technique
    that reveals the dynamics of fluorescently tagged molecules within live cells.
    The tagged molecules inside a small region are irreversibly photobleached by
    action of a high-power focused laser beam and the recovery rate of fluorescence
    is measured. The recovery rate is interpreted as unbleached molecules,
    which were outside of the region at the time of photobleaching, dissociating and
    diffusing into the bleached area, replacing the bleached molecules. It is assumed that the
    the fluorescence recovery reflects the protein natural movement.

    This technique has been extensively used to obtain qualitative and quantitative
    insight on the kinetic properties of proteins. Among these are also
    histone proteins, where FRAP was used to compare exchange ratios and
    soluble pools of different histones types and variants. These results
    show extremely slow recovery rates, with half-lives longer than 8~hours.

    %%FIXME we should probably be more descriptive of Kimura and Cook results

    Continuous development of FRAP has
    led to increasingly complex models that are both more precise and accurate than simple models
    based on inverse exponential decay.
    However, despite their sophistication, these models of recovery require certain important
    assumptions that are difficult to maintain for long experimental observation times:

    \begin{itemize}
      %% FIXME the first one is not really a problem related to long time FRAP
      \item the biological system has reached equilibrium before photobleaching;
      \item total amount of both Fluorescent Protein (FP) fusion protein and its
            binding sites remain constant over the time course of the recovery;
      %% FIXME we should probably group the 2 before into one
      \item distribution of tagged molecule mimics the endogenous protein;
      \item the binding sites are part of a large, relatively immobile complex, at
            least on the time and length scale of the recovery.
    \end{itemize}

  \subsection{Objectives}

    We aim to develop a technique capable of measuring subtle kinetic
    alterations of the nucleosome in live cells for study of the
    structure--function relationship of the nucleosome.
    Starting with the histone SIN mutant H4~R45H, know \textit{in vitro} to
    cause the highest increase in mobility, we can test FRAP for this
    corner-cases while validating \textit{in vivo} the previous results.

    These results lead to more detailed model of the nucleosome, allowing
    to predict effects of extra mutations which can be feed back to this
    technique for testing. A continuous cycle of this approach will then lead
    to continuous refining of the nucleosome model.

