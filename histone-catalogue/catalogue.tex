  \section{Build information}
  \begin{description}
    \item[Data accession date] \hfill \newline
      \printdate{\SequencesDate}
    \item[Build date] \hfill \newline
      \today
    \item[BioPerl version] \hfill \newline
      \BioPerlVersion{}
    \item[Bio-EUtilities version] \hfill \newline
      \BioEUtilitiesVersion
    \item[T-Coffee version] \hfill \newline
      \TCoffeVersion
    \item[WebLogo version] \hfill \newline
      \WebLogoVersion
  \end{description}

  \section{Count of histone genes by cluster and type}
  \begin{tabular}{l *{5}{r!{+}r<{$\psi$}}}
  \toprule
  \null   & \multicolumn{2}{c}{H2A}  & \multicolumn{2}{c}{H2B}
          & \multicolumn{2}{c}{H3}   & \multicolumn{2}{c}{H4}
          & \multicolumn{2}{c}{Total} \\
  \midrule
  HIST1   & \HTwoACodingInHISTOne{}     & \HTwoAPseudoInHISTOne{}
          & \HTwoBCodingInHISTOne{}     & \HTwoBPseudoInHISTOne{}
          & \HThreeCodingInHISTOne{}    & \HThreePseudoInHISTOne{}
          & \HFourCodingInHISTOne{}     & \HFourPseudoInHISTOne{}
          & \CoreCodingGenesInHISTOne{} & \CorePseudoGenesInHISTOne{} \\
  HIST2   & \HTwoACodingInHISTTwo{}     & \HTwoAPseudoInHISTTwo{}
          & \HTwoBCodingInHISTTwo{}     & \HTwoBPseudoInHISTTwo{}
          & \HThreeCodingInHISTTwo{}    & \HThreePseudoInHISTTwo{}
          & \HFourCodingInHISTTwo{}     & \HFourPseudoInHISTTwo{}
          & \CoreCodingGenesInHISTTwo{} & \CorePseudoGenesInHISTTwo{} \\
  HIST3   & \HTwoACodingInHISTThree{}   & \HTwoAPseudoInHISTThree{}
          & \HTwoBCodingInHISTThree{}   & \HTwoBPseudoInHISTThree{}
          & \HThreeCodingInHISTThree{}  & \HThreePseudoInHISTThree{}
          & \HFourCodingInHISTThree{}   & \HFourPseudoInHISTThree{}
          & \CoreCodingGenesInHISTThree{} & \CorePseudoGenesInHISTThree{} \\
  HIST4   & \HTwoACodingInHISTFour{}    & \HTwoAPseudoInHISTFour{}
          & \HTwoBCodingInHISTFour{}    & \HTwoBPseudoInHISTFour{}
          & \HThreeCodingInHISTFour{}   & \HThreePseudoInHISTFour{}
          & \HFourCodingInHISTFour{}    & \HFourPseudoInHISTFour{}
          & \CoreCodingGenesInHISTFour{} & \CorePseudoGenesInHISTFour{} \\
  \addlinespace
  Total   & \HTwoACodingGenes{}       & \HTwoAPseudoGenes{}
          & \HTwoBCodingGenes{}       & \HTwoBPseudoGenes{}
          & \HThreeCodingGenes{}      & \HThreePseudoGenes{}
          & \HFourCodingGenes{}       & \HFourPseudoGenes{}
          & \TotalCoreCodingGenes{}   & \TotalCorePseudoGenes{} \\
  \bottomrule
\end{tabular}



  \newpage
  \section{Catalogue of canonical core histone genes and products}
  \input{\ResultsDir/table-histone_catalogue}

  %% We don't really need a section* for the following, it's already
  %% on the table.
  \newpage
  \input{\ResultsDir/table-H2A-proteins-align}
  \includegraphics[width=\textwidth]{\FigsDir/seqlogo_H2A_proteins.pdf}

  \newpage
  \input{\ResultsDir/table-H2B-proteins-align}
  \includegraphics[width=\textwidth]{\FigsDir/seqlogo_H2B_proteins.pdf}

  \newpage
  \input{\ResultsDir/table-H3-proteins-align}
  \includegraphics[width=\textwidth]{\FigsDir/seqlogo_H3_proteins.pdf}

  \newpage
  \input{\ResultsDir/table-H4-proteins-align}
  \includegraphics[width=\textwidth]{\FigsDir/seqlogo_H4_proteins.pdf}

  \newpage
  \section{Catalogue of variant core histone genes and products}
  \input{\ResultsDir/table-variant_catalogue}

  \newpage
  \section{List of current curation anomalies}
  \begin{itemize}
  \input{\ResultsDir/histone_insanities}
  \end{itemize}
