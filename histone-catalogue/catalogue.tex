  \section{Build information}
  \begin{description}
    \item[Data accession date] \hfill \newline
      \printdate{\SequencesDate}
    \item[Build date] \hfill \newline
      \today
    \item[BioPerl version] \hfill \newline
      \BioPerlVersion{}
    \item[Bio-EUtilities version] \hfill \newline
      \BioEUtilitiesVersion
    \item[T-Coffee version] \hfill \newline
      \TCoffeVersion
    \item[WebLogo version] \hfill \newline
      \WebLogoVersion
  \end{description}

  \section{Count of histone genes by cluster and type}
  \input{sections/table-histone_count}

  \newpage
  \section{Catalogue of canonical core histone genes and products}
  \input{\ResultsDir/table-histone_catalogue}

  %% We don't really need a section* for the following, it's already
  %% on the table.
  \newpage
  \input{\ResultsDir/table-H2A-proteins-align}
  \includegraphics[width=\textwidth]{\FigsDir/seqlogo_H2A_proteins.pdf}

  \newpage
  \input{\ResultsDir/table-H2B-proteins-align}
  \includegraphics[width=\textwidth]{\FigsDir/seqlogo_H2B_proteins.pdf}

  \newpage
  \input{\ResultsDir/table-H3-proteins-align}
  \includegraphics[width=\textwidth]{\FigsDir/seqlogo_H3_proteins.pdf}

  \newpage
  \input{\ResultsDir/table-H4-proteins-align}
  \includegraphics[width=\textwidth]{\FigsDir/seqlogo_H4_proteins.pdf}

  \newpage
  \section{Catalogue of variant core histone genes and products}
  \input{\ResultsDir/table-variant_catalogue}

  \newpage
  \section{List of current curation anomalies}
  \begin{itemize}
  \input{\ResultsDir/histone_insanities}
  \end{itemize}
