\documentclass[a4paper,oneside,onecolumn,article,draft]{memoir}
\counterwithout{section}{chapter}

%% We need to follow NUIG structure and style in the thesis.  They
%% tell us how to do it but they don't really give a template or latex
%% document class to use.  I guess being able to format a document is
%% part.  Follows their documentation retrieved from ``University
%% Guidelines for Research Degree Programmes --- For Research
%% Students, Supervisors and Staff'', July 2016 edition, retrieved on
%% Mon 6 Feb 23:05:05 GMT 2017 from
%% HTTP://www.nuigalway.ie/media/graduatestudies/files/university_guidelines_for_research_degree_programmes.pdf
%%
%% From Section 6 - The PhD Examination Process
%%
%% 6.2.3 Directions on Format, Layout and Presentation
%%
%% The PhD thesis should not normally exceed 80,000 words, inclusive
%% of appendices, footnotes, tables and bibliography.  It is
%% university policy that the practice of engaging professional
%% editorial services to assist in writing the thesis is not
%% permitted.  There must be a title page which shall contain the
%% following information:
%%
%%  a. The full title (and subtitle, if any)
%%  b. The volume number and total number of volumes, if more than one
%%  c. The full name of the candidate, followed, if desired, by any
%%     degree and/or professional qualification(s)
%%  d. The name(s) of the supervisor(s), School(s), component
%%     Discipline(s), Institution
%%  e. The month and year of submission.
%%
%% Format and Layout
%%
%% The 'Table of Contents', which should not be over-detailed, shall
%% immediately follow the title page.  The text must be printed on good
%% quality (110g/m2) A4 size paper.  Line-spacing should be a maximum
%% of one-and-half; text must be left justified with a left-hand
%% margin of 4 cm and may be right justified.  An easily-readable
%% layout and double-sided printing are recommended for the body text.
%% For double sided printing ensure that the right hand margin is also
%% adequate for binding (i.e. a margin of 4 cm).  More compact formats,
%% with smaller font sizes, are usually appropriate for certain
%% sections, such as reference lists, bibliographies and some kinds of
%% appendices.  Pages must be numbered consecutively, with page numbers
%% located centrally at the bottom, and chapter headers at the top, of
%% each page.  Diagrams, graphs, photographs and tables should be
%% properly numbered and located in relation to the text.  The copies
%% of the thesis presented initially for examination must be spiral or
%% gum-bound.
%%
%% This is pretty much repeated in Appendix 1 --- Regulations for
%% Higher Research Degrees, Section 10 Submission of the Thesis.

\chapterstyle{veelo}

\OnehalfSpacing

\makepagestyle{NUIG}
\makeevenfoot{NUIG}{}{\thepage}{} % page numbers in the centre
\makeoddfoot{NUIG}{}{\thepage}{} % page numbers at the centre
\makeevenhead{NUIG}{\leftmark}{}{} % page marks in the edge
\makeoddhead{NUIG}{}{}{\rightmark} % page marks in the edge
\pagestyle{NUIG}

%% NUIG style requires 4cm for the spine margine (actually, it
%% requires 4 cm on the left margin, and 4 cm on the right margin if
%% it is to be double sided so that is is ``adequate for binding''.
%% Sounds like what matters is the spine margin and not right or left
%% margin).  Anyway, for an A4 page with font size 10pt (the default),
%% the memoir class default to ~3.5cm on the spine, and ~5.5cm on the
%% other side.  I kinda like the default format and text width so if
%% we add .5cm on one side, I'm taking it back from the other side.
%% Those default margins are computed based on the font size so the
%% text ends with roughly 66 characters per line.  Check the margin
%% sizes again if we have to change the font size.
\setlrmarginsandblock{4cm}{5cm}{*}
\checkandfixthelayout

%% remove colorlinks option when ready for print
\usepackage[final,hyperindex,hyperfootnotes,bookmarksnumbered,colorlinks]{hyperref}

\usepackage[T1]{fontenc}
\usepackage[utf8]{inputenc}
\usepackage{textcomp}

\usepackage{palatino}
\usepackage[euler]{textgreek}

\maxtocdepth{subsection}

\usepackage[final]{graphicx}

%% Input files that input others relative to themselves instead of
%% relative to the initial tex file.  Handy for each chapter but an
%% absolute requirement to include the pdf_tex figures from inkscape
%% which call includegraphics with the filename only.
\usepackage{import}

\usepackage{amsmath}

\usepackage[textsize=footnotesize]{todonotes}
  %% new command for box about missing references
  \newcommand{\addref}[1]{\todo[color=red!40,size=\tiny]{Add reference: #1}}

\usepackage{enumitem}       % so we can use the unboxed style when item names are too long
\usepackage{longtable}      % because memoir's ctabular does not work well with eqparbox
\usepackage{eqparbox}       % for adjusting size of table column (specially on appendices)
  %% Checks what LaTeX thinks is best for a column and save that value.
  %% It will then use it to calculate what's left of \textwidth, and
  %% use it for the other columns. See http://tex.stackexchange.com/questions/95397
  \newsavebox{\SolutionNameBox}
  \newcolumntype{\SolutionNameCol}{
    >{\begin{lrbox}{\SolutionNameBox}}l<{\end{lrbox}
    \eqmakebox[SolutionNameBox][l]{\unhcopy\SolutionNameBox}}
  }

\usepackage{tikz}

\newsubfloat{figure}        % subfigures with LaTeX

\usepackage{rotating}       % for sideways tables and figures
  \newcommand{\crows}[1]{\multicolumn{2}{c}{#1}}

%% Use agu style (American Geophysical Union) which only uses author
%% forenames and after too many authors, uses et. al.  All this helps
%% saves a lot of paper.
\usepackage[round]{natbib}
\bibliographystyle{agu}

\usepackage{seqsplit}
\usepackage{dnaseq}

\usepackage{siunitx}
  \DeclareSIUnit{\gn}{\textit{g$_n$}}   % standard gravity
  \DeclareSIUnit{\bp}{bp}               % base pairs
  \DeclareSIUnit{\cfu}{cfu}             % colony forming unit
  \DeclareSIUnit{\Molar}{\textsc{m}}
  \DeclareSIUnit{\mm}{\si{\milli}\si{\meter}}
  \DeclareSIUnit{\mM}{\si{\milli}\si{\Molar}}
  \DeclareSIUnit{\uM}{\si{\micro}\si{\Molar}}
  \DeclareSIUnit{\X}{\times}
  \newcommand{\dc}[1]{\SI{#1}{\degreeCelsius}}
  \newcommand{\pcent}[1]{\SI{#1}{\percent}}

%% This commands include the caption short description at the start of
%% long description and in bold.
\newcommand{\captionIntro}[2]{\caption[#1]{\textbf{#1.} #2}}
\newcommand{\captionofIntro}[3]{\captionof{#1}[#2]{\textbf{#2.} #3}}


%% Just like we have cite and citep to cite in text and between parentheses,
%% have the same for fref, tref, etc...
\newcommand{\frefp}[1]{(\fref{#1})}
\newcommand{\trefp}[1]{(\tref{#1})}
\newcommand{\Crefp}[1]{(\Cref{#1})}
\newcommand{\Srefp}[1]{(\Sref{#1})}
\newcommand{\Arefp}[1]{(\Aref{#1})}


\newcommand{\species}[1]{\textit{#1}}

%% NCBI Style Guide, Chapter 5 "Style Points and Conventions", recommends
%% italic for gene names (except in long list of genes), and roman for
%% protein names.
\newcommand{\gene}[1]{\textit{#1}}
\newcommand{\protein}[1]{#1}

\newcommand{\Kon}{$K_{on}$}
\newcommand{\Koff}{$K_{off}$}

\newcommand{\G}[1]{G$_#1$}  % for G0, G1, and G2 phases

\renewcommand{\abstractname}{Summary}

%% make it easy to center any dedication
\newcommand{\dedication}[1]{
{\clearpage\mbox{}\vfill\centering #1 \par\vfill\clearpage}}

\usepackage{makecell}
\usepackage[UKenglish,abbreviations]{foreign}

\input{methods/results/software_versions}

\input{results/variables-reference_comparison}
% 12 coding H2A, 10 coding H3, 6 coding H1
% 15 coding H2B, 12 coding H4
\newcommand{\MarzluffCodingGenesInHistOne}{50}
\newcommand{\MarzluffPseudoGenesInHistOne}{9}
\newcommand{\MarzluffTotalGenesInHistOne}{59}

% 3 coding H2A, 2 pseudo H3, 1 coding H3
% 4 pseudo H2B, 1 coding H2B, 1 coding H4
\newcommand{\MarzluffCodingGenesInHistTwo}{11}
\newcommand{\MarzluffPseudoGenesInHistTwo}{7}
\newcommand{\MarzluffTotalGenesInHistTwo}{XXXXX}

% 1 coding H2A, 1 coding H3
% 1 pseudo H2B, 1 coding H2B
\newcommand{\MarzluffCodingGenesInHistThree}{3}
\newcommand{\MarzluffPseudoGenesInHistThree}{XXXX}
\newcommand{\MarzluffTotalGenesInHistThree}{XXXXX}

% 1 coding H2A
\newcommand{\MarzluffCodingGenesInHistFour}{1}
\newcommand{\MarzluffPseudoGenesInHistFour}{0}
\newcommand{\MarzluffTotalGenesInHistFour}{0}

\newcommand{\MarzluffHistOneSpan}{2.3\,Mbp}
\newcommand{\MarzluffHistTwoSpan}{105\,kbp}
\newcommand{\MarzluffHistThreeSpan}{XXXXbp}
\newcommand{\MarzluffHistFourSpan}{XXXXbp}


%%
%% Some computations fit better here in LaTeX than the scripts.
%%

\FPeval{\result}{\SequenceChangeSinceReference{}
                + \PseudoSinceReference{}
                + \CodingSinceReference{}
                + \AddedSinceReference{}
                + \RemovedSinceReference{}}
\FPround{\TotalChangesSinceReference}{\result}{0} % of course is an integer

\FPeval{\result}{\SequenceChangeSinceReference{}
                +\PseudoSinceReference{}
                +\CodingSinceReference{}}
\FPround{\SequencesUpdatedSinceReference}{\result}{0} % of course is an integer


\author{David Miguel Susano Pinto \and Andrew Flaus}
\title{The Human Canonical Core Histone Catalogue}

\begin{document}

  \maketitle

  \begin{abstract}
    Core histone proteins H2A, H2B, H3, and H4 are encoded
    by a large family of genes distributed across the human genome.
    Canonical core histones contribute the majority of proteins to bulk chromatin packaging,
    and are encoded in \NumberOfClusters{} clusters
    by \TotalCoreCodingGenes{} coding genes comprising
    \HTwoACodingGenes{} for H2A,
    \HTwoBCodingGenes{} for H2B,
    \HThreeCodingGenes{} for H3,
    and \HFourCodingGenes{} for H4,
    along with at least \TotalCorePseudoGenes{} total pseudogenes.
    The canonical core histone genes display coding variation that gives rise to
    \HTwoAUniqueProteins{} H2A, \HTwoBUniqueProteins{} H2B,
    \HThreeUniqueProteins{} H3, and \HFourUniqueProteins{} H4 unique protein isoforms.
    Although histone proteins are highly conserved overall,
    these isoforms represent a surprising and seldom recognised variation
    with amino acid identity as low as
    \FPround{\result}{\HallMinPID}{0} \SI{\result}{\percent}
    between canonical histone proteins of the same type.
    The gene sequence and protein isoform diversity
    also exceeds commonly used subtype designations such as H2A.1 and H3.1,
    and exists in parallel with the well-known specialisation of variant histone proteins.
    RNA sequencing of histone transcripts shows evidence for
    differential expression of histone genes
    but the functional significance of this variation has not yet been investigated.
    To assist understanding of the implications of histone gene and protein diversity
    we have curated and catalogued the entire human canonical core
    histone gene and protein complement.
    In order to organise this information in a
    robust, accessible, and accurate form,
    we have applied software build automation tools to
    dynamically generate the canonical core histone repertoire
    based on current genome annotations
    and then to organise the information into a manuscript format.
    Automatically generated values are shown with a light grey background.
    Alongside recognition of the encoded protein diversity,
    this has led to multiple corrections to human histone annotations,
    reflecting the flux of the human genome as it is updated and
    enriched in reference databases.
    This approach is inspired by the aims of reproducible research
    and can be readily adapted to other gene families.
  \end{abstract}

  \section{Introduction}

  Histones are among the most abundant proteins in eukaryotic cells
  and contribute up to half the mass of chromatin \citep{AlbertsMBoC}.
  The core histone types H2A, H2B, H3, and H4
  define the structure and accessibility of the nucleosome
  as the fundamental repeating unit of genome organisation
  around which the DNA is wrapped \citep{Luger1997structure}.
  In addition, the many chemically reactive sidechains of histones
  are post-translationally modified
  as a nexus for signalling and heritable epigenetics \citep{Kouzarides2007}.

  Core histones are delineated as either canonical or variant based on
  their gene location, expression characteristics,
  and functional roles (\tref{tab:typical-histone-differences}).
  Canonical core histones contribute the majority of proteins to
  the bulk structure and generic function of chromatin,
  and are encoded by \TotalCoreGenes{} genes in \NumberOfClusters{}
  clusters named HIST1-HIST\NumberOfClusters{} in the human genome,
  of which \TotalCoreCodingGenes{} are coding genes and \TotalCorePseudoGenes{}
  are pseudogenes (\tref{tab:histone-gene-count}).

  \begin{table}
    \caption{Properties distinguishing canonical and variant core histone proteins.}
    \label{tab:typical-histone-differences}
    \centering
    \begin{tabular}{l l l}
      \toprule
      \null                     & Canonical             & Variants \\
      \midrule
      Expression timing         & Replication dependent & Replication independent \\
      Sequence identity         & High                  & Low \\
      Functional relationships  & Isoforms              & Specialised functions \\
      Transcript stabilisation  & Stem-loop             & poly(A) tail \\
      Gene distribution         & Clusters              & Scattered \\
      \bottomrule
    \end{tabular}
  \end{table}

  \begin{table}
    \caption{Count of human canonical core histone coding genes and pseudogenes
             by histone cluster and type. $\psi$ indicates pseudogenes.}
    \label{tab:histone-gene-count}
    \centering
    \begin{tabular}{l *{5}{r!{+}r<{$\psi$}}}
  \toprule
  \null   & \multicolumn{2}{c}{H2A}  & \multicolumn{2}{c}{H2B}
          & \multicolumn{2}{c}{H3}   & \multicolumn{2}{c}{H4}
          & \multicolumn{2}{c}{Total} \\
  \midrule
  HIST1   & \HTwoACodingInHISTOne{}     & \HTwoAPseudoInHISTOne{}
          & \HTwoBCodingInHISTOne{}     & \HTwoBPseudoInHISTOne{}
          & \HThreeCodingInHISTOne{}    & \HThreePseudoInHISTOne{}
          & \HFourCodingInHISTOne{}     & \HFourPseudoInHISTOne{}
          & \CoreCodingGenesInHISTOne{} & \CorePseudoGenesInHISTOne{} \\
  HIST2   & \HTwoACodingInHISTTwo{}     & \HTwoAPseudoInHISTTwo{}
          & \HTwoBCodingInHISTTwo{}     & \HTwoBPseudoInHISTTwo{}
          & \HThreeCodingInHISTTwo{}    & \HThreePseudoInHISTTwo{}
          & \HFourCodingInHISTTwo{}     & \HFourPseudoInHISTTwo{}
          & \CoreCodingGenesInHISTTwo{} & \CorePseudoGenesInHISTTwo{} \\
  HIST3   & \HTwoACodingInHISTThree{}   & \HTwoAPseudoInHISTThree{}
          & \HTwoBCodingInHISTThree{}   & \HTwoBPseudoInHISTThree{}
          & \HThreeCodingInHISTThree{}  & \HThreePseudoInHISTThree{}
          & \HFourCodingInHISTThree{}   & \HFourPseudoInHISTThree{}
          & \CoreCodingGenesInHISTThree{} & \CorePseudoGenesInHISTThree{} \\
  HIST4   & \HTwoACodingInHISTFour{}    & \HTwoAPseudoInHISTFour{}
          & \HTwoBCodingInHISTFour{}    & \HTwoBPseudoInHISTFour{}
          & \HThreeCodingInHISTFour{}   & \HThreePseudoInHISTFour{}
          & \HFourCodingInHISTFour{}    & \HFourPseudoInHISTFour{}
          & \CoreCodingGenesInHISTFour{} & \CorePseudoGenesInHISTFour{} \\
  \addlinespace
  Total   & \HTwoACodingGenes{}       & \HTwoAPseudoGenes{}
          & \HTwoBCodingGenes{}       & \HTwoBPseudoGenes{}
          & \HThreeCodingGenes{}      & \HThreePseudoGenes{}
          & \HFourCodingGenes{}       & \HFourPseudoGenes{}
          & \TotalCoreCodingGenes{}   & \TotalCorePseudoGenes{} \\
  \bottomrule
\end{tabular}


  \end{table}

  Relationships within the histone family have been described using a variety of terminologies
  reflecting biochemical, functional, and genomic perspectives that are briefly described below
  and summarised in \tref{tab:histone-divisions}.

  \afterpage{
    \captionof{table}{Terminology describing histone variation.}
    \label{tab:histone-divisions}
    \definecolor{shadecolor}{gray}{0.9}
    \begin{shaded}
      \begin{description}
        \item[Allelic variants] \hfill \newline
        Copies of canonical histone type genes,
        possibly with different sequences.
        Not located at same exact chromosomal locus as expected for alleles.
        Not histone variants.
        See also ``isoforms''.

        \item[Canonical core histones]\hfill \newline
        Core histones with properties described in \tref{tab:typical-histone-differences}.
        Contribute the majority of core histones in chromatin.
        Encompasses multiple protein isoforms.
        Complement of variant histones.

        \item[Core histones]\hfill \newline
        Histones that form part of the nucleosome core particle wrapping \SI{147}{\bp} of DNA,
        comprising types H2A, H2B, H3, and H4.
        Encompasses both canonical and variant histones,
        and complements linker histones.

        \item[Heteromorphous variants] \hfill \newline
        Core histone variants with distinct function and localisation
        that are readily separated by gel electrophoresis.

        \item[Homomorphous variants] \hfill \newline
        Canonical core histone subtypes
        requiring high resolution separation methods such as TAU PAGE.
        Synonym for ``subtypes''.

        \item[Families] \hfill \newline
        Synonym for ``histone types''.

        \item[Isoforms] \hfill \newline
        Proteins with high sequence identity and largely equivalent function.
        Functional equivalence has not been demonstrated for canonical histone isoforms.

        \item[Linker histones] \hfill \newline
        Histones binding to linker DNA adjacent to the nucleosome core particle.
        The two linker histone types are H1 and H5. Complement of core histones.

        \item[Non-allelic variants] \hfill \newline
        Synonym for ``variant histones''. See also ``allelic variants''.

        \item[Replacement histones] \hfill \newline
        Synonym for ``variant histones''.
        Named because they can replace canonical core histones assembled in S phase.

        \item[Replication-dependent histones] \hfill \newline
        Synonym for ``canonical core histones''.
        Named because expression occurs primarily in S phase.

        \item[Replication-independent histones] \hfill \newline
        Synonym for ``variant histones''.
        Named because expression is not predominantly in S phase.

        \item[Subtypes] \hfill \newline
        Canonical core histone type isoforms separable by TAU PAGE
        (e.g. H2A.1 and H2A.2).  Despite their naming, there is not necessarily functional evidence
        for differences between them.

        \item[Types] \hfill \newline
        Histone proteins sharing sequence homology
        that participate in specific combinations to define the repeating nucleosome structure.
        The 5 histone types are H1, H2A, H2B, H3, and H4.

        \item[Variant histones] \hfill \newline
        Core histones with properties described in \tref{tab:typical-histone-differences}.
        Contribute a minor proportion of histones in chromatin and perform specialised functions.
        Complement of canonical core histones.

        \item[Wild type histones] \hfill \newline
        Synonym of ``canonical core histones''.
      \end{description}
    \end{shaded}
  }

  \subsection{Biochemical perspective}

    Abundant histone proteins are readily isolated using their
    highly basic chemical character.
    Successive improvements in fractionation ultimately revealed 5 main histone types
    with nomenclature H1, H2A, H2B, H3, and H4 \citep{nomenclature}.
    An additional H1-related histone H5 is recognised in avian erythrocytes \citep{HFive-review}.

    The demonstration of the nucleosome as the fundamental
    repeating unit of chromatin \citep{Kornberg1974}
    showed that H2A, H2B, H3, and H4 associate as an octamer of two copies each within the
    nucleosome core particle. These four histones are referred to as core histones.
    In contrast, H1 associates with the linker DNA between nucleosome core particles
    and is referred to as a linker histone.
    The somatic H1 isoforms and its tissue-specific
    variants are described elsewhere \citep{HarshmanFreitas2013}.

    Arginine and lysine content was used as an early distinction between the histones \citep{ElginWeintraub1975}.
    The H1 linker histone has a low arginine/lysine ratio of
    \FPround{\result}{\LinkerArgLysRatio}{2} \result{}
    and became known as lysine-rich
    whereas the 4 core histones are arginine-rich
    with high arginine/lysine ratios of
    \FPround{\result}{\HTwoAArgLysRatio}{2} \result{} in H2A,
    \FPround{\result}{\HTwoBArgLysRatio}{2} \result{} in H2B,
    \FPround{\result}{\HThreeArgLysRatio}{2} \result{} in H3,
    and \FPround{\result}{\HFourArgLysRatio}{2} \result{} in H4 type isoforms.
    Nevertheless, the core histones contain many lysines particularly in their N-terminal tails.

    Separating histones by polyacrylamide gel electrophoresis (PAGE)
    using the strongly anionic detergent sodium dodecyl sulphate and neutral buffers (SDS PAGE)
    gives single bands for each histone type \citep{ShechterHake2007}.
    However, PAGE with non-ionic detergent Triton X--100 and urea as denaturants
    in acid buffers (TAU or AUT PAGE) allows the separation
    of histone types into multiple bands
    due to post-translational modifications and differences at specific amino acids
    in the polypeptides \citep{Zweidler1977}.
    These TAU PAGE separations gave rise to subtype designations
    H2A.1, H2A.2, H3.1, H3.2, and H3.3.

  \subsection{Functional perspective}

    Canonical core histone expression
    is significantly elevated during S~phase to provide chromatin packaging
    for DNA duplicated during replication \citep{WuBonner1981}.
    This led to their description as ``replication dependent'',
    although a supply of canonical histones is inevitably required
    to partner variants throughout the cell cycle.
    Metazoan canonical core histone genes are distinctive
    because they lack introns and give rise to non-polyadenylated protein coding transcripts.
    Turnover is independently regulated via a highly
    conserved 3' stem-loop (\tref{tab:typical-histone-differences}).

    In contrast, variant histones such as H2A.Z, TH2B, H3.3, and CENP-A have
    reduced sequence identity and lower abundance \citep{TalbertHenikoff2010}.
    They play functionally specific roles and are mostly expressed outside S~phase,
    so are described as ``replication independent''.
    Since histone variants are interpreted as taking the place
    of equivalent canonical core histone types,
    they are also referred to as ``replacement'' histones.

  \subsection{Genomic perspective}

    Canonical core histone genes are found in \NumberOfClusters{} clusters.
    The multiple gene copies in these clustered arrays are
    sometimes confusingly referred to as ``allelic''
    and the resulting combined protein isoforms are often considered to be ``wild type''
    although both genes and protein products display
    variation in primary sequence and relative abundance,
    and their functional equivalence has not been tested.
    In contrast, almost all variant histones are encoded by single genes dispersed in the genome
    with typical properties including introns, alternative splicing,
    and polyadenylated transcripts (\tref{tab:typical-histone-differences}).

  \subsection{Curation of canonical core histone diversity}

    Despite the importance of histones for chromatin organisation and extensive interest
    in their role in epigenetics and regulation, the systematic
    curation and classification of human histone
    gene and protein sequences has not been revisited
    since the landmark 2002 survey by \citet{Marzluff02}.
    The canonical core histone catalogue has since accumulated
    \TotalChangesSinceReference{}~differences (\tref{tab:difference-from-Marzluff02})
    from that survey due to rich annotations continuing to propagate
    into reference sequence databases.

    In this manuscript we provide a comprehensive catalogue
    of canonical core histone genes, encoded proteins, and pseudogenes
    based on reference genome annotations.
    This reveals a surprising and seldom recognised variation in encoded histone proteins
    that exceeds commonly used subtype designations
    but whose functional implications have not been investigated.

    Since curation and annotation are dynamic and evolving, we have
    implemented the manuscript so that it can be regenerated from the
    most current data in the NCBI RefSeq database in order to maintain
    its value as a reference source in an accessible format.  All
    figures and tables were automatically generated using NCBI RefSeq
    data from \printdate{\SequencesDate{}}.  In addition,
    automatically generated values in the text are displayed with a
    light grey background.  This manuscript generation process has
    remained stable in our laboratory for several years and represents
    an example of ``reproducible research'' \citep{Claerbout2000} that
    provides a novel model for curating relatively stable gene
    families.

  \section{Histone genes}

  \subsection{Canonical core histone gene nomenclature}
    Canonical core histone genes adhere to a Human Genome Organisation (HUGO)
    Gene Nomenclature Committee (HGNC)
    endorsed system derived from the cluster number and position relative
    to other histones \citep{Marzluff02}.
    This superseded an earlier arbitrary scheme with backslashes (e.g. H2b/b)
    that preceded genome sequencing \citep{AlbigGenomics1997,AlbigHumangen1997}.

    The canonical core histone gene symbols are divided into 3 parts:
    HIST cluster, histone type, and identifier letter
    for the order relative to other histone genes of the same type in
    the cluster (\fref{subfig:nomenclature-coding-canonical}).
    For example, \textit{HIST1H2BD} is nominally the fourth H2B coding gene in the HIST1 cluster.
    Identifiers are ordered by their genomic coordinates starting at
    the telomere of the short arm \citep{Marzluff02}.

    \begin{figure*}
      \centering
      \subbottom[canonical coding gene]{%
        \begin{minipage}{0.37\textwidth}
          \centering
          \Huge{%
            \colorbox{red}{\strut HIST1}%
            \colorbox{green}{\strut H2B}%
            \colorbox{blue!40}{\strut D}%
          }

          \scriptsize{%
            \colorbox{blue!40}{\strut 4\textsuperscript{th}}
            \colorbox{green}{\strut H2B}
            \colorbox{red}{\strut in Histone cluster 1}
          }
        \end{minipage}
        \label{subfig:nomenclature-coding-canonical}
      }
      \subbottom[canonical pseudogene]{%
        \begin{minipage}{0.37\textwidth}
          \centering
          \Huge{%
            \colorbox{red}{\strut HIST2}%
            \colorbox{green}{\strut H3}%
            \colorbox{yellow}{\strut PS}%
            \colorbox{blue!40}{\strut 2}%
          }

          \scriptsize{%
            \colorbox{blue!43}{\strut 2\textsuperscript{nd}}
            \colorbox{green}{\strut H3}
            \colorbox{yellow}{\strut pseudogene}
            \colorbox{blue!40}{\strut found}
            \colorbox{red}{\strut in cluster 2}
          }
        \end{minipage}
        \label{subfig:nomenclature-pseudo-canonical}
      }
      \subbottom[variant]{%
        \begin{minipage}{0.22\textwidth}
          \centering
          \Huge{%
            \colorbox{red}{\strut H2A}%
            \colorbox{green}{\strut F}%
            \colorbox{blue!40}{\strut X}%
          }

          \scriptsize{%
            \colorbox{red}{\strut H2A}
            \colorbox{green}{\strut Family}
            \colorbox{blue!40}{\strut member X}
          }
        \end{minipage}
        \label{subfig:nomenclature-variant}
      }
      \caption{Histone gene nomenclature.
               \subcaptionref{subfig:nomenclature-coding-canonical}
               Canonical core histone gene names encode relative genomic
               order by cluster.
               \subcaptionref{subfig:nomenclature-pseudo-canonical}
               Pseudogenes named since 2002 include cluster, PS label,
               and discovery order identifier.
               \subcaptionref{subfig:nomenclature-variant}
               Most variant core histone genes are identified by type
               then F for family and identifier letter.}
    \end{figure*}

    Two exceptions were originally applied to these simple naming rules \citep{Marzluff02}.
    Firstly, the positional identifier is omitted if there are no
    other histones of the same type in the cluster,
    so \textit{HIST3H2A} is the sole H2A gene in HIST3.
    Secondly, the human and mouse histone clusters are largely syntenic
    so positional identifier letters for missing orthologs were skipped
    to maintain the equivalence of gene symbols.
    Consequently there is no human \textit{HIST1H2AF} to accommodate
    \textit{Hist1h2af} in mouse while keeping both --E and --G identifiers
    consistent for mouse and human orthologs.

    Several new histone genes have been uncovered since the original
    naming (\tref{tab:difference-from-Marzluff02})
    and this required additional nomenclature exceptions.
    For example, new H2A encoding genes were identified in both human
    and mouse HIST2 cluster preceding \textit{HIST2H2AA}
    leading to the renaming of \textit{HIST2H2AA} to \textit{HIST2H2AA3}
    and the addition of a new human gene as \textit{HIST2H2AA4}.
    There are no human orthologs of mouse \textit{Hist2h2aa1} and \textit{Hist2h2aa2}.

    Furthermore, no distinction was originally made between
    pseudogenes and functional coding genes,
    so \textit{HIST3H2BA} is a pseudogene whereas neighbouring
    \textit{HIST3H2BB} is the only functional H2B coding gene in HIST3.
    The HGNC definition of a pseudogene is
    a sequence that is generally untranscribed and untranslated
    but which has at least 50\% predicted amino acid identity
    across 50\% of the open reading frame to a named gene \citep{HGNC2013}.
    Newly uncovered histone pseudogenes are now suffixed with PS and a
    number in order of discovery (\fref{subfig:nomenclature-pseudo-canonical}),
    such as \textit{HIST1H2APS5} as the fifth H2A pseudogene discovered in HIST1.
    However, the pseudogenes named in the original classification retain their symbols without PS.
    This means that the absence of a PS suffix does not indicate a functional gene,
    and that there is no positional information in the gene symbols of most pseudogenes.

  \subsection{Histone gene clustering}
    The human canonical core histone genes are located in clusters HIST1 to HIST\NumberOfClusters{},
    named in order of decreasing histone gene count (\tref{tab:histone-gene-count}).
    HIST1 is the major histone gene cluster at locus~\HISTOneLocus{}
    with \CoreCodingGenesInHISTOne{}~functional core histone genes plus all canonical H1 histones,
    representing \FPround{\result}{\PercentageGenesInHISTOne}{0} \SI{\result}{\percent}
    of all canonical core histone genes.
    HIST2 at locus \HISTTwoLocus{} contains \CoreCodingGenesInHISTTwo{}~coding genes,
    HIST3 at locus \HISTThreeLocus{} contains \CoreCodingGenesInHISTThree{}~coding genes,
    and HIST4 at locus \HISTFourLocus{} contains \CoreCodingGenesInHISTFour{}~coding gene.

    HIST1 and HIST2 are both contiguous high density arrays of histone genes.
    HIST1 spans \AutoSIPrefix{\HISTOneSpan}{1}{\bp}
    and is the second most gene dense region in the human genome at
    megabase scale after the MHC class III region \citep{MHC-III-analysis}.
    The only non-histone protein coding gene located within the principal region of this cluster
    is \textit{HFE}, encoding the hemochromatosis protein \citep{AlbigDoenecke1998},
    although a number of other genes are located
    proximal to the outlying \textit{HIST1H2AA} and \textit{HIST1H2BA} pair.

    It has also been argued that histone clustering
    does not contribute to gene conversion \citep{NeiRooney2005}.
    HIST1 is located towards the distal end of the major histocompatibility complex (MHC)
    in the extended class~I region \citep{MHC-I-transcript, MHC-complete-sequencing-1999}
    and it has even been suggested that this proximity
    may suppresses recombination \citep{MHC-repressed-by-HIST}.
    In contrast, HIST2 may be prone to deletions and frequent rearrangements
    \citep{HISTTwo-prone-deletion-discovery, HISTTwo-prone-deletion-focus}

    The functional significance of histone gene clustering remains to be demonstrated.
    It has been suggested that clustering may facilitate coordinate regulation \citep{Eirinlopez2009,close-regulators},
    but interpreting such a functional relationship with genome organisation
    requires an accurate catalogue of the histone genes and their individual roles.

    Conversely, progress in understanding histone gene function
    suggests there is a need to update the definitions of the canonical histone gene clusters.
    For example, the \textit{HIST1H2AA} and \textit{HIST1H2BA} genes
    are located \SI{300}{\kilo\bp} upstream of the rest of the HIST1 cluster
    and separated from other cluster members a number of non-histone genes.
    Although the two genes have been assigned canonical histone gene names,
    the \textit{HIST1H2BA} gene in fact encodes the most divergent canonical H2B protein
    which is also known as TH2, TH2B, or hTSH2B and considered to be
    a testes-specific histone protein \citep{Zalensky2002,LiAusio2005,Shinagawa2014}.
    Immediately adjacent is \textit{HIST1H2AA} encoding
    a H2A protein isoform of similarly high variation.
    The syntenic rat orthologues of \textit{HIST1H2AA} and \textit{HIST1H2BA}
    are divergently transcribed specifically in testes \citep{HuhChae1991},
    and the mouse orthologues H2AL1/TH2a and TH2B have been shown to participate in gametogenesis \citep{GovinCaron2007}
    as well as to enhance stem cell reprogramming \citep{ShinagawaIshii2014,PadavattanKumarevel2015}.
    It is therefore possible that \textit{HIST1H2AA} and \textit{HIST1H2BA}
    are undergoing sub-functionalisation and could be reclassified
    as histone variants H2A.L and H2B.1 \citep{Talbert2012}.
    This would in turn lead to a recalculation of the length of the HIST1 cluster.

    A similar case may exist for the small HIST3 cluster \SI{80}{\mega\bp} downstream of HIST2.
    HIST3 contains protein coding genes \textit{HIST3H2A}, \textit{HIST3H2BB}, and \textit{HIST3H3}.
    \textit{HIST3H3} encodes testes-specific H3T/H3.4
    that has distinctive biochemical properties \citep{WittExpCellRes1996,KurumizakaCOSB2013},
    while the proteins encoded by \textit{HIST3H2A} and \textit{HIST3H2BB}
    are amongst the most divergent canonical histones of their types.

    The examples of the \textit{HIST1H2AA} and \textit{HIST1H2BA} pair
    and the HIST3 cluster illustrate
    the evolving nature of the human canonical histone complement
    and the need for a dynamic approach to classification.

  \subsection{Histone gene sequences}
    Despite the high conservation of canonical core histone protein sequences,
    the coding regions of these genes exhibit considerable variation
    (\fref{fig:h2a-histone-gene-variation},~\ref{fig:h2b-histone-gene-variation},
    ~\ref{fig:h3-histone-gene-variation}, and~\ref{fig:h4-histone-gene-variation}).
    These differences are largely located in the third base position of codons
    and reflect the very high ratio of synonymous to non-synonymous
    substitutions (\tref{tab:histone-gene-differences}).

    The mean number of synonymous substitutions per site ($d_S$)
    in the sets of histone gene type isoforms is
    \FPround{\result}{\MeanHTwoAdS}{1} \result{} for H2A,
    \FPround{\result}{\MeanHTwoBdS}{1} \result{} for H2B,
    \FPround{\result}{\MeanHThreedS}{1} \result{} for H3,
    and \FPround{\result}{\MeanHFourdS}{1} \result{} for H4.
    This is consistent with observations that
    synonymous codon divergence far exceeds non-synonymous variation
    for histone genes across eukaryotes \citep{Piontkivska2002, Rooney2002}.
    It supports a hypothesis that histone protein sequence conservation
    results from birth and death evolution through strong selective pressure
    at the protein level rather than
    sequence homogenisation based on clustering \citep{NeiRooney2005}.

    Despite the level of synonymous substitution,
    codon usage is strongly biased towards the most frequently used
    human codons for most amino acids (\tref{tab:histone-gene-codonusage} and data not shown).
    This suggests that histone translation may be sensitive to tRNA abundance.

  \subsection{Histone variant genes}
    The \TotalCoreVariantGenes{} annotated human histone variant genes
    are listed in \tref{tab:variant-catalogue} for completeness.
    Histone variant gene families comprise only one or a few copies
    dispersed across the genome outside the canonical histone clusters.
    For example the three H3.3 variant encoding genes are located
    on chromosomes 1, 12, and 17, far removed from other histone genes.

    Most variant gene symbols have a separate 3 part nomenclature
    consisting of histone type, F for family,
    and an identifier letter (\fref{fig:nomenclature-variant}).
    Increasing interest in histone variant function \citep{MazeAllis2014}
    coupled with a variety of usages and conflicts between species
    has led to guidelines for improved consistency in histone variant nomenclature
    \citep{Talbert2012}.

  \section{Histone transcripts}

  Canonical core histone transcripts are among the only protein coding messages
  that do not undergo polyadenylation,
  instead carrying a unique stem-loop structure (\tref{tab:typical-histone-differences}).
  They do have a 5' 7--methyl--guanosine cap \citep{MarzluffNatRevGen2008}.

  Canonical core histone gene transcription is regulated
  by cell-cycle dependent phosphorylation of the histone-specific
  Nuclear Protein of the Ataxia-Telangiectasia locus (NPAT) coactivator
  and interaction with the accessory protein
  FADD-Like interleukin-1$\beta$-converting enzyme/caspase-8-ASsociated Huge protein (FLASH),
  resulting in assembly of histone locus bodies
  coordinating factors responsible for transcription and processing
  \citep{MarzluffNatRevGen2008,RattrayMueller2012,Hoefig2014}.
  Variability is observed in canonical core histone isoform gene transcription,
  both by analysis of non-polyadenylated transcripts \citep{YangGenomeBiol2011}
  and RNA polymerase II promoter occupancy \citep{Ederveen2011}.

  Overall there is estimated to be a 35 fold increase in mammalian
  canonical histone transcripts during S phase,
  principally as a result of a 10 fold increase in mRNA stabilisation
  via stem-loop dependent mechanisms,
  and a 3--5 fold up-regulation in canonical core histone gene transcription \citep{HarrisMCB1991}.

  This post-transcriptional regulation is achieved by 
  a stem-loop encoded in the mRNA 3' untranslated region
  that is recognised by spliceosome-related RNA
  processing and stabilisation complexes \citep{stem-loop-structure}.
  The start location of the annotated stem-loops in human canonical histone transcripts
  ranges from \StemLoopStartMin{} to \StemLoopStartMax{} bp after the stop codon
  with a mode of \StemLoopStartMode{} bp.
  The sequence logo of aligned stem-loops confirms that the stem-loop is
  highly conserved (\fref{fig:stem-loop-seqlogo}).

  The RNA stem-loop structure is bound by the Stem-Loop Binding Protein (SLBP)
  which is up-regulated 10--20 fold during S~phase to stabilise
  histone mRNAs \citep{SLBP-regulation}.
  Immediately downstream of the stem-loop a purine-rich Histone Downstream Element (HDE)
  interacts with U7 snRNA to direct efficient 3' end processing.
  Although this is not an annotated feature of histone genes,
  alignment of the canonical HDE \citep{HDE-sequence} to all canonical histone genes
  shows the modal location of the HDE is
  \HDEsDistanceFromStemLoopMode{} bp downstream of the stem-loop.
  The sequence logo confirms that this feature is also highly conserved (\fref{fig:HDE-seqlogo}).

  \begin{figure}
    \centering
    \subtop[Stem-loop]{
      \includegraphics{figs/seqlogo_stem_loops.pdf}
      \label{fig:stem-loop-seqlogo}
    }
    \hfill
    \subtop[HDE]{
      \includegraphics{figs/seqlogo_HDEs.pdf}
      \label{fig:HDE-seqlogo}
    }
    \caption{%
      Sequence logos for
      \subcaptionref{fig:stem-loop-seqlogo} annotated stem-loops and
      \subcaptionref{fig:HDE-seqlogo}  Histone Downstream Elements (HDEs)
      identified by homology
      for all canonical core histone gene 3' untranslated regions (UTRs).
    }
  \end{figure}

  Stabilisation and processing extend the half life of canonical core histone \mbox{mRNAs}
  during S phase and contribute to increased histone translation efficiency,
  enabling rapid production of histones to package the newly duplicated genomes.

  Although the vast bulk of canonical core histone transcripts
  appear to be regulated by this mechanism,
  1-5\% of transcripts are found to be 3' polyadenylated \citep{YangGenomeBiol2011}
  and some genes have annotations indicating two transcripts
  differing in whether they have stem-loop or polyadenylation signals (\tref{tab:curation-anomalies}).

  Core histone variant transcripts lack a 3' stem-loop and are
  polyadenylated in the same way as most protein coding genes.
  The exception is H2AX, which has alternatively processed transcripts
  exhibiting both the stem-loop characteristic of a canonical core histone
  and a poly(A) tail found on variant core histones \citep{HTwoAX-transcripts,our-H2AX-review}.

  \section{Histone proteins}
  Most depictions of chromatin imply that
  canonical core histone protein types behave as a single protein,
  often referred to as ``wild type''.
  The assumption is based on the relatively high identity of histone sequences
  between isoforms and species,
  and historical interest in functional roles of histone variants.

  However, the very strong selection pressure on
  amino acid sequences encoded by canonical core histone genes \citep{NeiRooney2005},
  the roles of specific amino acid differences in observed proteins \citep{MazeAllis2014},
  and the consequences of small variations in proteins on the structure of nucleosomes \citep{KurumizakaCOSB2013}
  all suggest that the minor differences in the encoded canonical core histone protein isoforms
  could have functional implications.

  Encoded isoform variation in canonical histone types H2A and H2B is most pronounced,
  with \HTwoAUniqueProteins{}~H2A and \HTwoBUniqueProteins{}~H2B distinct polypeptides
  for these histone type genes.
  In contrast, only \HThreeUniqueProteins{}~H3 and \HFourUniqueProteins{}~H4
  distinct polypeptides are encoded by similar numbers of genes (\tref{tab:histone-gene-count}).
  Although most variation is a result of amino acid substitutions,
  some H2A and H2B isoforms also show length differences.

  The nomenclature for histone protein isoforms follows directly
  from the gene names \citep{Marzluff02},
  and supersedes the earlier nomenclature with forward slash which used different isoform letters
  \citep{AlbigGenomics1997,AlbigHumangen1997}.
  The encoded proteins described below are products of genes and transcripts listed
  in table \tref{tab:histone-catalogue}.
  Human canonical core histone polypeptides have typically been numbered
  with omission of the N-terminal methionine
  since this is likely to be removed because most sequences have
  a small hydrophilic amino acid as the second residue \citep{XiaoPeiBiochem2010}.
  This convention predates the Human Genome Variation Society (HGVS) recommendation
  to include the initial methionine as residue 1.
  We have omitted the N-terminal methionine on figures, alignments,
  and amino-acid numbering for consistency.

%% To include methionine just uncomment 3 lines in the routine that reads the sequence files in MyLib.pm

  \subsection{Canonical H2A isoforms}
    Canonical H2A genes encode \HTwoAUniqueProteins{}~different
    protein isoforms with pairwise identity down to
    \FPround{\result}{\HTwoAPID}{0} \SI{\result}{\percent} (\tref{tab:H2A-consensus}).
    These are separable by TAU PAGE into two bands identified as H2A.1 and H2A.2.
    TAU PAGE distinguishes leucine from methionine at residue 51 \citep{FranklinZweidler1977,Zweidler1977},
    implying there are up to 8 different protein sequences in H2A.1 and 3 in H2A.2.
    There is no concordance between H2A.1 and H2A.2
    and the location of isoform-encoding genes in the histone gene clusters.
    The HIST2 cluster contains genes encoding isoforms with both Leu51 and Met51,
    while HIST1, HIST2, and HIST3 clusters all contain genes encoding H2A isoforms with Leu51.
    No functional distinction between H2A.1 and H2A.2 has been reported.

    Excluding \textit{HIST1H2AA} discussed above,
    the sites of difference in two or more isoforms of canonical H2A are
    serine or threonine at residue 16,
    alanine or serine at residue 40,
    lysine, arginine, or glycine at residue 99,
    and the C-terminal residues from 124 onwards (\tref{tab:H2A-consensus}).
    All these sites have implications for post-translational modifications.

    \begin{table}
      \caption{%
        Canonical H2A encoded protein isoforms.
        Upper panel shows isoform variations relative to most common isoform
        using HGVS recommended nomenclature \citep{mutnomenclature2003}.
        Lower panel shows sequence logo of all isoforms aligned
        with invariant residues in grey.
      }
      \label{tab:H2A-consensus}
      \input{results/table-H2A-proteins-align}
      \includegraphics[width=\textwidth]{figs/seqlogo_H2A_proteins.pdf}
    \end{table}

  \subsection{Canonical H2B isoforms}
    Canonical H2B has more isoforms than the other histone types (\tref{tab:H2B-consensus})
    with \HTwoBUniqueProteins{}~unique proteins
    diverging to \FPround{\result}{\HTwoBPID}{0}\result\% identity.
    Nevertheless, all isoforms appear to migrate together in TAU PAGE.

    There is significant variability between isoforms in the N-terminal region
    and this is one of the most variable sites between canonical
    core histones of different species.
    Human H2B has a unique and distinctive N-terminal proline-acidic-proline motif (PEP/PDP)
    followed by a very variable residue that can be alanine, serine,
    threonine, or valine (\tref{tab:H2B-consensus}).

    Excluding \textit{HIST1H2BA} discussed above,
    the remaining isoform differences in two or more isoform are mainly the chemically
    similar valine or isoleucine at residue 39,
    and serine to glycine and alanine at residues 75 and 124 respectively,
    which have have post-translational modification implications.
    A number of H2B genes are annotated to have multiple transcripts,
    although in most cases these transcripts encode identical protein isoforms.
    Variation in H2B transcript isoform levels between multiple cancer cell
    lines has been observed \citep{Molden2015},
    although the functional implications are unknown.

    \begin{table}
      \caption{%
        Canonical H2B encoded protein isoforms.
        Upper panel shows isoform variations relative to most common isoform
        using HGVS recommended nomenclature \citep{mutnomenclature2016}.
        For clarity, isoforms encoded by multiple transcripts of a single gene
        are distingushed by a numerical suffix (see \tref{tab:histone-catalogue}).
        Lower panel shows sequence logo of all isoforms aligned
        with invariant residues in grey.
      }
      \label{tab:H2B-consensus}
      \input{results/table-H2B-proteins-align}
      \includegraphics[width=\textwidth]{figs/seqlogo_H2B_proteins.pdf}
    \end{table}

  \subsection{Canonical H3 isoforms}
    Canonical H3 genes encode only \HThreeUniqueProteins{}~different
    protein isoforms (\tref{tab:H3-consensus}).
    The majority of H3 genes are in the HIST1 cluster and encode a
    single polypeptide sequence \citep{Ederveen2011}
    whereas the three canonical H3 genes in HIST2 encode a distinct isoform
    with the interesting difference of serine instead of cysteine at residue 96
    that is separable by TAU PAGE \citep{FranklinZweidler1977}.
    By apparent coincidence this means HIST1-encoded canonical H3 isoforms are
    identified as H3.1 and the HIST2-encoded copies are H3.2.
    As discussed above, \textit{HIST3H3} with four amino acid differences
    appears to be the largely testes specific variant H3T
    that has been assigned as H3.4 in variant nomenclature \citep{Talbert2012}
    even though it would be predicted to migrate with H3.1 in TAU PAGE \citep{FranklinZweidler1977}.

    \begin{table}
      \caption{%
        Canonical H3 encoded protein isoforms.
        Upper panel shows isoform variations relative to most common isoform
        using HGVS recommended nomenclature \citep{mutnomenclature2003}.
        Lower panel shows sequence logo of all isoforms aligned
        with invariant residues in grey.
      }
      \label{tab:H3-consensus}
      \input{results/table-H3-proteins-align}
      \includegraphics[width=\textwidth]{figs/seqlogo_H3_proteins.pdf}
    \end{table}

  \subsection{Canonical H4 isoforms}
    H4 is the most homogeneous of all canonical core histones,
    with all but one of the \HFourCodingGenes{}~genes encoding
    an identical protein sequence (\tref{tab:H4-consensus}).
    These genes are located across HIST1, HIST2, and the isolated
    \textit{HIST4H4} as the sole member of HIST4.

    The divergent \textit{HIST1H4G} gene in the middle of the HIST1 cluster
    encodes an isoform with 15 amino acid differences and a deletion of the C-terminal 5 residues.
    This gene is annotated as being transcribed and merits further investigation.

    \begin{table}
      \caption{%
        Canonical H4 encoded protein isoforms.
        Upper panel shows isoform variations relative to most common isoform
        using HGVS recommended nomenclature \citep{mutnomenclature2003}.
        Lower panel shows sequence logo of all isoforms aligned
        with invariant residues in grey.
      }
      \label{tab:H4-consensus}
      \input{results/table-H4-proteins-align}
      \includegraphics[width=\textwidth]{figs/seqlogo_H4_proteins.pdf}
    \end{table}

  \section{Reproducible research}

  A considerable number of builds and updates of the human genome sequence have been made
  since the last major survey of human canonical histone genes in 2002 \citep{Marzluff02}
  resulting in
  \AddedSinceReference{} canonical core histone genes added,
  \RemovedSinceReference{} removed,
  and \SequencesUpdatedSinceReference{}
  sequences updated (\tref{tab:difference-from-Marzluff02}).
  These changes reflect the ongoing nature and challenges of
  genome curation and annotation \citep{BorkKoonin1998}.

  The human canonical core histone gene complement is relatively mature
  and is unlikely to undergo major change.
  Nevertheless, it is inevitable that sequences and annotations will be revised over time
  based on continuous curation and improved experimental insights.
  This in turn prompts reevaluation of assumptions about their properties,
  illustrated by the recognition that some genes
  annotated as canonical core histone isoforms are testes-specific variants \citep{Talbert2012},
  and by new observations of cell type-specific expression \citep{Molden2015}.

  It is important for communities of researchers to contribute
  to ongoing curation and to feed back information to database maintainers \citep{SteinNRG2001},
  although many are unaware of this opportunity \citep{HollidaySPR2015}.
  During this work we have submitted a number of proposals
  for incorrect, inconsistent, or missing annotations to RefSeq curators.
  These annotation issues were identified automatically
  by scripted tests for consistency and uniformity
  between the database annotations and expected properties of histones.
  \tref{tab:curation-anomalies} lists current apparent anomalies from these tests
  and is the basis for ongoing feedbackd to RefSeq.

  Dynamic data can be presented in specialist online database resources
  such as the Histone Database \citep{HistoneDB2016} and HIstome \citep{HIstome2012}.
  These database interfaces provide comprehensive access to data
  but have limited curation and are difficult to cite.
  In contrast, manuscripts are the established method of scientific communication
  because they provide descriptive context and accessible formatting for readers,
  can be peer reviewed, and have well-established methods for citation.
  However, static manuscripts such as earlier surveys of histone genes \citep{AlbigHumangen1997,Marzluff02}
  inevitably become supersed by improvements in source data and cannot be updated to remove errors.

  A self-updating manuscript bridges the features of dynamic database and static manuscript
  presentation styles by providing convenient access to the most current information.
  It encourages communities of researchers to directly feed into the annotation process
  and enables rapid leveraging of the most current data for relatively stable gene families.

  Implementating a dynamic manuscript is also an example of ``reproducible research''
  \citep{reproducible-research-bioinformatics,reproducible-research-law}
  that can address the topical challenge of irreproducibility in biological data
  and interpretation \citep{ErrorProne2012,OpenPrograms2012}.

  This manuscript is generated directly from sequences and annotations
  in the core NCBI RefSeq resource \citep{OLearyRefseq2016}.
  The processing system does not cache intermediate information,
  so all changes contributed by the community of histone researchers
  and curated by professional database maintainers
  are directly and automatically reflected in the next manuscript refresh.
  All scripts including instructions for automatic builds
  are transparently available in a public repository.
  Core processing is based on a BioPerl program contributed to the Bio-EUtilities distribution
  and publicly available since 2013.
  Dependencies on raw data sources, alignment algorithms, or display output libraries
  can also be upgraded since the processing is automatic and script-based.

  One potential challenge for researchers is the referencing of dynamic data within the manuscript.
  This can be simply addressed by users citing the publication as a traditional static manuscript
  then stating the access date in Materials and Methods, as they would for a database.

  Although the dynamic data will remain current,
  major revisions in understanding of a gene family or accumulation of small insights
  may in time render the explanatory static manuscript text obsolete.
  In this case the manuscript can simply be refreshed and republished in revised form
  in the same way as a traditional manuscript.
  The underlying scripts generating the dynamic data do not need to be rewritten,
  although new functions can be added to reflect new insights.

  The core scripts underlying this manuscript are written to facilitate generating
  equivalent catalogues for other organisms via build options.
  We have successfully trialled this for \textit{Mus musculus} (not shown).
  Therefore, an equivalent dynamic manuscript for such gene sets
  only requires redrafting the surrounding static explanatory text.
  The approach is also applicable to larger and more diverse gene families such as
  our previous cataloguing of the Snf2 family \citep{andrew-snf2-catalogue}.

  \section{Conclusion}

  Comprehensive curation of the canonical core histone gene family
  reveals significant diversity of isoforms,
  and challenges assumptions of bulk chromatin homogeneity.
  It has led to improvements in consistency of genome annotations
  and provides a reference for interpreting genomic and proteomic datasets.
  The differences between canonical core histone protein isoforms
  uncovers novel questions about their functional significance
  that suggest future experimental investigations.
  As a dynamic manuscript this catalogue provides a continuously up-to-date overview
  of the human canonical core histone gene family.

  \section{Materials and methods}

  The primary manuscript is generated from \LaTeX{} sources, derived
  from dynamic data, and built using SCons \citep{SCons2005}.
  Search and download of fresh data is performed by bp\_genbank\_ref\_extractor
  which was implemented for the Bioperl project \citep{bioperl} and has
  been included in the Bio-EUtilities module. Analysis of the data relies
  heavily on BioPerl.

  All sources are freely available in a github repository
  \url{https://github.com/af-lab/histone-catalogue}, including all
  sources for figures, manuscript templates, and build system making public
  all parameters used for processing.

  This build of the manuscript was generated using BioPerl~\BioPerlVersion{}
  and Bio-EUtilities~\BioEUtilitiesVersion{}.
  Sequence and annotation data was obtained from NCBI RefSeq \citep{OLearyRefseq2016}
  on \printdate{\SequencesDate{}}.
  Sequence alignments were generated by T-Coffee \TCoffeVersion{} \citep{tcoffee2000}.
  Sequence logos were generated using WebLogo version \WebLogoVersion{} \citep{weblogo}.
  Description of sequence variants are represented following HGVS
  recommended nomenclature \citep{mutnomenclature2016}.


  \section{Acknowledgements}
    We are grateful to the Perl community on the \url{irc.freenode.net}
    channels \#perl and \#bioperl,
    especially Altreus, mst, LeoNerd, and pyrimidine.
    Invaluable support was also provided by the \TeX{} stackexchange community,
    in particular David Carlisle.

  \bibliography{references}

  \newpage
  \appendix
  \section*{Supplementary Data}

  \begin{supplement}
    \caption{
        Changes in human canonical core histone gene catalogue
        annotated in NCBI RefSeq obtained on \protect\printdate{\SequencesDate{}}
        compared to \citet{Marzluff02}.
    }
    \label{tab:difference-from-Marzluff02}
    \input{results/table-reference_comparison}
  \end{supplement}

  %% This is a very long table (ctabular), that spans multiple pages.  TeX
  %% standard floats are unable to handle it.  We could use the longtable
  %% or the xtab packages (see Section 11.8 named free tabulars of the
  %% memoir class).  We may still do that later but at the moment we use
  %% capt-of which allows to enter a caption outside a float.
  \newpage
  \captionof{supplement}{%
    Catalogue of annotated human canonical core histone genes, transcripts,
    and encoded proteins.
    Data obtained from NCBI RefSeq \citep{OLearyRefseq2016} on \protect\printdate{\SequencesDate{}}.
  }
  \label{tab:histone-catalogue}
  \input{results/table-histone_catalogue}

  \newpage
  \begin{supplement}
    \centering
    \caption{
        Sequence logo for the human canonical H2A gene coding regions
        listed in table \tref{tab:histone-catalogue}.
        Initiator codon ATG and stop codon are omitted.
    }
    \label{fig:h2a-histone-gene-variation}
    \includegraphics[width=1.0\textwidth]{figs/seqlogo_H2A_cds.pdf}
  \end{supplement}
  \newpage
  \begin{supplement}
    \centering
    \caption{
        Sequence logo for the human canonical H2B gene coding regions
        listed in table \tref{tab:histone-catalogue}.
        Initiator codon ATG and stop codon are omitted.
    }
    \label{fig:h2b-histone-gene-variation}
    \includegraphics[width=1.0\textwidth]{figs/seqlogo_H2B_cds.pdf}
  \end{supplement}
  \newpage
  \begin{supplement}
    \centering
    \caption{
        Sequence logo for the human canonical H3 gene coding regions
        listed in table \tref{tab:histone-catalogue}.
        Initiator codon ATG and stop codon are omitted.
    }
    \label{fig:h3-histone-gene-variation}
    \includegraphics[width=1.0\textwidth]{figs/seqlogo_H3_cds.pdf}
  \end{supplement}
  \newpage
  \begin{supplement}
    \centering
    \caption{
        Sequence logo for the human canonical H4 gene coding regions
        listed in table \tref{tab:histone-catalogue}.
        Initiator codon ATG and stop codon are omitted.
    }
    \label{fig:h4-histone-gene-variation}
    \includegraphics[width=1.0\textwidth]{figs/seqlogo_H4_cds.pdf}
  \end{supplement}

  \newpage
  \begin{supplement}
    \caption{
        Annotated histone variants.
        HGNC histone family names \citep{HGNC2015}, commonly used protein names,
        and nomenclature proposal of \citet{Talbert2012}.
    }
    \centering
    \begin{tabularx}{\linewidth}{l l l >{\raggedright\arraybackslash}X}
      \toprule
      Family & Common name & \citet{Talbert2012} & Notes \\
      \midrule
      H2AFB & H2A.Bbd & H2A.B & equivalent to H2A.L \\
      H2AFJ & H2A.J & H2A.J & at HIST4 locus \\
      H2AFV & H2A.Z-2 & H2A.Z.2 & - \\
      H2AFX & H2AX/H2A.X & H2A.X & - \\
      H2AFY & macroH2A/mH2A & macroH2A & - \\
      H2AFZ & H2A.Z & H2A.Z.1 & - \\
      H2BFM & ? & H2B.M & homologous to H2B.W, X-linked\\
      H2BFS & - & - & pseudogene \\
      H2BFWT & ? & H2B.W & testes specific, X-linked \\
      H2BFX & - & - & pseudogene \\
      HIST1H2BA & TH2B/TSH2B & TS H2B.1 & testes specific \\
      HIST3H3 & H3T & H3.4 & testes specific \\
      H3F3 & H3.3 & H3.3 & euchromatin related \\
      CENPA & CENP-A & - & centromere-specific \\
      \bottomrule
    \end{tabularx}
  \end{supplement}

  \newpage
  \begin{supplement}
    \caption{
        Mean synonymous and non-synonymous distances between pairs of canonical core histone genes.
        Mean over all pairwise comparisons of non-synonymous ($d_N$) and synonymous ($d_S$) 
        nucleotide substitutions and mean of $d_N/d_S$ ratios for canonical core histone coding regions
        computed using \citet{GoldmanYang1994} model implemented in
        \texttt{codeml} from the PAML package \citep{PAML2007}.
    }
    \label{tab:histone-gene-differences}
    \centering
    \begin{tabular}{l l l l}
      \toprule
      \null & \centercell{$d_N$} & \centercell{$d_S$} & \centercell{$d_N/d_S$} \\
      \midrule
      H2A & \MeanHTwoAdN  & \MeanHTwoAdS  & \MeanHTwoAdNdS \\
      H2B & \MeanHTwoBdN  & \MeanHTwoBdS  & \MeanHTwoBdNdS \\
      H3  & \MeanHThreedN & \MeanHThreedS & \MeanHThreedNdS \\
      H4  & \MeanHFourdN  & \MeanHFourdS  & \MeanHFourdNdS \\
      \bottomrule
    \end{tabular}
  \end{supplement}

  \newpage
  \newpage
  \captionof{supplement}{%
    Codon usage frequency for each amino acid
    and canonical core histone type.
  }
  \label{tab:histone-gene-codonusage}
  \input{results/table-codon_usage}

  \newpage
  \captionof{supplement}{%
    Catalogue of annotated human core histone variant genes, transcripts and encoded proteins.
    Catalogue of annotated human core histone variant genes, transcripts,
    and encoded proteins.
    Data obtained from NCBI RefSeq \citep{OLearyRefseq2016} on \protect\printdate{\SequencesDate{}}.
  }

  \label{tab:variant-catalogue}
  \input{results/table-variant_catalogue}

  \newpage
  \begin{supplement}
    \caption{
        Variations in canonical core histone gene annotations
        compared to expectation of single exon, single transcript, and stem-loop.
        Data obtained from NCBI RefSeq \citep{OLearyRefseq2016} on \protect\printdate{\SequencesDate{}}.
    }
    \label{tab:curation-anomalies}
    \begin{itemize}
    \input{results/histone_insanities.tex}
    \end{itemize}
  \end{supplement}


\end{document}
