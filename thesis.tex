\batchmode  % silence the compiler
\documentclass[a4paper,twoside,openright]{memoir}

%% We need to follow NUIG structure and style in the thesis.  They
%% tell us how to do it but they don't really give a template or latex
%% document class to use.  I guess being able to format a document is
%% part.  Follows their documentation retrieved from ``University
%% Guidelines for Research Degree Programmes --- For Research
%% Students, Supervisors and Staff'', July 2016 edition, retrieved on
%% Mon 6 Feb 23:05:05 GMT 2017 from
%% HTTP://www.nuigalway.ie/media/graduatestudies/files/university_guidelines_for_research_degree_programmes.pdf
%%
%% From Section 6 - The PhD Examination Process
%%
%% 6.2.3 Directions on Format, Layout and Presentation
%%
%% The PhD thesis should not normally exceed 80,000 words, inclusive
%% of appendices, footnotes, tables and bibliography.  It is
%% university policy that the practice of engaging professional
%% editorial services to assist in writing the thesis is not
%% permitted.  There must be a title page which shall contain the
%% following information:
%%
%%  a. The full title (and subtitle, if any)
%%  b. The volume number and total number of volumes, if more than one
%%  c. The full name of the candidate, followed, if desired, by any
%%     degree and/or professional qualification(s)
%%  d. The name(s) of the supervisor(s), School(s), component
%%     Discipline(s), Institution
%%  e. The month and year of submission.
%%
%% Format and Layout
%%
%% The 'Table of Contents', which should not be over-detailed, shall
%% immediately follow the title page.  The text must be printed on good
%% quality (110g/m2) A4 size paper.  Line-spacing should be a maximum
%% of one-and-half; text must be left justified with a left-hand
%% margin of 4 cm and may be right justified.  An easily-readable
%% layout and double-sided printing are recommended for the body text.
%% For double sided printing ensure that the right hand margin is also
%% adequate for binding (i.e. a margin of 4 cm).  More compact formats,
%% with smaller font sizes, are usually appropriate for certain
%% sections, such as reference lists, bibliographies and some kinds of
%% appendices.  Pages must be numbered consecutively, with page numbers
%% located centrally at the bottom, and chapter headers at the top, of
%% each page.  Diagrams, graphs, photographs and tables should be
%% properly numbered and located in relation to the text.  The copies
%% of the thesis presented initially for examination must be spiral or
%% gum-bound.
%%
%% This is pretty much repeated in Appendix 1 --- Regulations for
%% Higher Research Degrees, Section 10 Submission of the Thesis.

\chapterstyle{veelo}

\OnehalfSpacing

\makepagestyle{NUIG}
\makeevenfoot{NUIG}{}{\thepage}{} % page numbers in the centre
\makeoddfoot{NUIG}{}{\thepage}{} % page numbers at the centre
\makeevenhead{NUIG}{\leftmark}{}{} % page marks in the edge
\makeoddhead{NUIG}{}{}{\rightmark} % page marks in the edge
\pagestyle{NUIG}

%% NUIG style requires 4cm for the spine margine (actually, it
%% requires 4 cm on the left margin, and 4 cm on the right margin if
%% it is to be double sided so that is is ``adequate for binding''.
%% Sounds like what matters is the spine margin and not right or left
%% margin).  Anyway, for an A4 page with font size 10pt (the default),
%% the memoir class default to ~3.5cm on the spine, and ~5.5cm on the
%% other side.  I kinda like the default format and text width so if
%% we add .5cm on one side, I'm taking it back from the other side.
%% Those default margins are computed based on the font size so the
%% text ends with roughly 66 characters per line.  Check the margin
%% sizes again if we have to change the font size.
\setlrmarginsandblock{4cm}{5cm}{*}
\checkandfixthelayout

%% remove colorlinks option when ready for print
\usepackage[final,hyperindex,hyperfootnotes,bookmarksnumbered,colorlinks]{hyperref}

\usepackage[T1]{fontenc}
\usepackage[utf8]{inputenc}
\usepackage{textcomp}

\usepackage{palatino}
\usepackage[euler]{textgreek}

\maxtocdepth{subsection}

\usepackage[final]{graphicx}

%% Input files that input others relative to themselves instead of
%% relative to the initial tex file.  Handy for each chapter but an
%% absolute requirement to include the pdf_tex figures from inkscape
%% which call includegraphics with the filename only.
\usepackage{import}

\usepackage{amsmath}

\usepackage[textsize=footnotesize]{todonotes}
  %% new command for box about missing references
  \newcommand{\addref}[1]{\todo[color=red!40,size=\tiny]{Add reference: #1}}

\usepackage{enumitem}       % so we can use the unboxed style when item names are too long
\usepackage{longtable}      % because memoir's ctabular does not work well with eqparbox
\usepackage{eqparbox}       % for adjusting size of table column (specially on appendices)
  %% Checks what LaTeX thinks is best for a column and save that value.
  %% It will then use it to calculate what's left of \textwidth, and
  %% use it for the other columns. See http://tex.stackexchange.com/questions/95397
  \newsavebox{\SolutionNameBox}
  \newcolumntype{\SolutionNameCol}{
    >{\begin{lrbox}{\SolutionNameBox}}l<{\end{lrbox}
    \eqmakebox[SolutionNameBox][l]{\unhcopy\SolutionNameBox}}
  }

\usepackage{tikz}

\newsubfloat{figure}        % subfigures with LaTeX

\usepackage{rotating}       % for sideways tables and figures
  \newcommand{\crows}[1]{\multicolumn{2}{c}{#1}}

%% Use agu style (American Geophysical Union) which only uses author
%% forenames and after too many authors, uses et. al.  All this helps
%% saves a lot of paper.
\usepackage[round]{natbib}
\bibliographystyle{agu}

\usepackage{seqsplit}
\usepackage{dnaseq}

\usepackage{siunitx}
  \DeclareSIUnit{\gn}{\textit{g$_n$}}   % standard gravity
  \DeclareSIUnit{\bp}{bp}               % base pairs
  \DeclareSIUnit{\cfu}{cfu}             % colony forming unit
  \DeclareSIUnit{\Molar}{\textsc{m}}
  \DeclareSIUnit{\mm}{\si{\milli}\si{\meter}}
  \DeclareSIUnit{\mM}{\si{\milli}\si{\Molar}}
  \DeclareSIUnit{\uM}{\si{\micro}\si{\Molar}}
  \DeclareSIUnit{\X}{\times}
  \newcommand{\dc}[1]{\SI{#1}{\degreeCelsius}}
  \newcommand{\pcent}[1]{\SI{#1}{\percent}}

%% This commands include the caption short description at the start of
%% long description and in bold.
\newcommand{\captionIntro}[2]{\caption[#1]{\textbf{#1.} #2}}
\newcommand{\captionofIntro}[3]{\captionof{#1}[#2]{\textbf{#2.} #3}}


%% Just like we have cite and citep to cite in text and between parentheses,
%% have the same for fref, tref, etc...
\newcommand{\frefp}[1]{(\fref{#1})}
\newcommand{\trefp}[1]{(\tref{#1})}
\newcommand{\Crefp}[1]{(\Cref{#1})}
\newcommand{\Srefp}[1]{(\Sref{#1})}
\newcommand{\Arefp}[1]{(\Aref{#1})}


\newcommand{\species}[1]{\textit{#1}}

%% NCBI Style Guide, Chapter 5 "Style Points and Conventions", recommends
%% italic for gene names (except in long list of genes), and roman for
%% protein names.
\newcommand{\gene}[1]{\textit{#1}}
\newcommand{\protein}[1]{#1}

\newcommand{\Kon}{$K_{on}$}
\newcommand{\Koff}{$K_{off}$}

\newcommand{\G}[1]{G$_#1$}  % for G0, G1, and G2 phases

\renewcommand{\abstractname}{Summary}

%% make it easy to center any dedication
\newcommand{\dedication}[1]{
{\clearpage\mbox{}\vfill\centering #1 \par\vfill\clearpage}}

\usepackage{makecell}
\usepackage[UKenglish,abbreviations]{foreign}

\input{methods/results/software_versions}


\author{David Miguel Susano Pinto}
\newcommand{\supervisor}{Dr.~Andrew Flaus}
\newcommand{\cosupervisor}{Prof.~Kevin Sullivan}
\date{March 2013} % hopefully
\title{1461 days of rain} % working title for a build without errors
%% TODO possible titles
%%
%% Structure--function relationships in chromatin probed by quantitative dynamics in live cell nuclei.
%% Does not account for the the histone cataloguing part. Plus there wasn't that much
%% study on the structure function relationships as it was mostly methods and things not
%% working great...
%%
%% the main thing behind the thesis is automation (we are not monkeys) and quantitative analysis. Those
%% two words should probably go on the title

%% Note: this thesis has easter eggs
\begin{document}
  \frontmatter

  \maketitle

  \clearpage
  \tableofcontents
  \clearpage
  \listoffigures
  \clearpage
  \listoftables

  \clearpage
  \begin{abstract}  % limit of 300 words

    Chromatin is a dynamic complex that controls access to genetic information by
    undergoing structural reconfigurations. Understanding this dynamics will give
    us a better insight into the biological implications of chromatin organization.

    Quantitative fluorescence microscopy has been used extensively to obtain insights
    into the dynamics of multiple proteins in live cells. Despite the large advances on
    model design, fluorophores and imaging capabilities, limitations are still
    encountered that can lead to misinterpretation of data.

    By using histone proteins with extremely slow exchanging rates we have tested the
    limitations of Fluorescence Recovery After Photobleaching (FRAP) and developed
    approaches to overcome some of them. Importantly, we have shown that movement of
    chromatin does not allow for measurements of histone dynamics on the FRAP time scale.

    By combining FRAP with a FRAP variant technique using photoactivatable proteins, we have
    set up a framework for a new model capable to simultaneously estimate both \Kon
    and \Koff. For this purpose we have tested the photoconvertible protein mEos2,
    and then constructed a new fluorescent protein combining PA--GFP with mRuby in tandem,
    to avoid the limitations of mEos2.

    Finally, we have undertaken a detailed catalogue of the human histone genes and completed
    their annotations. Based on the ``reproducible research'' concept, we made this
    catalogue not only a self-updatable paper, but also a model for similar projects
    which can be continually improved driven by genome annotations. As proof of concept, we
    are making a catalogue of the current mouse and chicken histone genes by minimal
    adjustment of the code we produced for the human homologues.

    In these studies we have tested the limits of several existing models by designing
    novel reagents, software and approaches in the field of chromatin dynamics.

  \end{abstract}

  \mainmatter

  \chapter{Introduction}
\label{ch:intro}

\epigraph{You need a David filter or you'll go crazy.}{Holger Stephan}
%% because this is the chapter where I'll talk a lot
%%
%% "Don't indulge in sesquipedalian lexicological constructions" William Safire's Rules for Writers
%%
%% "A month in the laboratory can often save an hour in the library" F. H. Westheimer
%%
%% "Science is what we understand well enough to explain to a computer, Art is all the rest."
%% by Donald E. Knuth (foreword to “A=B” by Petkovsek, W and Z)
%%
%% "By relieving the brain of all unnecessary work, a good notation sets it free to
%% concentrate on more advanced problems, and, in effect, increases the mental power
%% of the race." Quoted in P. Davis and R. Hersh The Mathematical Experience (Boston 1981)
%%
%% "The gall of them, fighting back!" -- Tyrion Lannister

%% FIXME 2nd year report abstract
%%The nucleosome is the fundamental structural unit of chromatin and the foundation
%%of it dynamics. Certain mutants have been shown to have a large effect on the
%%stability of the nucleosome \textit{in vitro} and to mimic chromatin remodelling
%%complexes in yeast.
%%
%%Fluorescence Recovery After Photobleaching (FRAP) has been previously used to
%%measure kinetics of endogenous histones. We are attempting to use it to measure
%%differences between wild type histones and mutants showing unusual \textit{in vitro} dynamics.
%%
%%We have developed two programs on the GNU Octave programming language, CropReg and
%%FRAPINATOR which was released under GPL. CropReg is able to track the nuclei of moving
%%cells. FRAPINATOR is able to do all the image processing, data extraction and fitting
%%of our FRAP experiments. These have revealed significant challenges to the determination
%%of relevant kinetic parameters, principally due to movement of chromatin.







\section{Chromatin organization}
  Chromatin is a dynamic complex that controls access to genetic information by
  undergoing reconfiguration of its structure. Since nucleosomes are the chromatin
  basic structural unit, their structural properties are on the basis of such reconfigurations.

  \subsection{Chromatin as substrate}
  \subsection{Nucleosome structure}
  \subsection{Chromatin remodelling}

\section{Centromeres --- specialized chromatin}
  \subsection{Centromere and kinetochore structure}
    \subsubsection{Timing of assembly}
  \subsection{Histone Fold Domain}
    \subsubsection{CENP--T/W}
    \subsubsection{CENP--S/X}

\section{In vivo techniques for quantitative protein dynamics}
  \subsection{Fluorescent proteins}
  \subsection{FRAP}

    Fluorescence Recovery After Photobleaching has been extensively used to obtain
    qualitative and quantitative insight on the kinetic properties of proteins. Development of
    this technique has led to complex models that are both more precise and accurate than
    simple models based on inverse exponential decay. These take into account important
    parameters that were previously discarded but have since been shown as important, such as
    diffusion and the bleach spot profile shape.

    However, despite the their sophistication, these models of recovery make certain
    important assumptions:
    \begin{itemize}
      \item the biological system has reached equilibrium before photobleaching;
      \item total amount of both Fluorescent Protein (FP) fusion protein and its
            binding sites remain constant over the time course of the recovery;
      \item the binding sites are part of a large, relatively immobile complex, at
            least on the time- and length-scale of the recovery.
    \end{itemize}

    FRAP has been successfully used before to show that different core histones have
    different kinetics and populations. We expect to be able to use it for \textit{in vivo}
    validation of previously identified differences between wild type and mutant histones.

  \subsection{iFRAP}
  \label{sec:iFRAP}

  \subsection{FLAP}












  \chapter{Materials and Methods}
\label{ch:methods}

\section{Chemicals, solutions and reagents used}
  The chemicals used during this project were supplied by Sigma unless otherwise
  stated. Solutions were prepared according to the \Aref{app:solutions} with
  dH$_2$O and autoclaved prior to use when appropriate.
  
  Restriction enzymes, T4 DNA ligase, DNA ladders and calf intestinal alkaline phosphatase
  were from New England BioLabs. Protein ladder was SeeBlue Plus2.


\section{Molecular biology}
  \subsection{Preparation of competent bacteria}
    Competent \species{E.~coli} cells were prepared from a culture of Invitrogen's One Shot
    TOP10 Chemically Competent \species{E.~coli}. This culture was set on \SI{1}{\l} of LB at
    \SI{37}{\dc} until it reach an OD\SI{600}{\nm} of \numrange{0.35}{0.40} when it was
    transferred to centrifuge tubes and chilled on ice. All following steps were carried at a temperature of
    \SI{4}{\dc} with previously chilled equipment and solutions.
    
    The culture was pelleted by centrifugation at \SI{1500}{\gn} for 20 minutes, the supernatant
    decanted, the pellet resuspended in \SI{400}{\ml} of TFB I and then incubated on ice for
    5 min. Another centrifugation was made at same speed and temperature, the supernatant decanted,
    the pellet resuspended in 40 mL of TFB II and incubated on ice for another 5 minutes. Aliquots of
    \SI{200}{\ul} were transferred to eppendorf tubes and immediately transferred to \SI{-80}{\dc}
    for storage.
    
    To assess transformation efficiency, pBOS-H2B-GFP was used and a value of approximately
    \SI{3d6}{colonies\per\mg} plasmid, was found following the procedure detailed in.
    To confirm the absence of antibiotic-resistant contamination, untransformed cells were plated
    on selective plates.

  \subsection{\species{E.~coli} transformations}
    Competent cells were thawed on ice and split into aliquots of \SI{50}{\ul} to pre-chilled
    eppendorf tubes before adding \SI{1}{\ml} of DNA. The transformation mixture was
    incubated on ice for 30 minutes, followed by 60 seconds at \SI{42}{\dc} and another 5 minutes on ice.
    
    \SI{300}{\ul} of non-selective LB was added to each tube and incubated with vigorous
    shaking for 45 minutes at \SI{37}{\dc}, after which samples were plated onto the appropriate
    antibiotic containing plates and incubated overnight at \SI{37}{\dc}.

  \subsection{Plasmid DNA preparation}
    Plasmid DNA was prepared depending on the situation with kits from QIAGEN (QIAprep Spin Miniprep,
    HiSpeed Plasmid, QIAquick Gel extraction and QIAprep PCR purification) following the
    manufacturer's instructions. Once prepared, DNA was stored at \SI{-20}{\dc}.
    
    DNA concentrations of \SIrange{40}{60}{\ng\per\ul} were routinely obtained from all DNA
    preparations. DNA concentrations were measured with spectrophotometer (NanoDrop 1000 Spectrophotometer
    from Thermo Scientific).

  \subsection{Ethanol precipitation}
    \label{sec:ethanol-precipitation}
    The DNA solution was mixed with \num{2.5} volumes of \SI{100}{\percent} ethanol and \num{1/10} volumes
    of NaCH$_3$COO, stored at \SI{-20}{\dc} for 1 hour and then centrifuged at \SI{18000}{\gn} for
    20 minutes at \SI{4}{\dc}. The supernatant was removed, and the pellet left to dry until all
    traces of solvent evaporated before resuspension in the desired solvent (usually
    H$_2$O).

  \subsection{Phenol:chloroform extraction}
    \label{sec:phenol-extraction}
    To extract proteins, an equal volume of phenol:chloroform was added and the
    mixture centrifuged at \SI{6000}{\gn} for 15 minutes. The top aqueous phase (chloroform)
    was pipetted to a new tube and the process repeated a total of 3 times.

  \subsection{Agarose gels electrophoresis}
    Agarose gels of concentrations ranging from \SIrange{0.6}{2.0}{\percent} were prepared in \SI{1}{$\times$} TBE
    buffer supplemented with ethidium bromide to a final concentration of \SI{0.5}{\ug\per\ml}.
    The gels ran in chambers with \SI{1}{$\times$} TBE buffer at \SIrange{60}{100}{\volt} until the required
    separation was achieved and were visualized in a transilluminator (ChemiImager 5500 from Alpha Innotech) with UV light.

  \subsection{DNA sequencing and oligonucleotide preparation}
    DNA sequencing was performed by Cogenics to ensure that clones contained no
    unexpected mutations.
    
    Oligonucleotides were obtained from MWG Biotech AG and listed in \Aref{app:primers}.
    All were stored at a final concentration of \SI{100}{\micro\Molar} in distilled H$_2$O
    and at \SI{-20}{\dc}.

  \subsection{Polimerase chain reaction}
    \subsubsection{Gene amplification}
    \subsubsection{PCR mutagenesis}
    \subsubsection{Screening}

  \subsection{Western blotting}
    \subsubsection{Protein concentration determination}
      Concentration of protein was measured with Bradford reagent. \SI{2}{\ul} of the
      sample after sonication (\Sref{sec:cell-extract}) was mixed with \SI{48}{\ul} of H$_2$O
      and \SI{50}{\ul} of NaOH and incubated at \SI{65}{\dc} for 8 minutes before adding
      \SI{900}{\ul} of Bradford reagent from Pierce. The mixture was transferred to plastic
      cuvettes and the absorvance at \SI{595}{\nm} measured in a Shimadzu spectrophotometer. The
      values obtained were interpolated from a standard curve prepared using known concentrations
      of BSA.
    
    \subsubsection{SDS--PAGE}
      Resolving and stacking SDS--PAGE gels of \SIrange{15}{5}{\percent} respectively, both with a
      cross-linking ratio of \num{37.5}:1. The resolving gel was poured directly after addition
      of TEMED and it was covered with a layer of isopropanol during polymerisation
      to ensure a sharp interface between the resolving and stacking layers.
      Protein samples and markers were boiled at \SI{99}{\dc} for 3 minutes and each
      was loaded twice, with volumes for \SI{3.3}{\ug} and \SI{16.5}{\ug} of protein. Gels ran at
      \SI{180}{\volt} for 1 hour in \SI{1}{$\times$} TG buffer.
    
    \subsubsection{Protein transfer}
      Protein transfer occurred through the wet transfer system. The SDS-PAGE gel
      was placed onto pre-cut nitrocellulose transfer membrane previously soaked in
      transfer buffer. It was then set between a pair of extra thick blotting paper and
      cushions before being placed inside a transfer apparatus. The transfer ran at
      \SI{4}{\dc} for 60 minutes.
    
    \subsubsection{Probing of blot with antibody}
      Blocking of the membrane was performed with \SI{10}{\percent} non-fat dry milk in \SI{1}{$\times$} TBST
      at room temperature for 30 minutes. Blocking was followed by primary antibody
      incubation which occurred in \SI{5}{\percent} non-fat dry milk in \SI{1}{$\times$} TBST overnight at
      \SI{4}{\dc}. Concentrations of antibody used were 1:500 and 1:20000 for anti-GFP
      (catalogue number 11 814 460 001 from Roche) and anti-H3 (code ab1791 from
      abcam). The membrane was then washed with \SI{1}{$\times$} TBST for 15 minutes 3 times
      before the secondary antibody incubation which occurred in \SI{5}{\percent} non-fat dry
      milk in \SI{1}{$\times$} TBST for 1 hour. The membrane was washed once more in the same
      conditions as before for the detection. All blocking, antibody incubation and
      washing steps occurred on a rocker.\todo{should list antibodies and their conditions on appendix}
      
      Detection was performed using the SuperSignal West Pico Chemiluminescent
      Substrate from Pierce, adding 1:1 of the solutions and allowing it to incubate
      with the membrane for 5 minutes. The membrane was exposed to x-ray films for
      10, 60, 5, 180 and 1800 seconds which were then developed.


\section{Tissue culture}
  \subsection{Cell culture}
    HeLa cells were supplied by Agnieszka Kaczmarczyk from National University
    of Ireland, Galway. They were maintained at \SI{37}{\dc} and \SI{5}{\percent} CO$_2$ in \SI{10}{\cm}
    diameter plates with \SI{10}{\ml} of growth medium. Dulbecco's Phosphate Buffered
    Saline (DPBS) and trypsin--EDTA solutions were used, respectively, to wash
    and split the cells 1:10 once they reached a confluence of \SIrange{80}{90}{\percent}.

  \subsection{Cell stock storage}
    For long-term storage of HeLa cell lines, they were grown until they reached
    a confluence of \SIrange{80}{90}{\percent} and then trypsinized as usual. A volume of Freezing
    Media was added, equal to the volume of trypsin--EDTA, and \SI{2}{\ml} aliquots of
    cells transferred to cryotubes. Tubes were immediately wrapped in cotton and
    placed at \SI{-80}{\dc}.

  \subsection{Whole cell extract}
    \label{sec:cell-extract}
    To obtain whole cell extracts, HeLa cells were trypsinized as usual. Growth
    medium added and the suspension was centrifuged at \SI{900}{\gn} for 10 minutes. Further
    steps were carried at \SI{4}{\dc} and with previously chilled reagents. The supernatant
    was discarded and the pellet resuspended in \SI{500}{\ul} of chilled PBS before being
    sonicated 3 times at \SI{40}{\percent} amplitude for 10 seconds. Avoiding the foam formed at
    the top, \SI{300}{\ul} of suspension were transferred from the bottom of the tube to a
    new eppendorf and mixed with an equal volume of Laemmli buffer before being
    stored at \SI{-80}{\dc}. \SI{2}{\ul} from the suspension was also transferred to a new tube
    for determination of protein concentration.

  \subsection{Genomic DNA extraction}
    To extract genomic DNA of HeLa cells, they were trypsinized as usual and
    growth medium was added in the end before counting with an hemocytometer.
    Cells were centrifuged at \SI{1500}{\gn} for 10 minutes at \SI{4}{\dc}, the supernatant discarded
    and the pellet resuspended in TE buffer to achieve a desired concentration of
    \SI{4e7}{cells\per\ml}. 9 volumes of Genomic lysis buffer was added and the mixture
    incubated at \SI{37}{\dc} for 90 minutes. Proteinase K was added to a final concentration of
    \SI{100}{\ug\per\ml} and the mixture incubated at 50 °C for 3 h and swirled every 20 min.
    DNA was then extracted by phenol:chloroform (\Sref{sec:phenol-extraction}) and purified by ethanol
    precipitation (\Sref{sec:ethanol-precipitation}).

  \subsection{Viable cells count}
    Trypan blue was used to assess the number of viable cells. After trypsination
    as usual, cells were diluted in growth medium, to a final concentration of \SI{2e5}{cells\per\ml}.
    To \SI{0.5}{\ml} of the cell suspension, was added \SI{0.1}{\ml} of \SI{0.4}{\percent} Trypan
    Blue Stain and the mixture left for 5 minutes at room temperature before counting
    the cells in an hemocytometer making a distinction between stained (non-viable)
    and non-stained (viable) cells.

  \subsection{Transfection of HeLa cells}
    Cells were transfected using Lipofectamine 2000, a cationic lipid reagent, from
    Invitrogen. Cells were trypsinized as usual on the day before transfection and
    replated on 6 well plates (surface area of \SI{9.5}{\square\cm\per well}) with \SI{2.5}{\ml} of growth
    medium so they would be \SI{90}{\percent} confluent on the following day. For each well,
    two tubes with \SI{250}{\ul} of transfection medium were prepared, one with \SI{7.5}{\ul}
    of Lipofectamine 2000 and another with \SI{3750}{\ng} of DNA (ethanol precipitation
    (\Sref{sec:ethanol-precipitation}) was used to increase the DNA concentration for values around \SI{500}{\ng\per\ul}
    before transfection). Both tubes were incubated at room temperature for 5 minutes,
    mixed together, and incubated again at room temperature for 20 minutes. Cells
    were washed with DPBS during this time and growth medium switched to \SI{2}{\ml}
    of transfection medium. The mixture was then added to the cells medium who
    were incubated at \SI{37}{\dc} for 6 hours after which time it was switched back to
    \SI{0.5}{\ml} of growth medium.

  \subsection{Fixation and staining of HeLa cells}
    For microscopy visualization, cells grown directly on top of HCl washed coverslips since
    HeLa cells have difficulty attaching to glass. At least 24 hours passed
    between the plating and fixation. Growth medium was removed and the cells
    washed with PBS once before incubation with \SI{4}{\percent} formaldehyde in PBS
    for 4 minutes. The solution was removed and the cells washed with H$_2$O 2 more
    times, after which coverslips were removed from the wells and left to air dry.
    For each coverslip, \SI{2}{\ul} of SlowFade Light Antifade kit from Molecular Probes
    was used for mounting the coverslip on a microscope slide. DAPI was added
    to the mounting media when needed. Coverslips were then sealed with a 1:1
    mixture of clear nail polish and acetone and stored on a dark box at \SI{4}{\dc}.


\section{Microscopy}
\section{Software used}
  %% we probably should use a script to get the actual version when building the document
  %% this versions numbers are the ones I'm using at the moment... 4 years ago, they didn't exist
  %% must redo the analysis and get the new version numbers.
  The European Molecular Biology Open Software Suite (EMBOSS) version 6.4.0
  was used for analysis of codon usage, RNA folding, sequence alignment, reading of abi
  files, search for restriction sites, prediction of molecular weight and
  other trivial tasks.

  The Perl language and the BioPerl module version 1.6.901 was also extensively used for automation of
  several tasks.
  
  GNU Octave version 3.6.2 and the image package version 2.0.0.
  
  ImageJ version 1.47h packaged though FIJI.

\section{Software developed}





  \chapter{Histone catalogue}
\label{ch:catalog}


  \chapter{FRAP with histones}
\label{ch:frap}


  \chapter{Dynamics of CENP HFD}
\label{ch:cenp}

\section{The FRAP/FLAP model}
\section{Development of MaryI}
\section{HFD association rates}
\section{HFD stoichiometry on the kinetochore}
\section{Involvement of FACT in assembly of CENP–T/W complex}
\section{Cell cycle}
  \subsection{Timing of assembly with respect to DNA replication}
  \subsection{Dynamics as a function of the cell cycle phase}
\section{Requirements for CCAN in HFD CENP assembly}
  \subsection{CENP–T/W influence on CENP–S/X}
\section{Conclusions}

  \chapter{GNU Octave}
\label{ch:octave}

%% if there's enough time, write a chapter on how the image package can be used
%% for image processing and microscopy, listing all the functions that I wrote
%% This would read like a small manual of the image package

  \appendix
\chapter{Solutions}
  \label{app:solutions}
  
  %% an alternative method to make this lists is with description environment
  %%
  %% \begin{description}
  %%   \item[Freezing media] \hfill \\
  %%     91\% FCS;
  %%     10\% DMSO
  %%   \item[Growth medium (HeLa)] \hfill \\
  %%     87\% DMEM;
  %%     10\% FCS;
  %%     1\% NEAA;
  %%     ...
  %% \end{description}
  
  %% This requires the eqparbox package. It checks what LaTeX thinks is best for a column
  %% and save that value. It will then use it to calculate what's left of \textwidth, and
  %% use it for the other columns. See http://tex.stackexchange.com/questions/95397
  %% It doesn't work with memoir's ctabular so we are using longtable instead
  \newsavebox{\firstentrybox}
  \newcolumntype{N}{%
    >{\begin{lrbox}{\firstentrybox}}%
      l%
    <{\end{lrbox}%
    \eqmakebox[firstentry][l]{\unhcopy\firstentrybox}}}

  \begin{longtable}{>{\bfseries}N p{\dimexpr(\textwidth-\eqboxwidth{firstentry}-4\tabcolsep)}}
    \toprule
    Name & Recipe\\
    \midrule
    2YT broth               & \\
    %% FIXME get 2YT media recipe
    
    DMEM                    & \SI{4.5}{\g\per\l}   glucose;
                              \SI{110}{\mg\per\l}  L-glutamine;
                              \SI{584}{\ug\per\l}  sodium pyruvate;
                              \SI{15.9}{\mg\per\l} phenol red.\\
    
    DNA loading buffer (10$\times$) & \pcent{25} Ficoll ($w/v$);
                              \SI{100}{\mM}      Tris--HCl pH=\num{7.4};
                              \SI{100}{\mM}      EDTA.\\
    
    Freezing media          & \pcent{90} FCS;
                              \pcent{10} DMSO.\\
    
    Growth medium (HeLa)    & \pcent{89}            DMEM;           % 500ml
                              \pcent{9}             FCS;            % 50ml
                              \SI{1}{$\times$}      NEAA solution;  % 5.5mL
                              \SI{50}{units\per\ml} penicillin;     % 5.5mL (Pen/Strep solution)
                              \SI{50}{\ug\per\ml}   streptomycin.\\ % 5.5mL (Pen/Strep solution)
    
    Growth medium (horse)   & \pcent{81}            DMEM;           % 500ml
                              \pcent{16}            FCS;            % 100ml
                              \SI{2}{$\times$}      NEAA solution;  %  12mL
                              \SI{50}{units\per\ml} penicillin;     % 5.5mL (Pen/Strep solution)
                              \SI{50}{\ug\per\ml}   streptomycin.\\ % 5.5mL (Pen/Strep solution)
    
    %% TODO check exactly the LB recipes
    LB agar                 & \pcent{2}   LB ($w/v$);
                              \pcent{1.5} agar.\\
    
    LB broth                & \pcent{2} LB ($w/v$).\\
    
    PBS                     & \\
    %% TODO check exact PBS recipe from the tablets
    
    Ponceau S solution      & \pcent{5} Ponceau-S ($w/v$);
                              \pcent{5} Acetic acid ($v/v$).\\
    
    Running buffer          & \SI{1}{$\times$} TG;
                              \pcent{0.1}      SDS ($w/v$).\\
    
    SSC (Saline Sodim Citrate) & \SI{150}{\mM} NaCl;
                              \SI{15}{\mM}     Trisodium citrate.\\
    
    TAE (Tris Acetate EDTA) & \\
    %% TODO get TAE buffer recipe
    
    TBE (Tris Borate EDTA)  & \SI{89}{\mM} Tris;
                              \SI{89}{\mM} Boric acid;
                              \SI{2}{\mM}  EDTA.\\
    
    TBS (Tris Buffered Saline)  & \SI{50}{\mM} Tris--HCl pH=\num{7.5};
                              \SI{100}{\mM}    NaCl.\\
    
    PBS-T (TBS-Tween)       & \pcent{99.95} PBS ($v/v$);
                              \pcent{0.05}  Tween 20 ($v/v$).\\
    
    TG (Tris Glycine)       & \SI{25}{\mM}  Tris;
                              \SI{192}{\mM} glycine.\\
    
    Transfer buffer         & \SI{1}{$\times$} TG;
                              \pcent{15} methanol ($v/v$).\\
    \bottomrule
  \end{longtable}
    

\chapter{List of plasmids}
  \label{app:plasmids}
  %% generate this automatically from SQlite db
  %% should we print the maps on genbank format, just the genbank features
  %% table, or a draw of the map, poiting to the database for the sequence
  %% details?

\chapter{List of primers}
  \label{app:primers}
  %% generate this automatically from SQlite db


  \backmatter

  \bibliographystyle{plainnat}
  \bibliography{references}
\end{document}
