\documentclass[12pt,a4paper,twoside,openright,draft]{memoir}

%% We need to follow NUIG structure and style in the thesis.  They
%% tell us how to do it but they don't really give a template or latex
%% document class to use.  I guess being able to format a document is
%% part.  Follows their documentation retrieved from ``University
%% Guidelines for Research Degree Programmes --- For Research
%% Students, Supervisors and Staff'', July 2016 edition, retrieved on
%% Mon 6 Feb 23:05:05 GMT 2017 from
%% HTTP://www.nuigalway.ie/media/graduatestudies/files/university_guidelines_for_research_degree_programmes.pdf
%%
%% From Section 6 - The PhD Examination Process
%%
%% 6.2.3 Directions on Format, Layout and Presentation
%%
%% The PhD thesis should not normally exceed 80,000 words, inclusive
%% of appendices, footnotes, tables and bibliography.  It is
%% university policy that the practice of engaging professional
%% editorial services to assist in writing the thesis is not
%% permitted.  There must be a title page which shall contain the
%% following information:
%%
%%  a. The full title (and subtitle, if any)
%%  b. The volume number and total number of volumes, if more than one
%%  c. The full name of the candidate, followed, if desired, by any
%%     degree and/or professional qualification(s)
%%  d. The name(s) of the supervisor(s), School(s), component
%%     Discipline(s), Institution
%%  e. The month and year of submission.
%%
%% Format and Layout
%%
%% The 'Table of Contents', which should not be over-detailed, shall
%% immediately follow the title page.  The text must be printed on good
%% quality (110g/m2) A4 size paper.  Line-spacing should be a maximum
%% of one-and-half; text must be left justified with a left-hand
%% margin of 4 cm and may be right justified.  An easily-readable
%% layout and double-sided printing are recommended for the body text.
%% For double sided printing ensure that the right hand margin is also
%% adequate for binding (i.e. a margin of 4 cm).  More compact formats,
%% with smaller font sizes, are usually appropriate for certain
%% sections, such as reference lists, bibliographies and some kinds of
%% appendices.  Pages must be numbered consecutively, with page numbers
%% located centrally at the bottom, and chapter headers at the top, of
%% each page.  Diagrams, graphs, photographs and tables should be
%% properly numbered and located in relation to the text.  The copies
%% of the thesis presented initially for examination must be spiral or
%% gum-bound.
%%
%% This is pretty much repeated in Appendix 1 --- Regulations for
%% Higher Research Degrees, Section 10 Submission of the Thesis.

\chapterstyle{veelo}

\OnehalfSpacing

\makepagestyle{NUIG}
\makeevenfoot{NUIG}{}{\thepage}{} % page numbers in the centre
\makeoddfoot{NUIG}{}{\thepage}{} % page numbers at the centre
\makeevenhead{NUIG}{\leftmark}{}{} % page marks in the edge
\makeoddhead{NUIG}{}{}{\rightmark} % page marks in the edge
\pagestyle{NUIG}

%% NUIG style requires 4cm for the spine margine (actually, it
%% requires 4 cm on the left margin, and 4 cm on the right margin if
%% it is to be double sided so that is is ``adequate for binding''.
%% Sounds like what matters is the spine margin and not right or left
%% margin).  Anyway, for an A4 page with font size 10pt (the default),
%% the memoir class default to ~3.5cm on the spine, and ~5.5cm on the
%% other side.  I kinda like the default format and text width so if
%% we add .5cm on one side, I'm taking it back from the other side.
%% Those default margins are computed based on the font size so the
%% text ends with roughly 66 characters per line.  Check the margin
%% sizes again if we have to change the font size.
\setlrmarginsandblock{4cm}{5cm}{*}
\checkandfixthelayout

%% remove colorlinks option when ready for print
\usepackage[final,hyperindex,hyperfootnotes,bookmarksnumbered,colorlinks]{hyperref}

\usepackage[T1]{fontenc}
\usepackage[utf8]{inputenc}
\usepackage{textcomp}

\usepackage{palatino}
\usepackage[euler]{textgreek}

\maxtocdepth{subsection}

\usepackage[final]{graphicx}

%% Input files that input others relative to themselves instead of
%% relative to the initial tex file.  Handy for each chapter but an
%% absolute requirement to include the pdf_tex figures from inkscape
%% which call includegraphics with the filename only.
\usepackage{import}

\usepackage{amsmath}

\usepackage[textsize=footnotesize]{todonotes}
  %% new command for box about missing references
  \newcommand{\addref}[1]{\todo[color=red!40,size=\tiny]{Add reference: #1}}

\usepackage{enumitem}       % so we can use the unboxed style when item names are too long
\usepackage{longtable}      % because memoir's ctabular does not work well with eqparbox
\usepackage{eqparbox}       % for adjusting size of table column (specially on appendices)
  %% Checks what LaTeX thinks is best for a column and save that value.
  %% It will then use it to calculate what's left of \textwidth, and
  %% use it for the other columns. See http://tex.stackexchange.com/questions/95397
  \newsavebox{\SolutionNameBox}
  \newcolumntype{\SolutionNameCol}{
    >{\begin{lrbox}{\SolutionNameBox}}l<{\end{lrbox}
    \eqmakebox[SolutionNameBox][l]{\unhcopy\SolutionNameBox}}
  }

\usepackage{tikz}

\newsubfloat{figure}        % subfigures with LaTeX

\usepackage{rotating}       % for sideways tables and figures
  \newcommand{\crows}[1]{\multicolumn{2}{c}{#1}}

%% Use agu style (American Geophysical Union) which only uses author
%% forenames and after too many authors, uses et. al.  All this helps
%% saves a lot of paper.
\usepackage[round]{natbib}
\bibliographystyle{agu}

\usepackage{seqsplit}
\usepackage{dnaseq}

\usepackage{siunitx}
  \DeclareSIUnit{\gn}{\textit{g$_n$}}   % standard gravity
  \DeclareSIUnit{\bp}{bp}               % base pairs
  \DeclareSIUnit{\cfu}{cfu}             % colony forming unit
  \DeclareSIUnit{\Molar}{\textsc{m}}
  \DeclareSIUnit{\mm}{\si{\milli}\si{\meter}}
  \DeclareSIUnit{\mM}{\si{\milli}\si{\Molar}}
  \DeclareSIUnit{\uM}{\si{\micro}\si{\Molar}}
  \DeclareSIUnit{\X}{\times}
  \newcommand{\dc}[1]{\SI{#1}{\degreeCelsius}}
  \newcommand{\pcent}[1]{\SI{#1}{\percent}}

%% This commands include the caption short description at the start of
%% long description and in bold.
\newcommand{\captionIntro}[2]{\caption[#1]{\textbf{#1.} #2}}
\newcommand{\captionofIntro}[3]{\captionof{#1}[#2]{\textbf{#2.} #3}}


%% Just like we have cite and citep to cite in text and between parentheses,
%% have the same for fref, tref, etc...
\newcommand{\frefp}[1]{(\fref{#1})}
\newcommand{\trefp}[1]{(\tref{#1})}
\newcommand{\Crefp}[1]{(\Cref{#1})}
\newcommand{\Srefp}[1]{(\Sref{#1})}
\newcommand{\Arefp}[1]{(\Aref{#1})}


\newcommand{\species}[1]{\textit{#1}}

%% NCBI Style Guide, Chapter 5 "Style Points and Conventions", recommends
%% italic for gene names (except in long list of genes), and roman for
%% protein names.
\newcommand{\gene}[1]{\textit{#1}}
\newcommand{\protein}[1]{#1}

\newcommand{\Kon}{$K_{on}$}
\newcommand{\Koff}{$K_{off}$}

\newcommand{\G}[1]{G$_#1$}  % for G0, G1, and G2 phases

\renewcommand{\abstractname}{Summary}

%% make it easy to center any dedication
\newcommand{\dedication}[1]{
{\clearpage\mbox{}\vfill\centering #1 \par\vfill\clearpage}}

\usepackage{makecell}
\usepackage[UKenglish,abbreviations]{foreign}

\input{methods/results/software_versions}

%% Seems like subinputfrom (import package) does not work so well in
%% the preamble, so we still have to specify the full path on the
%% filename.  Relative paths inside the input'ed' files works fine.
\subinputfrom{histone-catalogue/}{histone-catalogue/preamble}

\author{David Miguel Susano Pinto}
\newcommand{\supervisor}{Dr.~Andrew Flaus}
\newcommand{\cosupervisor}{Prof.~Kevin Sullivan}
\date{March 2013} % hopefully
\title{1461 days of rain} % working title for a build without errors
%% TODO possible titles
%%
%% Structure--function relationships in chromatin probed by quantitative dynamics in live cell nuclei.
%% Does not account for the the histone cataloguing part. Plus there wasn't that much
%% study on the structure function relationships as it was mostly methods and things not
%% working great...
%%
%% the main thing behind the thesis is automation (we are not monkeys) and quantitative analysis. Those
%% two words should probably go on the title

\begin{document}
  \frontmatter

  \maketitle

  %% \dedication{Just get it done.}

  \clearpage
  \tableofcontents
  \clearpage
  \listoffigures
  \clearpage
  \listoftables

  \clearpage
  \begin{abstract}  % limit of 300 words

    Chromatin is a dynamic complex that controls access to genetic information by
    undergoing structural reconfigurations. Understanding this dynamics will
    provide insights into the biological implications of chromatin organisation.

    We have undertaken a detailed catalogue of
    the human core histone genes and contributed to
    their annotations. Based on the reproducible research concept,
    we produced this
    catalogue in the format of a self-updatable manuscript
    as a model for similar projects
    which can be continually improved driven by genome annotations.
    As proof of concept, we
    also implemented a catalogue of the current mouse
    histone genes by minimal adjustment of the code.

    Quantitative fluorescence microscopy has been used extensively to obtain insights
    into the dynamics of multiple proteins in live cells. Despite the large advances in
    model design, fluorophores and imaging capabilities, limitations are still
    encountered that can lead to misinterpretation of data.
    By using histone proteins with extremely slow exchanging rates we have tested the
    limitations of Fluorescence Recovery After Photobleaching (FRAP) and developed
    approaches to overcome some of them. Importantly, we have shown that movement of
    chromatin does not allow for measurements of histone dynamics on the FRAP time scale.

    \todo[inline]{3 sentences summarising software}

    In these studies we have tested the limits and developed new
    approaches to existing methods of chromatin analysis by designing
    novel reagents and software for the field of chromatin dynamics.

  \end{abstract}

  \mainmatter

  \chapter{Introduction}
  \subinputfrom{intro/}{intro}

  \subinputfrom{methods/}{methods}

  %% To write this chapter we defined a bunch of commands that are
  %% unique to human genome.  We also want to use the same commands
  %% with values for mice genome but the command names.  So we define
  %% the commands locally to this group.  See:
  %% https://tex.stackexchange.com/questions/353002/redefine-existing-commands-outside-preamble
  \chapter[Human Canonical Core Histone Catalogue]%
          [Histone Catalogue]%
          {Human Canonical Core Histone Catalogue}
  \label{ch:histone-catalogue}
  \begingroup
  \newcommand{\ResultsDir}{results-homo-sapiens}
  \newcommand{\FigsDir}{figs-homo-sapiens}
  \newcommand{\ReferenceDir}{data/reference-homo-sapiens}
  \subinputfrom{histone-catalogue/}{variables}
  \subinputfrom{histone-catalogue/}{manuscript}
  \endgroup

  \chapter[Application of FRAP to Histones in Human Cell Nuclei]%
          [Application of FRAP to Histones]%
          {Application of FRAP to Histones in Human Cell Nuclei}
  \label{ch:kill-frap}
  \subinputfrom{kill-frap/}{kill-frap}

  \chapter[Software Tools for Image and Sequence Analysis]%
          [Software Tools]%
          {Software Tools for Image and Sequence Analysis}
  \label{ch:software}
  \subinputfrom{software/}{software}

  \chapter{Discussion and Future Perspectives}
  \subinputfrom{discussion/}{discussion}

  \appendix
\chapter{Solutions}
  \label{app:solutions}
  
  %% an alternative method to make this lists is with description environment
  %%
  %% \begin{description}
  %%   \item[Freezing media] \hfill \\
  %%     91\% FCS;
  %%     10\% DMSO
  %%   \item[Growth medium (HeLa)] \hfill \\
  %%     87\% DMEM;
  %%     10\% FCS;
  %%     1\% NEAA;
  %%     ...
  %% \end{description}
  
  %% This requires the eqparbox package. It checks what LaTeX thinks is best for a column
  %% and save that value. It will then use it to calculate what's left of \textwidth, and
  %% use it for the other columns. See http://tex.stackexchange.com/questions/95397
  %% It doesn't work with memoir's ctabular so we are using longtable instead
  \newsavebox{\firstentrybox}
  \newcolumntype{N}{%
    >{\begin{lrbox}{\firstentrybox}}%
      l%
    <{\end{lrbox}%
    \eqmakebox[firstentry][l]{\unhcopy\firstentrybox}}}

  \begin{longtable}{>{\bfseries}N p{\dimexpr(\textwidth-\eqboxwidth{firstentry}-4\tabcolsep)}}
    \toprule
    Name & Recipe\\
    \midrule
    2YT broth               & \\
    %% FIXME get 2YT media recipe
    
    DMEM                    & \SI{4.5}{\g\per\l}   glucose;
                              \SI{110}{\mg\per\l}  L-glutamine;
                              \SI{584}{\ug\per\l}  sodium pyruvate;
                              \SI{15.9}{\mg\per\l} phenol red.\\
    
    DNA loading buffer (10$\times$) & \pcent{25} Ficoll ($w/v$);
                              \SI{100}{\mM}      Tris--HCl pH=\num{7.4};
                              \SI{100}{\mM}      EDTA.\\
    
    Freezing media          & \pcent{90} FCS;
                              \pcent{10} DMSO.\\
    
    Growth medium (HeLa)    & \pcent{89}            DMEM;           % 500ml
                              \pcent{9}             FCS;            % 50ml
                              \SI{1}{$\times$}      NEAA solution;  % 5.5mL
                              \SI{50}{units\per\ml} penicillin;     % 5.5mL (Pen/Strep solution)
                              \SI{50}{\ug\per\ml}   streptomycin.\\ % 5.5mL (Pen/Strep solution)
    
    Growth medium (horse)   & \pcent{81}            DMEM;           % 500ml
                              \pcent{16}            FCS;            % 100ml
                              \SI{2}{$\times$}      NEAA solution;  %  12mL
                              \SI{50}{units\per\ml} penicillin;     % 5.5mL (Pen/Strep solution)
                              \SI{50}{\ug\per\ml}   streptomycin.\\ % 5.5mL (Pen/Strep solution)
    
    %% TODO check exactly the LB recipes
    LB agar                 & \pcent{2}   LB ($w/v$);
                              \pcent{1.5} agar.\\
    
    LB broth                & \pcent{2} LB ($w/v$).\\
    
    PBS                     & \\
    %% TODO check exact PBS recipe from the tablets
    
    Ponceau S solution      & \pcent{5} Ponceau-S ($w/v$);
                              \pcent{5} Acetic acid ($v/v$).\\
    
    Running buffer          & \SI{1}{$\times$} TG;
                              \pcent{0.1}      SDS ($w/v$).\\
    
    SSC (Saline Sodim Citrate) & \SI{150}{\mM} NaCl;
                              \SI{15}{\mM}     Trisodium citrate.\\
    
    TAE (Tris Acetate EDTA) & \\
    %% TODO get TAE buffer recipe
    
    TBE (Tris Borate EDTA)  & \SI{89}{\mM} Tris;
                              \SI{89}{\mM} Boric acid;
                              \SI{2}{\mM}  EDTA.\\
    
    TBS (Tris Buffered Saline)  & \SI{50}{\mM} Tris--HCl pH=\num{7.5};
                              \SI{100}{\mM}    NaCl.\\
    
    PBS-T (TBS-Tween)       & \pcent{99.95} PBS ($v/v$);
                              \pcent{0.05}  Tween 20 ($v/v$).\\
    
    TG (Tris Glycine)       & \SI{25}{\mM}  Tris;
                              \SI{192}{\mM} glycine.\\
    
    Transfer buffer         & \SI{1}{$\times$} TG;
                              \pcent{15} methanol ($v/v$).\\
    \bottomrule
  \end{longtable}
    

\chapter{List of plasmids}
  \label{app:plasmids}
  %% generate this automatically from SQlite db
  %% should we print the maps on genbank format, just the genbank features
  %% table, or a draw of the map, poiting to the database for the sequence
  %% details?

\chapter{List of primers}
  \label{app:primers}
  %% generate this automatically from SQlite db


  \backmatter

  \bibliography{%
    intro/references,%
    methods/references,%
    software/references,%
    histone-catalogue/references,%
    kill-frap/references,%
    discussion/references,%
    h2ax-review/H2AXreview-biblio%
  }

\end{document}
