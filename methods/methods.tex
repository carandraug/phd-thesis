\chapter{Materials and Methods}
\label{ch:methods}

  %% \epigraph{As palavras que nunca te direi.}{Translation of ``Message in a bottle''}
  %% \epigraph{The devil is in the details.}

  All chemicals used were purchased from Sigma unless otherwise stated. All
  solutions were prepared according to \Aref{app:solutions} with Milli-Q
  purified water, and autoclaved prior to use when appropriate.

  Restriction enzymes, T4 DNA ligase, other DNA modifying enzymes, and DNA
  ladders were obtained from New England BioLabs. Protein ladders from Invitrogen.

  For use in tissue culture, FCS and DMEM (without phenol red and L-glutamine)
  were obtained from Lonza; DMEM (supplemented with glucose, sodium pyruvate,
  L-glutamine and phenol red), Non-Essential Amino Acid (NEAA) solution, and
  PBS (without Ca$^{2+}$ and Mg$^{2+}$) from Sigma; Penicillin--Streptomycin
  solution and Trypsin-EDTA from Gibco.

  \section{DNA methods}
    \subsection{Bacterial cultures}
      \species{E.~coli} cultures were prepared with either LB broth or agar
      at \dc{37}. For antibiotic selection, ampicillin, kanamycin, and
      chloramphenicol, were used at concentrations of 100, 30,
      and \SI{34}{\mg\per\l} respectively.

    \subsection{Preparation of competent bacteria}
      Competent \species{E.~coli} cells were prepared from a culture of
      Invitrogen's One Shot TOP10 Chemically Competent \species{E.~coli}. LB
      cultures of \SI{1}{\l} were set at \dc{37} until an OD$_{\SI{600}{\nm}}$
      of \numrange{0.4}{0.5}. Further steps were carried at \dc{4} with
      previously chilled equipment and solutions.

      Cultures were centrifuged at \SI{6000}{\gn} for 10 minutes, the
      pellet resuspended in \SI{500}{\ml} of \SI{0.1}{\mM} CaCl$_2$, and
      incubated on ice for 30 minutes. The suspensions were centrifuged
      again at \SI{6000}{\gn} for 10 minutes, and the new pellet resuspended in
      \SI{100}{\ml} of CaCl$_2$ with \pcent{15} glycerol. Aliquots of
      competent cells were prepared and stored at \dc{-80}.

      Transformation efficiencies were measured after preparation of each
      batch and discarded if less than \SI{1d6}{\cfu\per\mg} of plasmid was
      obtained. Absence of antibiotic-resistant contaminations was assessed
      by streaking the cells on selective plates.

    \subsection{Transformation of competent cells}
      Competent cells were thawed on ice and split into aliquots of
      \SI{50}{\ul} to pre-chilled \SI{2}{\ml} tubes where \SI{1}{\ul} DNA
      was added. Cells were incubated on ice for 30 minutes, followed by a
      60 seconds heat-shock at \dc{42}, and 5 more minutes on ice.
      \SI{300}{\ul} of non-selective LB was added to each tube and the
      cultures incubated at \dc{37} with vigorous shaking for 45 minutes.
      Samples from the cultures were plated onto the appropriate
      antibiotic containing plates, and incubated overnight at \dc{37}.

      For concentrations of plasmid DNA higher than \SI{500}{\ng\per\ug}, only
      \SI{0.3}{\ul} of DNA was used, and both the initial and final incubation
      steps were shorten to 10 minutes.

    \subsection{Plasmid DNA preparation}
      Plasmid DNA was prepared with kits QIAprep Spin Miniprep,
      QIAGEN Plasmid \textit{Plus} Midi, QIAquick Gel Extraction, and QIAquick
      PCR purification from QIAGEN following the manufacturer's instructions.
      Once prepared, DNA was stored at \dc{-20}. DNA concentrations were measured
      with a spectrophotometer (NanoDrop 2000c spectrophotometer from
      Thermo Scientific).

    \subsection{Ethanol precipitation}
      \label{sec:ethanol-precipitation}
      The DNA solution was mixed with \num{2.5} volumes of \pcent{100} ethanol
      and \num{1/10} volumes of Sodium Acetate (\SI{3}{\Molar}, pH=\num{5.2}),
      and incubated at \dc{4} for 15 minutes. When DNA concentrations was below
      \SI{50}{\ng\per\ug}, incubation was performed overnight.
      Solution was centrifuged at
      \SI{18000}{\gn} for 30 minutes at \dc{4}, the supernatant discarded, and
      the pellet left to dry until all traces of solvent evaporated. DNA pellet
      was resuspended in the desired solvent: H$_2$O when used for transfection,
      EB~buffer from QIAGEN otherwise.
      low DNA plasmid concentrations)

    \subsection{Agarose gels electrophoresis}
      Agarose gels with concentrations ranging from \SIrange{0.6}{2.0}{\percent}
      were prepared with TAE buffer, and supplemented with ethidium bromide.
      DNA samples were loaded into the gel with DNA loading buffer and a
      choice of loading dyes between bromophenol blue, cresol red, orange G, or
      xylene cyanol, to avoid shadowing of the DNA bands. Electrophoresis was
      performed in electrophoresis chambers with \SI{1}{\X}~TAE buffer at
      \SIrange{80}{120}{\volt} until the required separation was achieved.
      Gels were visualized at an UV transilluminator ChemiImager 5500 from
      Alpha Innotech.

    \subsection{DNA sequencing and oligonucleotide preparation}
      DNA sequencing was performed by LGC Genomics after cloning for sequence
      confirmation and avoiding unexpected mutations.

      Oligonucleotides were ordered from Eurofins MWG operon in lyophilized
      format, dissolved in H$_2$O to a \SI{100}{\micro\Molar} concentration,
      and stored at \dc{-20}. A list of all designed oligonucleotides is
      produced at \Aref{app:primers}.

    \subsection{Polimerase Chain Reaction}
      Different types of PCR experiments were performed for different purposes
      with different DNA polymerases (\tref{tab:pcr-settings}). Taq with ThermoPol
      buffer was obtained from New England Biolabs. KOD Hot Start with Mg$^{2+}$
      free buffer was obtained from Novagen. PCRs were performed on a thermocycler
      Mastercycler epgradient from Eppendorf.

      Use of different DNA polymerases was based on their cost-benefit for
      each application. For example, screening clones requires a large number
      of reactions in simultaneous and the introduction of small mutations
      is of no consequence. For this two reasons, the much cheaper Taq DNA
      polymerase was used for screening despite its low fidelity and
      amplification rates. However, in PCR mutagenesis the whole plasmid
      is synthesized anew and we have no reasonable method to verify its whole
      sequence. As a result, KOD DNA polymerase was used for PCR mutagenesis.

      \begin{sidewaystable}
        \centering
        \captionIntro{PCR mixtures and conditions used.}
          {
            Since each reaction was unique, with different pair of primers
            and template DNA, the optimal salt concentrations, temperature
            of the annealing step, and time of extension step actually used were
            sometimes different. Listed values correspond to the standard
            usage and initial attempts.
          }
        \label{tab:pcr-settings}

        \newcolumntype{W}{r<{\si{\second}}}
        \newcolumntype{T}{l<{\si{\degreeCelsius}}}

        \begin{tabular}{l W@{ at }T W@{ at }T W@{ at }T W@{ at }T W@{ at }T}
          \toprule
          \null                        & \multicolumn{4}{c}{cloning} & \crows{colony} & \crows{mutagenesis} & \crows{screening} \\
                                                \cmidrule(r){2-5}
          \null                        & \crows{genomic} & \crows{plasmid} \\
          \midrule
          Template (\si{\ng})          & \crows{1000}      & \crows{50}        & \crows{n/a}       & \crows{750}       & \crows{50}        \\
          DMSO (\si{\ul})              & \crows{---}       & \crows{---}       & \crows{1}         & \crows{---}       & \crows{1}         \\
          Buffer (\si{\X})             & \crows{1}         & \crows{1}         & \crows{1}         & \crows{1}         & \crows{1}         \\
          MgSO$_4$ (\si{\nmol})        & \crows{1.5}       & \crows{1.5}       & \crows{---}       & \crows{1.5}       & \crows{---}       \\
          dNTPs (\si{\mM} each)        & \crows{0.2}       & \crows{0.2}       & \crows{0.2}       & \crows{0.2}       & \crows{0.2}       \\
          Primer (forward) (\si{\uM})  & \crows{2}         & \crows{2}         & \crows{2}         & \crows{0.4}       & \crows{2}         \\
          Primer (reverse) (\si{\uM})  & \crows{2}         & \crows{2}         & \crows{2}         & \crows{0.4}       & \crows{2}         \\
          DNA polymerase (\si{U})      & \crows{0.5 (KOD)} & \crows{0.5 (KOD)} & \crows{2.5 (Taq)} & \crows{0.5 (KOD)} & \crows{2.5 (Taq)} \\
          \addlinespace
          Total volume (\si{\ul})      & \crows{25}        & \crows{25}        & \crows{25}        & \crows{25}        & \crows{25}        \\
          \addlinespace
          \midrule
          \addlinespace
          Initialization                & 120 & 94    & 120 & 94    & 180 & 94    & 120 & 94    & 120 & 94 \\
          Denaturation                  &  30 & 94    &  30 & 94    &  15 & 94    &  30 & 94    &  15 & 94 \\
          Annealing                     &  20 & 58    &  20 & 58    &  15 & 58    &  20 & 58    &  15 & 62 \\
          Extension (per \si{\kilo\bp}) &  30 & 68    &  20 & 72    &  60 & 72    &  30 & 68    &  60 & 72 \\
          Final extension               & 300 & 62    & 300 & 72    & 300 & 72    & 300 & 68    & 300 & 72 \\
          Final hold       & \crows{\dc{4}}      & \crows{\dc{4}}      & \crows{\dc{4}}      & \crows{\dc{4}}      & \crows{\dc{4}} \\
          Number of cycles & \crows{\SI{30}{\X}} & \crows{\SI{30}{\X}} & \crows{\SI{30}{\X}} & \crows{\SI{15}{\X}} & \crows{\SI{20}{\X}} \\
          \bottomrule
        \end{tabular}
      \end{sidewaystable}

      \subsubsection{Colony PCR}
        When screening multiple clones after transformation and a set of
        appropriate primers was available, PCRs were set directly from
        the bacteria colonies by adding them directly to the reaction mixture
        (\tref{tab:pcr-settings}). Bacteria from individual colonies were
        used to simultaneously perform a PCR and start a small culture.
        Plasmid purification was performed on cultures whose sample was
        positive by PCR.

      \subsubsection{Gene cloning}
        PCRs were used to clone and subclone genes from genomic DNA and
        plasmids, into different vectors. The most commonly used vectors
        were pEF--BOS, pEGFP-C1, pEGFP-N1, pET3, and pET15 but a complete
        list is displayed on \Aref{app:plasmids}. Primers were usually
        extended to engineer restriction sites and create DNA linkers but
        were even longer on the 5' end to account for the minimum
        required \si{\bp} around the restriction sites \citep{neb_catalogue_2011}.

      \subsubsection{Mutagenesis}
        PCR mutagenesis
        \footnote{
          In PCR mutagenesis, since the newly synthesized strands will
          be linear and cannot be used as template, the amplification
          is linear rather than exponential. Because of this, the name
          PCR mutagenesis is misleading. There is \emph{no} actual chain reaction.
        }
        was used to insert or correct mutations in plasmids. When choice
        was possible, the selected codon used for mutation was the one
        with highest frequency in the expressing organism, following the
        codon usage database \citep{codon_usage}. After amplification,
        \SI{1.5}{\ul} of restriction enzyme DpnI
        \footnote{
          Because DpnI activity is blocked by DNA methylation, it will
          only digest the template DNA which was synthesized in bacteria,
          leaving the newly \textit{in vitro} synthesized DNA intact.
        }
        was added directly to the PCR mixture, and incubated overnight at
        \dc{37}. \SI{1}{\ul} was used for transformation and individual
        clones screened by sequencing.

      \subsubsection{Screening plasmid}
        PCRs were frequently used to screen plasmids for DNA sequences
        in the absence of opportune restriction sites or even plasmid maps.

  \section{Protein methods}
    \subsection{Phenol:chloroform extraction}
      \label{sec:phenol-extraction}
      To extract proteins, an equal volume of phenol:chloroform was
      added and the mixture centrifuged at \SI{6000}{\gn} for 15 minutes.
      The top aqueous phase (chloroform) was pipetted to a new tube and
      the process repeated a total of 3 times.

    \subsection{Western blotting}
      \subsubsection{Protein concentration determination}
        Concentration of protein was measured with Bradford reagent.
        \SI{2}{\ul} of the sample after sonication (\Sref{sec:cell-extract})
        was mixed with \SI{48}{\ul} of H$_2$O and \SI{50}{\ul} of NaOH and
        incubated at \dc{65} for 8 minutes before adding \SI{900}{\ul} of
        Bradford reagent from Pierce. The mixture was transferred to plastic
        cuvettes and the absorbance at \SI{595}{\nm} measured in a Shimadzu
        spectrophotometer. The values obtained were interpolated from a
        standard curve prepared using known concentrations of BSA.

      \subsubsection{SDS--PAGE}
        Resolving and stacking SDS--PAGE gels of \SIrange{15}{5}{\percent}
        respectively, both with a cross-linking ratio of \num{37.5}:1 as
        described in \citet{harlow_electrophoresis_1988}. The
        resolving gel was poured directly after addition
        of TEMED and it was covered with a layer of isopropanol during polymerisation
        to ensure a sharp interface between the resolving and stacking layers.
        Protein samples and markers were boiled at \dc{99} for 3 minutes and each
        was loaded twice, with volumes for \SI{3.3}{\ug} and \SI{16.5}{\ug} of
        protein. Gels ran at \SI{180}{\volt} for 1 hour in \SI{1}{\X} TG buffer.

      \subsubsection{Protein transfer}
        Protein transfer occurred through the wet transfer system. The
        SDS-PAGE gel was placed onto pre-cut nitrocellulose transfer membrane
        previously soaked in transfer buffer. It was then set between a pair
        of extra thick blotting paper and cushions before being placed inside
        a transfer apparatus. The transfer ran at \dc{4} for 60 minutes.

      \subsubsection{Probing of blot with antibody}
        Blocking of the membrane was performed with \SI{10}{\percent}
        non-fat dry milk in \SI{1}{\X} TBST
        at room temperature for 30 minutes. Blocking was followed by
        primary antibody incubation which occurred in \SI{5}{\percent} non-fat
        dry milk in \SI{1}{\X} TBST overnight at \dc{4}. Concentrations of
        antibody used were 1:500 and 1:20000 for anti-GFP (catalogue
        number 11~814~460~001 from Roche) and anti-H3 (code ab1791 from
        abcam). The membrane was then washed with \SI{1}{\X} TBST for
        15 minutes 3 times before the secondary antibody incubation which
        occurred in \SI{5}{\percent} non-fat dry milk in \SI{1}{\X} TBST for
        1 hour. The membrane was washed once more in the same
        conditions as before for the detection. All blocking, antibody
        incubation and washing steps occurred on a rocker.

        Detection was performed using the SuperSignal West Pico Chemiluminescent
        Substrate from Pierce, adding 1:1 of the solutions and allowing it to incubate
        with the membrane for 5 minutes. The membrane was exposed to x-ray films for
        10, 60, 5, 180 and 1800 seconds which were then developed.


  \section{Cell methods}
    \subsection{Cell culture}
      HeLa and HEp-2 cells were supplied by Agnieszka Kaczmarczyk from the
      National University of Ireland, Galway, Department of Biochemistry,
      and Volker D\"oring from the Leibniz Institute for Age Research -- Fritz Lipmann Institute.
      Primary horse fibrolasts were a gift from Prof.~Elena Giulotto from the
      University of Pavia, Department of Genetics and Microbiology.

      All cell lines were maintained at \dc{37} and \pcent{5} CO$_2$ in \SI{10}{\cm}
      diameter plates with \SI{10}{\ml} of their respective growth medium (see \Aref{app:solutions}).
      Dulbecco's Phosphate Buffered
      Saline (DPBS) and trypsin--EDTA solutions were used, respectively, to wash
      and split the cells 1:10 once they reached a confluence of \SIrange{80}{90}{\percent}.

    \subsection{Cell stock storage}
      For long-term storage of HeLa cell lines, they were grown until
      they reached a confluence of \SIrange{80}{90}{\percent} and then
      trypsinized as usual. A volume of Freezing Media was added, equal
      to the volume of trypsin--EDTA, and \SI{2}{\ml} aliquots of
      cells transferred to cryotubes. Tubes were immediately wrapped in
      cotton and placed at \dc{-80}.

    \subsection{Whole cell extract}
      \label{sec:cell-extract}
      To obtain whole cell extracts, HeLa cells were trypsinized as usual. Growth
      medium added and the suspension was centrifuged at \SI{900}{\gn}
      for 10 minutes. Further steps were carried at \dc{4} and with previously
      chilled reagents. The supernatant was discarded and the pellet
      resuspended in \SI{500}{\ul} of chilled PBS before being
      sonicated 3 times at \pcent{40} amplitude for 10 seconds. Avoiding the
      foam formed at the top, \SI{300}{\ul} of suspension were transferred
      from the bottom of the tube to a new eppendorf and mixed with an equal
      volume of Laemmli buffer before being stored at \dc{-80}.
      \SI{2}{\ul} from the suspension was also transferred to a new tube
      for determination of protein concentration.

    \subsection{Genomic DNA extraction}
      To extract genomic DNA of HeLa cells, they were trypsinized as usual and
      growth medium was added in the end before counting with an hemocytometer.
      Cells were centrifuged at \SI{1500}{\gn} for 10 minutes at \dc{4}, the
      supernatant discarded and the pellet resuspended in TE buffer to achieve
      a desired concentration of \SI{4e7}{cells\per\ml}. 9 volumes of
      Genomic lysis buffer was added and the mixture incubated at \dc{37} for
      90 minutes. Proteinase K was added to a final concentration of
      \SI{100}{\ug\per\ml} and the mixture incubated at 50 °C for 3 hours and
      swirled every 20 minutes. DNA was then extracted by
      phenol:chloroform (\Sref{sec:phenol-extraction}) and purified by ethanol
      precipitation (\Sref{sec:ethanol-precipitation}).

    \subsection{Viable cells count}
      \label{sec:methods:trypan-blue}
      Trypan blue was used to assess the number of viable cells.
      After trypsinization as usual, cells were diluted in growth medium,
      to a final concentration of \SI{2e5}{cells\per\ml}.
      To \SI{0.5}{\ml} of the cell suspension, was added \SI{0.1}{\ml}
      of \pcent{0.4} Trypan Blue Stain and the mixture left for 5 minutes
      at room temperature before counting
      the cells in an hemocytometer making a distinction between
      stained (non-viable) and non-stained (viable) cells.

    \subsection{Kill curve}
      \label{sec:methods:kill-curve}
      Cells were trypsinized as usual and plated at \pcent{25} confluence on
      a 24-well plate. After 24 hours, medium was replaced using different
      concentrations of the antibiotic per well, 3 replicas for each.
      After 3 days, medium was
      renovated with the same antibiotic concentrations. After 4 days, total of
      one week after addition of antibiotics, cells were trypsinized and a count
      of viable cells was performed with Trypan Blue (\Sref{sec:methods:trypan-blue}).

      The highest concentration tested with no viable cells in all replicas after
      7 days, was used for selection of transfected cells when establishing
      stable cell lines.

    \subsection{Transfection by lipofection}
      \label{methods:lipofection}
      Cells were transfected using Lipofectamine~2000, a cationic lipid
      reagent, from Invitrogen. Cells were trypsinized as usual on the
      day before transfection and replated on 6 well plates (surface area
      of \SI{9.5}{\square\cm\per well}) with \SI{2.5}{\ml} of growth
      medium so they would be \SI{90}{\percent} confluent on the following
      day. For each well, two tubes with \SI{250}{\ul} of transfection
      medium were prepared, one with \SI{7.5}{\ul} of Lipofectamine~2000
      and another with \SI{3750}{\ng} of DNA from a stock with concentration
      of \SI{500}{\ng\per\ul} and prepared by ethanol precipitation
      (\Sref{sec:ethanol-precipitation}).
      Both tubes were incubated at room temperature for 5 minutes,
      mixed together, and incubated again at room temperature for 20 minutes. Cells
      were washed with DPBS during this time and growth medium switched to \SI{2}{\ml}
      of transfection medium. The mixture was then added to the cells medium who
      were incubated at \dc{37} for 6 hours after which time it was switched back to
      \SI{0.5}{\ml} of growth medium.

    \subsection{Transfection by electroporation}
      Cells were transfected by electroporation using an Amaxa nucleofector
      device with Ingenio Electroporation solution and cuvettes from Mirus Bio.
      Cells were grown to confluence and trypsinized as usual. Growth medium was
      added to a total volume of \SI{8}{\ml}, and \SI{1}{\ml} aliquots (for
      approximately \SI{10e6}{cells}) were centrifuged at \SI{160}{\gn} for 5~minutes.
      The supernatant was discarded and a volume of \SI{2}{\ul} of plasmid at
      a concentration of \SI{500}{\ng\per\ul} was added to the top of the pellet.
      The cell pellet was ressuspended in \SI{100}{\ul} of Mirus Bio Ingenio
      Electroporation solution and transferred to \SI{0.2}{\cm} cuvettes for
      electroporation in an Amaxa nucleofector. According to the manufacturers
      instructions, the preset programs I-013 and O-17 were used to transfect
      HeLa and HEp-2 cells respectively.


    \subsection{Fixation and staining}
      For microscopy visualization, cells grown directly on top of HCl washed coverslips since
      HeLa cells have difficulty attaching to glass. At least 24 hours passed
      between the plating and fixation. Growth medium was removed and the cells
      washed with PBS once before incubation with \SI{4}{\percent} formaldehyde in PBS
      for 4 minutes. The solution was removed and the cells washed with H$_2$O 2 more
      times, after which coverslips were removed from the wells and left to air dry.
      For each coverslip, \SI{2}{\ul} of SlowFade Light Antifade kit from Molecular Probes
      was used for mounting the coverslip on a microscope slide. DAPI was added
      to the mounting media when needed. Coverslips were then sealed with a 1:1
      mixture of clear nail polish and acetone and stored on a dark box at \dc{4}.

    \subsection{Flow citometry}

      Cells were trypsinized as usual and after addition of medium, centrifuged
      at \SI{160}{\gn} for 5~minutes. The supernatant was discarded and the
      pellet washed with PBS. The sample was centrifuged one more time at
      \SI{160}{\gn} for 5~minutes. The supernatant was discarded and the
      pellet resuspended in FACS buffer.

      Both sample analysis and cell sorting were performed with live cells, the
      first with a BD FACSCanto~II, and the later with a BD FACSAria~II.

  \section{Software used}
    \label{sec:methods:software}
    The European Molecular Biology Open Software Suite (EMBOSS) version 6.4.0
    was used for analysis of codon usage, RNA folding, and reading of
    chromatogram files.

    Image analysis was performed using GNU Octave version \OctaveVersion{},
    and the Octave Forge image, optim, statistics, bioformats, and signal
    packages, versions \OctaveImageVersion{}, \OctaveOptimVersion{},
    \OctaveStatisticsVersion{}, \OctaveBioformatsVersion{}, and
    \OctaveSignalVersion{} respectively.

    ImageJ, as distributed by the Fiji project, was routinely used
    for microscope image visualization.
