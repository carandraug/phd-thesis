\chapter{Quantitative dynamics of histones in the human cell nucleus}
\label{ch:kill-frap}

  %% \epigraph{Agora desenmerda-te.}{Portuguese ``saying''}

  %% chapter concept: in this chapter goes the whole kill FRAP project. I started
  %% positive that this should work and so should the text. We start by studying
  %% the technique and list the assumptions it requires. We do one experiment and
  %% face some problems. Each problem has its own section that ends in a solution.
  %% When the solution was not done, we could offer one. The last one is the cause
  %% that this is not possible and has no solution. Conclusion lists all problems
  %% and their solutions again.

  %% AF: Think this needs to be an abstract, summarising capter in 250 words
  %% Why would you have anything else than an abstract?

  \chapterprecis{
    To further the understanding of nucleosome structure and positioning
    in DNA, we aim to measure kinetic differences between wild type and
    mutant histones in live cells. Fluorescent Recovery After Photobleaching
    is a technique successfully used before to show the different kinetics and
    populations between different histone types. Using the same approach with
    our most disrupting histone mutant, we test the limits of FRAP when
    faced with extremely slow exchange ratios.
  }

  \section{Introduction}

    The chromatin packaging of eukaryote genomes
    compacts very large lengths of DNA into the
    microscopic cell nucleus, facilitates chromosomal movements
    during cell division, and provides a substrate for molecular
    mechanisms acting on the genome.

    The building block of eukaryotic chromatin is the
    nucleosome structure, comprising \SI{147}{\bp} of DNA wrapped around an
    octamer of two copies each of core histones
    H2A, H2B, H3, and H4 \citep{Luger1997structure}.
    In this structure, two H2A/H2B dimers flank a central H3/H4 tetramer.
    Nucleosome core particles are arranged
    in a linear chain separated by DNA linkers, and can be further
    compacted into higher order chromatin structures.

    Chromatin functionality at the molecular and cellular levels
    requires the capability for dynamic rearrangement.  Chromatin
    structure can be modulated through nucleosomes by changing the
    arrangement of histones and DNA in a process known as remodelling
    \citep{flaus2011mechanisms}, or by altering histone chemical
    composition through post-translational modification
    \citep{bannister2011ptm-review}, or exchange of histone variants
    \citep{TalbertHenikoff2010}.

    A large amount of information has been accumulated about the static
    structure of the nucleosome at atomic resolution
    \citep{mcginty2014histone}, and about the arrangement of polymeric
    chromatin \citep{kuznetsova2016chromatin}.  However, the
    mechanisms for dynamic rearrangement of chromatin are much less
    well understood or integrated between the molecular and polymer
    levels \citep{andrews2011nucleosome}.

  \subsection{Nucleosome dynamics and histone SIN mutants}

    The archetype of ATP-dependent nucleosome remodelling enzymes is
    the SWI/SNF complex, which was identified in screens for mating
    type SWItching \citep{SWI-mutants} and Sucrose Non Fermentation
    \citep{SNF-mutants-original-discovery, SNF-mutants2} in
    \species{Saccharomyces cerevisiae}.

    Mutations were subsequently identified that compensate for the
    loss of the SWI/SNF complex and these are collectively known as SIN
    mutations because they provide SWI/SNF INdependence
    \citep{kruger1995amino}.  A subset of SIN mutants are single
    amino-acid changes in core histones H3 and H4, providing a direct
    link between SWI/SNF and chromatin.  This also suggests the
    mutated residues in the histone proteins influence the same
    pathways for nucleosome dynamics that are leveraged
    by the remodelling enzyme, and that
    these residues are significant for nucleosome stability.

    The prediction that the stability of SIN mutant containing
    nucleosomes is affected in chromatin has been tested \textit{in vitro},
    where it was observed that SIN mutant nucleosomes display higher
    thermally driven nucleosome sliding mobility \citep{flaus2004sin}
    and that the mutated residues affect histone-DNA contacts in
    crystal structures \citep{muthurajan2004crystal}.

    However, the effect of histone protein SIN mutants
    in the more complex \textit{in vivo} chromatin environment
    of mammalian cells has not been demonstrated.
    Validating the functional significance of these residues
    is important for understanding nucleosome dynamics and
    explaining the high degree of conservation of
    histone protein sequences in eukaryotes.

  \subsection{Fluorescence Recovery After Photobleaching}

    Fluorescence Recovery After Photobleaching (FRAP) is an optical technique
    that can be used to report the dynamics of fluorescently
    tagged molecules within live cells.
    Tagged molecules inside a small region are irreversibly photobleached by
    a focused high power laser beam and the recovery rate of fluorescence
    in the bleached area is measured. The recovery rate is interpreted
    as unbleached fluorescing molecules from outside the region
    at the time of photobleaching diffusing into the bleached area.
    It is assumed that this fluorescence recovery
    reflects natural protein movement.

    A simple chemical equilibrium underlies the model for FRAP for a
    molecule with a single binding reaction:

    \begin{displaymath}
      F + S \overset{k_{on}}{\underset{k_{off}}{\rightleftharpoons}} FS
    \end{displaymath}

    where $F$ represents freely diffusing proteins,
    $S$ represents immobile vacant binding sites,
    and $FS$ is the complex between the two when the
    proteins are bound to the sites.
    The value of \kon{} and \koff{},
    are estimated from the rate at which photobleached $F$
    is replaced in the $FS$ complex.
    However, complexity is added to this simple model if
    diffusion and space are considered.

    Ongoing development of FRAP has led to increasingly complex models
    that are both more precise and more accurate than
    inverse of an exponential decay, namely $1 - Ae^{-k_{off}t}$
    \citep{mcnally-frap-2010}.
    Despite the sophistication of current models, FRAP requires assumptions
    that are difficult to maintain over long experimental observation times.
    Firstly, equilibrium must be maintained throughout the entire experiment
    so that both \kon{} and \koff{} remain constant.
    This also requires that concentrations of both $F$ and $S$ remain constant.
    Secondly, distribution of the fluorescently tagged molecules
    must mimic the endogenous protein.
    And finally, the binding sites must be part of a large,
    relatively immobile complex
    on the time and length scale of the recovery.
    In addition, different FRAP models will have further constraints
    based on the assumptions involved in its design \citep{mcnally-frap-2010}.

  \subsection{FRAP measurements of histones}

    FRAP has been extensively used to obtain qualitative and
    quantitative insight into the kinetic properties of chromatin bound
    proteins \citep{phair2000high, essers2005nuclear, agresti2005gr}.
    These rely on the established assumption that chromatin is
    relatively immobile in the interphase nucleus
    \citep{abney1997chromatin} since most proteins show recovery
    on the scale of seconds to minutes.
    H2B--GFP \citep{KevinH2BGFP} has become
    the standard reference for the immobile fraction in these
    FRAP experiments \citep{dey2000bromodomain, kuipers2011highly,
    jullien2016chromatibody}.

    However, the dynamics of the core histones themselves
    was measured by FRAP in a seminal
    and widely cited study by \citet{KimuraCook}. Multiple H2B
    populations were delineated with distinctive exchange rates.  Some
    \pcent{3} of H2B had a rapid recovery within minutes, \pcent{40}
    had slow recovery with \halflife[] of \SI{130}{\minute},
    and over \pcent{50} of H2B molecules had a
    very slow recovery with \halflife[] of
    over 8 hours that was considered to be effectively immobile.

    In contrast to H2B which is a histone dimer component, the tetramer
    histones H3 and H4 were found to have even slower mobility.  There
    were no rapid populations, with only slow and very slow populations
    of \SIrange{16}{22}{\percent} and \SIrange{62}{68}{\percent} being
    identified respectively.

    In combination with additional heterokaryon data, Kimura and Cook
    interpreted the rapid, slow, and very slow by exchanging H2B
    populations as correlating with transcription units, euchromatin,
    and heterochromatin respectively.  They assigned over \pcent{80}
    of H3 and H4 as immobile, whereas the remaining
    \SI{\approx 20}{\percent} was suggested to be mobilised by
    remodelling.  This latter small but significant
    slowly exchanging fraction of histones
    provides an opportunity to observe the dynamics of
    tetramer histones such as SIN point mutants in live mammalian cells.

    %% Their mobile fraction was odd.  The claim is that it represents
    %% the freely moving histones, but the text says that it was the
    %% recovery before the first post-bleach image could be acquired.
    %% This sounds like they assumed complete photobleaching.
    %% However, it can be that this is actually the unbleached
    %% fraction or some other background.

  \subsection{Aims and objectives}

    In order to test the implications for nucleosome structure and function
    of H4 mutant R45H that exhibited the highest increase
    in nucleosome mobility \textit{in vitro} amongst
    the histone SIN mutants \citep{flaus2004sin},
    we set out to determine its exchange characteristics by FRAP.

    In attempting to achieve quantitative measurements
    we encountered multiple technical challenges associated with
    measuring subtle kinetic alterations in nucleosome dynamics
    over long time periods in live cells.
    This required us to define the limitations of FRAP for observing molecules
    such as core histones with extremely slow exchange rates.

  \section{Materials and Methods}

  Stable HeLa cell lines were created for H2B--EGFP, H2B--EGFP D25G V118I,
  H3-EYFP, H3--EYFP T45A, H3--EYFP T45E, H4--EYFP, and H4--EYFP R45H
  \Srefp{sec:plasmid-construction}.
  Cells were transformed by lipofection \Srefp{methods:lipofection},
  split to a low confluence, and treated
  with \SI{3}{\ug\per\ml} \mbox{Blasticidin-S} for one week.
  Colonies were screened by fluorescence microscopy
  and positive individual colonies were selected for growth
  and by Fluorescence Activated Cell Sorting (FACS) to generate
  homogeneous highly fluorescing cell lines.
  Transiently expressing cell lines, both HeLa and horse fibroblasts,
  were transformed by lipofection in
  the same manner, and imaging performed 48~hours later.
  Primary horse fibrolasts were a gift from Prof.\@ Elena Giulotto
  (University of Pavia). HeLa cells were ATCC line CCL-2.

  Confocal microscopy was performed on a Zeiss LSM510 Meta microscope
  using glass bottom LabTek~II chambers.  Wide-field fluorescence
  microscopy was performed with an Applied Precision DeltaVision Core
  system using \SI{35}{\mm} glass bottom MatTek dishes.  In both
  cases, imaging was performed within an acrylic environmental chamber
  at a temperature of \dc{37} and \pcent{5} CO$_2$.
  Images were acquired and deconvolved on the DeltaVision system.

  %% See the CropReg.m script for more details.  For example, only the
  %% region surrounding the original position was used for performance
  %% and robustness.

  Cell movement between time frames was calculated by consecutive
  template-based registration using normalised cross-correlation.
  The CropReg script developed for this purpose is available
  as free open source software in this manuscript repository.
  Nuclei of interest were identified
  on the first frame and used as templates on subsequent images.
  To correct for rotational movement
  around the $z$ dimension of the optical axis system,
  registered frames were aligned by rigid body geometric transformation
  using the ImageJ \citep{imagej1} plugin StackReg \citep{stackreg}.

  Automatic extraction and processing of FRAP recovery curves was
  performed with the GNU Octave programming language \citep{octave}
  and the Octave Forge Image package.  Source code written in Matlab
  for a previously reported circle FRAP model \citep{mcnally-frap-code}
  was kindly gifted by the original authors and ported to GNU Octave.
  We developed \command{frapinator}, a new program written in GNU Octave
  to automate analysis with multiple command line options
  and released it in the FRAP Octave package as free open source software.
  \command{frapinator} includes all individual functions for image
  pre-processing and FRAP fitting \Srefp{sec:software:octave-frap}.

  \section{Results}

  \subsection{FRAP analysis in Octave}

  \subsection{Tracking of cell nuclei}

    %% We could show this but it would only be to "encher chouricos"
%    \begin{figure}
%      \centering
%      \missingfigure{Our first FRAP experiment}
%      \captionIntro{Long time series of HeLa cells expressing H2B--EGFP.}
%                   {Cells were transfected with pBOS--H2B-EGFP and imaged for 8
%                    hours, with intervals of 20 minutes.}
%      \label{fig:kill-frap:cell-movement}
%    \end{figure}

    Histone proteins exhibit extremely slow kinetics of exchange
    requiring FRAP to be performed over several hours during which cell movement is
    a particular issue. % \frefp{fig:kill-frap:cell-movement}.

    %% This needs some sentences describing the nature of effects of the movement,
    %% and a semi-emirical statement of the scale of the movement
    %% since this problem is the nub of the entire chapter.
    %% Need to justify the need for going to the trouble
    %% and define threshold for success (that couldn't be achieved!)

    The central requirement to accurately identify and quantitate the signal
    in the photobleached spot led us to pursue both cell biological approaches to
    minimise cell movement and and computational approaches to track imaged regions.

    \subsubsection{Contact inhibition at high cell density}

      Mammalian cells display contact inhibition,
      a cellular growth mechanism by which cells enter senescence and reduce motion
      when surrounded by other cells with no free space for movement.
      Although transformed cell lines lose this property,
      reduced space does place a restriction in the movement.
      %% Can you provide a semi-quantitative estimate of the cell density you mean?
      We attempted to use this effect to limit movement of cells
      by performing FRAP measurements with cells at higher confluence levels.
      We observed some decrease of movement for HeLa cells but not a complete immobilization.
      %% Can you provide a semi-quantitative estimate of the degree of reduction in movement?

      \begin{figure}
        %% We are only showing one cell rather than the whole field of
        %% of view because otherwise it's hard to notice the movement of
        %% individual cells. If we do display everything, we cell many
        %% nuclei that seem like their movement is smaller. If we do
        %% show it, we comment that we are unsure whether the movement
        %% is cellular or only nuclear.
        \centering
        \includegraphics[width=\textwidth]{results/confluent-hela.png}
        \captionIntro{Movement of confluent HeLa cells during FRAP experiment}
          {
            Cells reached confluence before the start of the
            experiment in an attempt to reduce motion. Instead, this caused
            cell nuclei to undergo heavy reshape as the cell apparently
            squeezes in between its neighbours. Half-nuclear FRAP performed in
            a confocal microscope over an interval of 8~hours. Top left panel
            is the pre-bleach image, while the others have a time-interval of
            21~minutes. Cells are a stable line derived from HeLa, expressing
            the H4~R45H mutant tagged with YFP.
          }
        \label{fig:kill-frap:confluent-hela}
      \end{figure}

      Tracking of the ROIs over time was still required \frefp{fig:kill-frap:confluent-hela}.
      %% Isn't this out of order since CropReg section is below.
      %% Even if you did it in a different order, why not put the horse cells first.

    \subsubsection{Primary cell lines}

      Since HeLa cells have lost the ability to activate contact inhibition,
      we obtained an immortalised primary horse fibroblast cell line
      that displayed contact inhibition. Using this cell line we could
      maintain a layer of healthy cells covering a Petri dish for several days???
      after reaching confluence (data not shown).
      %% Cell growth was halted instead of becoming over-confluent ???what do you mean???
      %% Although obvious, it would be good to have some sort of image to justify the statement?

      However, primary cell line transfections have typically much lower efficiency rate and
      confluent cells have lower expression with each cell division.
      To balance transfection efficiency and expression we transfected cells at
      \SI{70}{\percent} confluence and imaged them after 3 days.

      Even after reaching confluence, we observed movement of transfected horse cells \frefp{fig:kill-frap:confluent-horse}.
      %% Is there some sort of relative statement about the degree of reduction or residual movement?
      In addition, the observed movement was dramatically different in primary horse than in HeLa cells.
      Transfected horse cells displayed a helical motion about the vector of their movement
      whereas rotation of HeLa nuclei was mostly restricted to the $z$ axis.
      %% You need to define the coordinate system to talk about z axis!
      %% Do you mean vertical relative to the cell dish layer?

      \begin{figure}
        \centering
        \includegraphics[width=\textwidth]{results/confluent-horse.png}
        \captionIntro{Movement of confluent primary cells during FRAP experiment}
          {
            Primary horse fibrolasts display contact inhibition and halt growth
            once they reach confluence. However, this does not stop cell
            motion which can still be seen moving. In addition, when compared
            to the cancer cell line HeLa \frefp{fig:kill-frap:confluent-hela},
            the horse fibroblasts frequently rotated around the $x$ and $y$
            axis. Circle FRAP was performed in a widefield microscope.
            Top left panel is the pre-bleach image, while the others have a
            time-interval of 15~minutes. Cells were transiently transfected
            and are expressing H2B type1-J tagged with EGFP.
          }
        \label{fig:kill-frap:confluent-horse}
      \end{figure}

    \subsubsection{Tracking of cell movement}

      As an alternative strategy, we implemented cell tracking
      in order to transform images into a common frame using
      consecutive image cropping and image registration.
      This approach was implemented as a program named CropReg.

      Nuclei of interest were tracked by template-based matching using normalized cross-correlation.
      Briefly, the nucleus to track was identified on the first frame and
      used as template against the image on the subsequent frame.
      For increased performance and robustness only the region surrounding the original position is used.
      %% Should you state approximately how large this region was?
      Sequentially applying this method created a stack of smaller images centred on the nuclei of interest.

      Achieving this functionality required implementing a ``coeff'' option
      for scaling in the \texttt{xcorr2} function in GNU Octave.
      This was contributed and released in version 1.2.0 of the Octave Forge signal package.
      %% TODO since there's more than one way to actually do the normalization,
      %% it might be a good idea to write down the actual math formula
      To correct for rotational movement around the $z$ axis,
      frames were aligned using rigid body geometric transformation using ImageJ plugin StackReg \citep{stackreg}.

      \begin{figure}
        \centering
        \includegraphics[width=\textwidth]{results/cropreg.png}
        %% imaging was done every 10 minutes, but we are skipping
        %% every other panel
        \captionIntro{Automatic tracking and alignment of moving cells}
          {
            Using CropReg, we successfully tracked individual cells during
            a time-series microscope experiment. The top left corner of each
            panel displays the tracked and aligned cell. Imaging was performed
            in a widefield microscope. Time interval between panels 20~minutes.
            Cells are a stable line derived from HeLa, expressing H3 tagged
            with YFP.
          }
        \label{fig:kill-frap:cropreg}
      \end{figure}

      Using this image processing approach we were able to track individual cell nuclei
      throughout an entire sequence of FRAP experiments \frefp{fig:kill-frap:cropreg} provided
      that nuclei did not overlap.
      Although only a small minority of image sequences satisfied this requirement,
      it was possible to collect sufficient observations for FRAP calcultions.

  \subsection{Chromatin movement}

    While performing the FRAP experiments, we observed some movement
    within the cell nuclei. These could not be accounted for simple rotational
    movement around the $x$ or $y$ axis, and resembled more the movement
    of individual bodies within the nuclei.

    \subsubsection{Selection of \G1{} cells}
      %% There's no chemical equilibrium in S phase

      A possible cause of this chromatin movement comes from changes in
      the cell cycle phase. During the S~phase, the DNA is replicated,
      doubling the content of the chromatin.
      More importantly, this breaks
      a core assumption of FRAP, that the system remains in equilibrium
      during the entire experiment. This does not hold if the DNA, the
      binding sites for our model, duplicate in number.

      If the FRAP experiments can't be performed during S~phase and
      mitosis, we are limited to \G1{} and \G2{}. Considering
      the length of the HeLa cell cycle and the requirements to image
      for a time period of 8~hours, we are further limited to \G1{}.
      In addition, the FRAP experiment must be performed early in
      \G1{}~phase to avoid crossing over to the S~phase.

      %% The only reason this was required was because the LSM 510
      %% that we were using could not make Z stack and time lapse
      %% at the same time.
%      \begin{figure}
%        \centering
%        \missingfigure{Hela cells splitting}
%        \captionIntro{Picking cells at early G$_1$.}
%                     {We imaged cells that were entering mitosis and picked their
%                      daughter cells for the FRAP experiments. Because HeLa cells lift
%                      away from the dish during mitosis, opening the
%                      pinhole and set the Z-center in between the cell dividing plane
%                      and dish bottom was necessary. Ends up nothing being properly in focus but we
%                      can track things fine. Of course, some cells still floated away.}
%        %% TODO explicit parameters
%        \label{fig:kill-frap:picking-early-g1}
%      \end{figure}

      To do this, cells in mitosis were selected and tracked during 4~hours.
      After this time period, we used the daughter cells which we could be
      confident of being in early \G1{}.
      %% we also waited some 2 hours after mitosis since that's when cells
      %% unpack their chromosomes.

      During mitosis, HeLa cells form a sphere slightly above the plane of
      other cells, and keep a weak connection to the growth surface.
      Because of this, they easily detach, which is the basis for the
      mitotic shake-off method, and float away from the field of vision
      which requires a larger number
      of initial selected cells. In addition, to minimize any effect that
      may arise from imaging, it was done at minimal laser power and every
      30~minutes, just enough to allow manual tracking.
      Finally, since our system did not permit simultaneous Z-stack and time
      lapse imaging, and cells in mitosis are in a separate focal plane,
      imaging was performed with the pinhole sized to the max and focused
      in between the two planes. While this
      created very blurred images, it allowed to visualize all cells during
      the entire procedure.

      However, even after selecting cells in this cell cycle, movement within
      the bleach spot could still be observed.

    \subsubsection{Inverse FRAP}

      Due to the non-homogeneous nature of the chromatin, it was difficult
      to assess the total extent of the observed movement. To
      better visualize this, we performed inverse FRAP which allows us
      to track the movement of the bleach spot only.

      For this purpose, we replaced the EGFP tag in our H2B plasmid
      with photoactivatable GFP (PAGFP), a GFP derivative that requires
      activation by a specific wavelength to become fluorescent. This
      allows us to activate a specific spot of the nucleus and visualize
      its movement.

      Since PAGFP cannot be easily detected before photoactivation, cells
      were co-transfected with mCherry--\textalpha--tubulin which localises
      exclusively to the cytoplasm, giving an outline of the nuclear region
      \frefp{fig:kill-frap:ifrap}.

      \begin{figure}
        \centering
        \subbottom[pre-activation]{
          \includegraphics[width=0.45\textwidth]
          {results/ifrap-pre.png}
          \label{fig:kill-frap:ifrap-pre}
        }
        \hfill
        \subbottom[post-activation]{
          \includegraphics[width=0.45\textwidth]
          {results/ifrap-post.png}
          \label{fig:kill-frap:ifrap-post}
        }
        \subbottom[activated spot over time]{
          \includegraphics[width=\textwidth]
          {results/ifrap.png}
          \label{fig:kill-frap:ifrap-timeframe}
        }
        \captionIntro{Inverse FRAP experiment showing chromatin movement}
          {
            HeLa cells co-transfected with mCherry--\textalpha--tubulin and
            H2B type1-J tagged with PAGFP.
            \subcaptionref{fig:kill-frap:ifrap-pre} The cell nucleus, target
            for photoactivation, can be easily identified as the ``empty''
            region via the mCherry channel on which would otherwise be an
            invisible feature on the GFP channel;
            \subcaptionref{fig:kill-frap:ifrap-post} spot after activation;
            \subcaptionref{fig:kill-frap:ifrap-timeframe} detail of the
            activated spot every 20~minutes. Rather than a gradual loss of
            fluorescence that maintains the circular shape, the activated spot
            kind of unfolds itself spreading the region of interest.
          }
        \label{fig:kill-frap:ifrap}
      \end{figure}

      Using this FRAP variant, the movement of chromatin was more noticeable.
      Rather than an homogeneous loss of fluorescence, the activated
      spot uncurled itself overtime with individual branches of
      localized PAGFP appearing in the nuclei \frefp{fig:kill-frap:ifrap}.

  \section{Discussion}

    We wished to quantitatively determine the
    effect on human chromatin dynamics
    of SIN mutations in core histones H3 and H4 known
    to be destabilising \textit{in vitro} and to affect
    cell growth in \species{S. cerevisiae}.
    We set out to use a previously reported circle FRAP model
    which accounts for multiple factors in a typical
    FRAP modelling \citep{mcnally-frap-code}.

    However, FRAP recovery is incomplete even
    after 8~hours for core histones \citep{KimuraCook}.
    This led us to address a series of technical challenges in
    collecting valid quantitative recovery data over extended time periods.

% Cell movement

    The first problem faced was cell motility, which
    is an expected property of actively dividing cells.
    We attempted to reduce motility by taking advantage of
    the fact that many primary cells display contact inhibition of
    locomotion and proliferation when they reach high densities.
    This contact inhibition is a natural mechanism
    that controls cellular growth in
    multicellular organisms, and results in a stop in proliferation
    with the formation of a monolayer of healthy cells in tissue culture.

    However, the approach has disadvantages including
    increased cell handling and reduced transfection efficiency.
    The potential inability to compare results with published data for
    immortalised cell lines such as HeLa is also undesirable.

    Despite achieving a monolayer of healthy cells that
    could be maintained stably over 2 weeks,
    individual transfected primary horse fibroblasts still showed motility
    despite exhibiting overall characteristics of contact inhibition.
    Furthermore, nuclei in these cells displayed a helical motion
    on the direction of cell movement \frefp{fig:kill-frap:confluent-horse}.

    The possibility of chemically inhibiting
    cells to reduce motion was considered
    since previous FRAP experiments with core
    histones were performed using multiple inhibitors
    of protein synthesis \citep{KimuraCook}. However, these studies revealed
    inhibitor-dependent variations in kinetics and the authors qualified
    their conclusions about the absolute accuracy
    of the histone exchange parameters measured.

    To better address the problem of cell motility we
    instead developed a computational approach
    by writing the program CropReg for cell tracking by normalised
    cross-correlation template matching.
    Using automated analysis enabled us to process
    the large numbers of cell images
    required to provide statistically valid
    quantitative measurements of core histone exchange.

% Compositional changes

    The second challenge to measuring core histone exchange by FRAP is that
    a chemical equilibrium is required between freely diffusing proteins and
    formation of a complex. Although absolute equilibrium is unlikely
    in the dynamic cell environment undergoing
    complex transcriptional and translation responses anyway,
    DNA replication involving polymerase passage
    and repackaging of the duplicated genome
    in S~phase will certainly unbalance any equilibrium.

    Chromosome compaction in mitosis also generates a chromatin environment
    that is distinct from interphase.
    This limits FRAP experiment to either \G1{} or \G2{} phases.
    The HeLa cell cycle has a typical \G1{} phase of 11.7~hours
    and a \G2{} phase of 3~hours \citep{HeLaCellCycle}
    so the extended time periods needed for FRAP of core histones requires
    starting FRAP early in \G1{} \frefp{fig:kill-frap:cell-cycle}.

    Post-mitotic chromosomes take approximately 2~hours
    to migrate within the nucleus
    and rebuild the interphase nuclear architecture during early \G1{}
    \citep{visualizationG1chromosomes,earlyg1position,RelativeChromosomePosition}.
    This defines the window for extended FRAP experiments
    from approximately 3 to 11 hours after mitosis
    in HeLa cells, although cells lines with even longer
    \G1{} phase could also be used \citep{PancreaticCells}.

    We wished to avoid the use of drugs for cell
    cycle arrest since this has been
    shown to influence FRAP results \citep{KimuraCook}.
    We also discounted serum starvation to move cells into the
    quiescent \G0{} phase since this could affect
    the relevance of measuring core histone
    exchange \citep{SerumStarvation}.

      \begin{figure}
        \centering
        %% based on original code from Robert Vollmert
        %% http://www.texample.net/tikz/examples/pie-chart/
        \newcommand{\slice}[4]{
          \pgfmathparse{0.5*#1+0.5*#2}
          \let\midangle\pgfmathresult

          % slice
          \draw[thick,fill=black!10] (0,0) -- (#1:1) arc (#1:#2:1) -- cycle;

          % outer label
          \node[label=\midangle:#4] at (\midangle:1) {};

          % inner label
          \pgfmathparse{min((#2-#1-10)/110*(-0.3),0)}
          \let\temp\pgfmathresult
          \pgfmathparse{max(\temp,-0.5) + 0.8}
          \let\innerpos\pgfmathresult
          \node at (\midangle:\innerpos) {#3};
        }
        \begin{tikzpicture}[scale=3]
          \newcounter{a}
          \newcounter{b}
          %% Total cell cycle is 24.5 hours, G1 is 11.7h, S is 8.8h,
          %% G2 is 3h, M is 1h. The problem is that the counters can't handle
          %% decimal places so we have a variable with the actual time for
          %% the text, and another one times 10 to calculate the angle.
          \foreach \p/\t/\l in {117/11.7/\G1, 9/0.9/M,
                                31/3.1/\G2, 88/8.8/S}
            {
              \setcounter{a}{\value{b}}
              \addtocounter{b}{\p}
              \slice{36*\thea/24.5} % we multiply by 36 instead of 360 because
                    {36*\theb/24.5} % the time is already times 10
                    {\l}{\t{} hours}
            }
        \end{tikzpicture}
        \captionIntro{HeLa cell cycle phases and timing}
          {
            Under optimal growth conditions the HeLa cell has a median
            doubling time of 24~hours, with \G1~and
            S~phases of 11.7 and 8.8 hours respectively \citep{HeLaCellCycle}.
          }
        \label{fig:kill-frap:cell-cycle}
      \end{figure}

    Instead, we developed a procedure to track
    progression of cells manually during mitosis
    where visual identification of the cell cycle is possible.
    This allowed us to minimise the variations of normal cell growth
    and to identify individual cells exactly 3 hours after start of \G1{}.
    This has the added advantage of allowing time for maturation of GFP
    expressed during the establishment of interphase.
    The time interval between images during manual selection was increased and
    both resolution and laser intensity were reduced
    to minimise phototoxicity or bleaching.
    This resulted in a set of selected early \G1{}
    cells suitable for FRAP experiments.

    The fluorescently tagged histone proteins are constitutively expressed
    under the control of an EF-1\textalpha{} promoter,
    so they lack the 3' regulatory features of native histone genes.
    This regulation does not follow the normal
    expression program of a histone gene
    and could affect the distribution of the histone in chromatin.
    Constant expression of tagged histones by a strong constitutive promoter
    will enrich them in the \G1{} and early S~phase pools
    making subsequent incorporation in euchromatin more likely,
    relative to mid-late S~phase where heterochromatic sequences
    are replicated and packaged \citep{DNA-replication-timing}.

    A more realistic tagged histone expression profile could be achieved using
    flanking regulatory regions from native histone genes,
    as demonstrated for H3 and CENP--A \citep{pMH3-plasmid,Kevin-pCA-TAG}.
    Another potential solution is to insert GFP
    in-frame into the native gene locus by genome engineering,
    although the redundancy between the multiple canonical histone genes means
    that identifying the most appropriate isoform to target
    could introduce complexities.

    Protein synthesis inhibitors were used by \citet{KimuraCook}
    to address this issue,
    but this has the disadvantage of potentially affecting
    many other processes as discussed above.

      %% TODO: would be cool to create this figure
%      \begin{figure}
%        \centering
%        \missingfigure{a schematic of cell cycle, soluble pool}
%        \captionIntro{Distribution of tagged and endogenous histones during cell cycle}
%                     {
%                       This would be at least 3 different subplots. The first
%                       and the second are like the ones in Fig 7A of Kimura and
%                       Cook paper. The third one would show the ratio of each
%                       histone over time, i.e., 100\% tagged during all cell
%                       cell cycle and some endogenous during S phase. In this
%                       plots, also note where euchromatin and heterochromatin
%                       are replicated.
%                     }.
%        \label{fig:kill-frap:messy-histone-expression}
%      \end{figure}

% Movement of the reference

    The final challenge to measuring core histone
    exchange by FRAP that we identified
    was non-homogenous regional movement of chromatin itself.

    One possible cause for this movement is chromatin repackaging
    during DNA replication which we addressed by selecting cells at
    early \G1{} phase.
    One other cause is chromatin remodelling as part of a DNA damage
    response caused by the FRAP photobleaching event itself.
    The phototoxicity effects of a FRAP experiment are often dismissed
    on the basis that the photobleaching event of a typical FRAP
    experiment does not affect cell viability
    \citep{kruhlak2000reduced, KimuraCook, carrero2003using} but the
    DNA damage that such experiment may introduce, and the effect that
    the repair response of such damage may have on chromatin
    reconfiguration, has not yet been addressed.
    Still, the use of laser beams of different wavelength and longer
    exposure times are often used to introduce both single and double
    DNA strand breaks on which FRAP experiments are then performed
    \citep{stixova2014advanced, mari2006dynamic, kim2002specific}.

    Independently of the cause for the chromatin movement,
    it undermines the assumption of FRAP analysis that binding sites
    remain immobile throughout the FRAP experiment.
    This assumption is required to interpret recovery
    as the rate of movement of freely diffusing unbleached molecules into the
    bleached area which allows the kinetic
    rates \kon{} and \koff{} to be estimated.
    If chromatin binding sites also move then the recovery curve becomes a
    much more complex function of both binding site movement and free diffusion.

    Chromatin movement is recognisable
    by changes in the intra-nuclear features of the fluorescent chromatin
    and by changes in the circular bleach spot.
    Although some of these effects are subtle when observed by photobleaching,
    the photoactivation of an equal circular spot demonstrates
    clear non-homogenous reshaping of chromatin.
    Equivalent chromatin movement has also been reported
    for H4--PAGFP in strip photoactivation \cite{H4PAGFP-chromatin-movement}.

    The movements we observed were in the
    range of \SI{4}{\um}, which is double the size of the bleach spot,
    and exhibited complicated shapes reminiscent of channelling.
    This is consistent with chromosome distribution in nuclei that is
    territorial on the scale of \SI{5}{\um} \citep{sun2000size}
    separated by interchromosomal channels of
    \SIrange{10}{100}{\nm} \citep{gorisch2005histone}.

    The clarity of H2B--GFP imaging by photoactivation
    suggests the opportunity to analyse the
    paths taken by diffusing core histones.
    For example, simultaneous use of combined
    photoactivation and photobleaching
    of complementary dimer and tetramer histones could
    enable relative diffusion rates and paths to be determined.
    Alternatively, an enzymatic mechanism to incorporate a
    complementary photo-differented label
    into DNA  would facilitate masking for
    the original location at the same time as tracking the histone diffusion
    and enable quantitative FRAP.
    Nevertheless, it is important to recognise
    that such experiments would test the
    resolution and sensitivity of microscopes.

\section{Conclusion}

    Since its inception over 30 years ago, FRAP has been continuously
    improved
    through technical capabilities of light microscopy
    and sensitive kinetic models that are now able to take into account
    an increasing number of biophysical features such as container size,
    non-homogeneous distribution of fluorescence, and profile of bleach spot.

    Despite these advances, the ability to perform FRAP
    over extended time periods of several hours for highly stable complexes
    such as core histones is limited by the dynamic nature of the cell.

    We overcame the challenges of cell motility
    and selection of cells in \G1{} phase,
    but were not able to develop a method to adjust
    for changes in chromatin structure within the cell nucleus.
    While a photobleached spot appears stable and
    can be tracked over several hours,
    small natural disturbances and non-homogeneous diffusion
    impact on photorecovery
    and estimation of kinetic parameters.
    We find that FRAP is suitable for semi-quantitative
    estimates of slowly diffusing molecules
    but not for the precise quantitative comparisons
    required to compare core histone mutations.

    Ultimately, the issue of long observation times stems from the
    requirements of FRAP models to achieve full recovery of the mobile
    populations being measured.

    Single particle tracking is a microscopy technique where, as the
    name suggests, the motion of isolated molecules is observed.  In
    this technique, a small number of particles, small enough that
    they can be resolved and their individual movements tracked, is
    observed over a short amount of time, typically on the scale of
    seconds.  The short time is a limitation and not a requirement of
    the technique, and is caused by observational photobleaching.
    This would remove the requirement of hours long observation and
    overcome the issue of chromatin movement.  In addition, it would
    provide an overview of different types of protein motion where
    FRAP would only provide an average of the movement, and as a super
    resolution microscopy technique, would also be able to identify
    confined movement that FRAP would otherwise classify as immobile
    as been previously the case of MHC class I proteins
    \citep{smith1999anomalous}.
    However, the high concentration of histones in the nucleus makes
    it challenging to observe the required individual molecules.  This
    could possibly be overcame with the use of photo-activatable FPs
    such as PAGPF which would allow for the activation of a small
    population and the use of a Selective Plane Illumination
    Microscopy (SPIM) which allows the observation and excitation of
    fluorescent molecules in a single focal plane.

    Fluorescence Correlation Spectroscopy (FCS) is a fluorescent
    microscopy technique providing estimates of dynamic parameters by
    using fluorescence fluctuations in a femtolitre volume.
    More spatiotemporal dynamics can be obtained by observing the
    fluorescence fluctuations while moving the measurement volume
    across the sample, a technique named Spatio-Temporal Image
    Correlation Spectroscopy (STICS) \citep{hebert2005spatiotemporal}.
    One other variation of FCS is Raster Scan Image Correlation
    Spectroscopy (RICS) which is similar to STICS but can be performed
    on a standard confocal microscope \citep{digman2005rics}.  In both
    cases, image regions that may span the entire cell nuclei may be
    observed and populations with different dynamics localised within
    it.  Such techniques
    could potentially provide not only a method to compare core
    histone mutations \textit{in vivo} but also their effect in
    different chromatin domains.
    However, their implementation is still difficult, from the
    optimisation of scanning parameters to the complex data
    processing, and there are not many papers on the literature
    reporting their use outside the laboratory that developed them.

    Overall, the current period of rapid technological advances in
    cell biology research means that new techniques such as RICS and
    single particle tracking offer
    hope to address specific challenges such as histone mobility for
    which existing approaches such as FRAP are poorly suited.

%  Single-molecule imaging of histones for short period of times in
%  live cells has recently been reported using super-resolution
%  imaging\addref[nature methods 7(9):717-719, 2010 and nature methods
%  8(1):7-9, 2011].

%  Also, use of PAGFP has been used to measure dynamics of H4 over
%  \SI{90}{\ms} reporting differences between interphase chromatin and
%  mitotic chromosomes\addref[Saera Hihara et al 2012].  However, the
%  difference between these two phases is the highest and might not be
%  comparable to the difference between histones variants\todo{study
%  this. Someone must have measured this}.

