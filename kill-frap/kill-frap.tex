  %% \epigraph{Agora desenmerda-te.}{Portuguese ``saying''}

  \begin{chapterabstract}
    Nucleosomes enable the stable compaction of almost all eukaryotic
    genomes but also require dynamic properties to enable access to
    the packaged DNA sequences.  The stability of core histones within the
    nucleosome should be reflected in their capability for dynamic exchange by
    Fluorescence Recovery After Photobleaching (FRAP).  To assay the
    effect of histone SWI/SNF INdependence (SIN)
    mutants known to destabilise nucleosomes
    \textit{in vitro} and in \species{S. cerevisiae}, we sought to
    apply FRAP to chromatin in mammalian cell lines.  This uncovered a
    number of challenges resulting from cell motility, nuclear movement
    within the cell, and chromatin motion within the nucleus with the
    long time frames required for FRAP of histones.  We were able to
    compensate for the former difficulties by a combination of cell
    biological and computational techniques, but we were unable to
    establish an appropriate approach to compensate for motion of the immobile
    binding sites required for standard FRAP analysis.
    Visualisation using photoactivated tagged histones
    demonstrated the extent of this chromatin motion
    and a further complexity arising from complex
    non-homogenous channelling tagged histone
    during diffusion.  This reveals the
    limitations of FRAP over extremely long time scales, and suggests
    that this technique is unsuitable for quantitative measurement of
    histone dynamics in the mammalian nucleus.
  \end{chapterabstract}

  \section{Introduction}

  \subsection{Histone contribution to nucleosome dynamics}

    The building block of eukaryotic chromatin structure is the nucleosome, comprising
    \SI{147}{\bp} of DNA wrapped around an octamer of two copies of core histones H2A,
    H2B, H3, and H4 \citep{luger1997crystal}.
    Nucleosome core particles are arranged in a linear chain separated by DNA linkers, and
    can be further compacted into higher order chromatin structures.
    The role of chromatin not only achieves DNA compaction,
    but also provides a dynamic complex to mediate access to genetic
    information through its capability to undergo reconfiguration of structure. \addref{Flaus?}

    Local remodelling of chromatin can be achieved
    by changing nucleosome structure or altering its composition.
    This can be performed intrinsically through post-translational modification
    or incorporation of histone variants,
    causing nucleosomes to alter DNA sequence preferences or recruit other proteins.
    Alternatively, chaperones and ATP-dependent chromatin remodelling complexes
    can act extrinsically to alter nucleosome structure or position.

  \subsection{Histone SIN mutants}

    The archetype of ATP-dependent nucleosome remodelling is the SWI/SNF complex
    whose deficiency causes growth defects in yeast.
    This complex was identified independently through screens for
    mating type SWItching \citep{SWI-mutants}
    or Sucrose Non Fermentation \citep{SNF-mutants-original-discovery, SNF-mutants2}.

    A set of mutations were identified that compensate
    for the loss of the SWI/SNF complex
    that are collectively known as SIN mutations because they
    provide SWI/SNF INdependence  \addref{?}.
    A subset of these SIN mutations result in single amino-acid changes to core histones,
    providing a direct link between SWI/SNF and chromatin and suggesting
    that the mutated histone protein residues influence the same 
    nucleosome dynamic pathways leveraged by the enzyme and are therefore
    of major importance in the nucleosome structure.
    This predicted that the stability of SIN mutant containing nucleosomes would be affected in chromatin.
    The hypothesis has been tested \textit{in vitro}
    where it was observed that SIN mutant nucleosomes display higher 
    thermally driven nucleosome sliding mobility \citep{flaus2004sin}
    and that the mutated residues alter histone-DNA contacts in crystal structures \addref{muthurajan2004crystal}.

    However, the effect of histone protein SIN mutants
    in the more complex \textit{in vivo} chromatin environment 
    of mammailian cells has not been demonstrated. This is important to
    validate the functional significance of the residues
    and their implications for the basis of high conservation of 
    histone protein sequences in eukaryotes.

  \subsection{Fluorescence Recovery After Photobleaching}

    Fluorescence Recovery After Photobleaching (FRAP) is an optical technique
    that reveals the dynamics of fluorescently tagged molecules within live cells.
    Tagged molecules inside a small region are irreversibly photobleached by
    a high power focused laser beam and the recovery rate of fluorescence
    in the bleached area is measured. The recovery rate is interpreted as unbleached molecules
    from outside of the region at the time of photobleaching diffusing into the bleached area.
    It is assumed that this fluorescence recovery reflects natural protein movement.

    FRAP can be described by a simple chemical equilibrium:

    \begin{displaymath}
      F + S \overset{k_{on}}{\underset{k_{off}}{\rightleftharpoons}} FS
    \end{displaymath}

    where $F$ represents freely diffusing proteins, $S$ represents immobile vacant
    binding sites, and $FS$ the complex between the two, when the protein is bound
    to the binding site. The value of \Kon{} and \Koff{},
    are estimated from the rate at which photobleached $F$ is replaced in the $FS$ complex.

    Ongoing development of FRAP has led to increasingly complex models
    that are both more precise and accurate than simple inverse exponential decay \addref{?}.
    Despite their sophistication, these models require assumptions
    that are difficult to maintain over long experimental observation times.
    Firstly, equilibrium must be maintained throughout the entire experiment 
    so that both \Kon{} and \Koff{} remain constant.
    This also requires that concentrations of both $F$ and $S$ remain constant.
    Secondly, distribution of the fluorescently tagged molecule must mimic the endogenous protein.
    And finally, the binding sites must be part of a large, relatively immobile complex
    on the time and length scale of the recovery.

  \subsection{FRAP measurements of histones}

    FRAP has been extensively used to obtain qualitative and quantitative
    insight on the kinetic properties of proteins, including histones \citep{KimuraCook}.
    These results show extremely slow recovery rates
    of histones with residency half-lives longer than 8~hours.
    \todo{Much more detail on Kimura and Cook results}

  \subsection{Aims}

    In order to observe the implications for nucleosome structure and function for
    the histone SIN mutant H4~R45H that exhibited the
    highest increase in nucleosome mobility \textit{in vitro}
    we attempted to determine its exchange characteristics by FRAP.
    To achieve quantitative measurements we addressed the multiple technical challenges 
    of measuring subtle kinetic alterations of the nucleosome dynamics 
    over long time periods in live cells. This allowed us to define 
    the limitations of FRAP in observing molecules with extremely slow exchange rates.

  \chapter{Materials and Methods}
\label{ch:methods}

\section{Chemicals, solutions and reagents used}
  The chemicals used were obtained from Sigma unless otherwise stated. Solutions were prepared
  according to \Aref{app:solutions} with dH$_2$O and autoclaved prior to use when appropriate.
  
  Restriction enzymes, T4 DNA ligase, and other DNA modifying enzymes, DNA ladders were obtained
  from New England BioLabs. Protein ladder was SeeBlue Plus2 from Invitrogen.
  
\section{DNA methods}
  \subsection{Preparation of competent bacteria}
    Competent \species{E.~coli} cells were prepared from a culture of Invitrogen's One Shot
    TOP10 Chemically Competent \species{E.~coli}. LB cultures of \SI{1}{\l} were set at
    \dc{37} until an OD$_{\SI{600}{\nm}}$ of \numrange{0.4}{0.5}. Further steps were carried
    at \dc{4} with previously chilled equipment and solutions.
    
    Cultures were centrifuged at \SI{6000}{\gn} for 10 minutes, the pellet resuspended in
    \SI{500}{\ml} of \SI{0.1}{\mM} CaCl$_2$, and incubated on ice for 30 minutes. The suspensions
    were centrifuged again at \SI{6000}{\gn} for 10 minutes, and the new pellet resuspended in
    \SI{100}{\ml} of CaCl$_2$ with \pcent{15} glycerol. Aliquots of competent cells were prepared
    and stored at \dc{-80}.
    
    Transformation efficiencies were measured after preparation of each batch and discarded
    if less than \SI{1d6}{\cfu\per\mg} of plasmid was obtained. Absence of antibiotic-resistant
    contaminations was assessed by streaking the cells on selective plates.

  \subsection{Transformation of competent cells}
    Competent cells were thawed on ice and split into aliquots of \SI{50}{\ul} to pre-chilled \SI{2}{\ml}
    tubes where \SI{1}{\ul} DNA was added. Cells were incubated on ice for 30 minutes, followed by a 60 seconds
    heat-shock at \dc{42}, and 5 more minutes on ice. \SI{300}{\ul} of non-selective LB was added to each tube
    and the cultures incubated at \dc{37} with vigorous shaking for 45 minutes. Samples from the cultures
    were plated onto the appropriate antibiotic containing plates, and incubated overnight at \dc{37}.
    
    %% FIXME add concentration of antibiotic used for both agar and broth
    
    For high concentrations of plasmid DNA (more than \SI{500}{\ng\per\ug}), smaller volumes of DNA were used,
    and the initial and final incubation steps were lowered to 10 minutes.

  \subsection{Plasmid DNA preparation}
    Plasmid DNA was prepared with kits from QIAGEN (QIAprep Spin Miniprep, QIAGEN Plasmid \textit{Plus} Midi,
    QIAquick Gel Extraction and QIAquick PCR purification) following the manufacturer's instructions. Once
    prepared, DNA was stored at \dc{-20}. DNA concentrations were measured with a spectrophotometer (NanoDrop
    1000 Spectrophotometer from Thermo Scientific).
    %% FIXME confirm nanodrop model

  \subsection{Ethanol precipitation}
    \label{sec:ethanol-precipitation}
    The DNA solution was mixed with \num{2.5} volumes of \pcent{100} ethanol and \num{1/10} volumes of
    Sodium Acetate (\SI{3}{\Molar}, pH=\num{5.2}), and incubated at \dc{4} for 15 minutes (overnight for
    low DNA plasmid concentrations). Solution was centrifuged at \SI{18000}{\gn} for 30 minutes at \dc{4},
    the supernatant discarded, and the pellet left to dry until all traces of solvent evaporated. DNA pellet
    was resuspended in the desired solvent (usually H$_2$O for transfection).

  \subsection{Agarose gels electrophoresis}
    Agarose gels with of concentrations ranging from \SIrange{0.6}{2.0}{\percent} were prepared with TAE buffer,
    and supplemented with ethidium bromide. Gels were ran in chambers with \SI{1}{$\times$}~TAE buffer at
    \SIrange{80}{120}{\volt} until the required separation was achieved. Gels were visualized at a UV
    transilluminator (ChemiImager 5500 from Alpha Innotech).

  \subsection{DNA sequencing and oligonucleotide preparation}
    DNA sequencing was performed by LGC Genomics after cloning for confirmation and avoid unexpected mutations.
    
    Oligonucleotides were ordered from Eurofins MWG operon in lyophilised format, dissolved in H$_2$O to a
    \SI{100}{\micro\Molar} concentration, and stored at \dc{-20}. A list of all designed oligonucleotides is
    produced at \Aref{app:primers}.

  \subsection{Polimerase Chain Reaction}
    \subsubsection{Colony PCR}
      When screening multiple clones for a specific plasmid and a set of appropriate oligonucleotides was available,
      PCRs were set directly from the bacteria colonies by adding them directly to the reaction mixture
      (\tref{tab:pcr-mixture}). Bacteria from individual colonies were used to simultaneously perform a PCR and
      start a small culture. Plasmid purification was performed on cultures whose sample was positive by PCR.

    \subsubsection{Gene amplification}
    \subsubsection{Mutagenesis}
    \subsubsection{Screening}
    
    \begin{table}
      %% FIXME I should be using subcaptions here
      \centering
      \captionIntro{PCR mixtures.}{}
      \label{tab:pcr-mixture}
      \todo[inline]{get the actual values used here}
      \begin{tabular}{l c c c c}
        \toprule
        \null                                 & colony  & amplification & mutagenesis & screening \\
        \midrule
        Template (\si{\ng})                   & 6       & 67            & 78          & 89    \\
        DMSO                                  & 5       & 78            & 78          & 89    \\
        Buffer                                & 6       &     67        & 78          & 89    \\
        MgSO$_4$                              & 67      &   78          & 78          & 89    \\
        dNTPs                                 & 8       &   78          & 78          & 89    \\
        Primer (forward) (\si{\micro\Molar})  & 8       &   78          & 78          & 89    \\
        Primer (reverse) (\si{\micro\Molar})  & 9       &   78          & 78          & 89    \\
        DNA polymerase                        & 9       &   78          & 78          & 89    \\
        \addlinespace
        Total volume (\si{\ul})               & 25      &   78          & 78          & 89    \\
        \bottomrule
      \end{tabular}

      \begin{threeparttable}
        \captionIntro{PCR conditions.}{}
        \label{tab:pcr-conditions}
        \begin{tabular}{l r<{\si{\second}}@{ at }l<{\si{\degreeCelsius}} r<{\si{\second}}@{ at }l<{\si{\degreeCelsius}}
                          r<{\si{\second}}@{ at }l<{\si{\degreeCelsius}} r<{\si{\second}}@{ at }l<{\si{\degreeCelsius}}}
          \toprule
          \null               & \multicolumn{2}{c}{colony}      & \multicolumn{2}{c}{amplification} &
                                \multicolumn{2}{c}{mutagenesis} & \multicolumn{2}{c}{screening} \\
          \midrule
          Initialization      & 180 & 94                   & 60  & 94                   & 60  & 94                   & 60  & 94                   \\
          Denaturation        & 15  & 94                   & 60  & 94                   & 60  & 94                   & 60  & 94                   \\
          Annealing \tnote{1} & 15  & 94                   & 60  & 94                   & 60  & 94                   & 60  & 94                   \\
          Extension \tnote{2} & 60  & 94                   & 60  & 94                   & 60  & 94                   & 60  & 94                   \\
          Number of cycles    & \multicolumn{2}{c}{30}     & \multicolumn{2}{c}{30}     & \multicolumn{2}{c}{30}     & \multicolumn{2}{c}{30}     \\
          Final extension     & 300 & 72                   & 60  & 94                   & 60  & 94                   & 60  & 94                   \\
          Final hold          & \multicolumn{2}{c}{\dc{4}} & \multicolumn{2}{c}{\dc{4}} & \multicolumn{2}{c}{\dc{4}} & \multicolumn{2}{c}{\dc{4}} \\
          \bottomrule
        \end{tabular}
        \begin{tablenotes}
          \item [1] Temperature is primer dependent. Displayed values correspond to the typical usage.
          \item [2] Time is dependent on sequence length. Displayed time is by \si{\kilo\bp}.
        \end{tablenotes}
      \end{threeparttable}
    \end{table}



\section{Protein methods}
  \subsection{Phenol:chloroform extraction}
    \label{sec:phenol-extraction}
    To extract proteins, an equal volume of phenol:chloroform was added and the
    mixture centrifuged at \SI{6000}{\gn} for 15 minutes. The top aqueous phase (chloroform)
    was pipetted to a new tube and the process repeated a total of 3 times.



  \subsection{Western blotting}
    \subsubsection{Protein concentration determination}
      Concentration of protein was measured with Bradford reagent. \SI{2}{\ul} of the
      sample after sonication (\Sref{sec:cell-extract}) was mixed with \SI{48}{\ul} of H$_2$O
      and \SI{50}{\ul} of NaOH and incubated at \dc{65} for 8 minutes before adding
      \SI{900}{\ul} of Bradford reagent from Pierce. The mixture was transferred to plastic
      cuvettes and the absorvance at \SI{595}{\nm} measured in a Shimadzu spectrophotometer. The
      values obtained were interpolated from a standard curve prepared using known concentrations
      of BSA.
    
    \subsubsection{SDS--PAGE}
      Resolving and stacking SDS--PAGE gels of \SIrange{15}{5}{\percent} respectively, both with a
      cross-linking ratio of \num{37.5}:1. The resolving gel was poured directly after addition
      of TEMED and it was covered with a layer of isopropanol during polymerisation
      to ensure a sharp interface between the resolving and stacking layers.
      Protein samples and markers were boiled at \dc{99} for 3 minutes and each
      was loaded twice, with volumes for \SI{3.3}{\ug} and \SI{16.5}{\ug} of protein. Gels ran at
      \SI{180}{\volt} for 1 hour in \SI{1}{$\times$} TG buffer.
    
    \subsubsection{Protein transfer}
      Protein transfer occurred through the wet transfer system. The SDS-PAGE gel
      was placed onto pre-cut nitrocellulose transfer membrane previously soaked in
      transfer buffer. It was then set between a pair of extra thick blotting paper and
      cushions before being placed inside a transfer apparatus. The transfer ran at
      \dc{4} for 60 minutes.
    
    \subsubsection{Probing of blot with antibody}
      Blocking of the membrane was performed with \SI{10}{\percent} non-fat dry milk in \SI{1}{$\times$} TBST
      at room temperature for 30 minutes. Blocking was followed by primary antibody
      incubation which occurred in \SI{5}{\percent} non-fat dry milk in \SI{1}{$\times$} TBST overnight at
      \dc{4}. Concentrations of antibody used were 1:500 and 1:20000 for anti-GFP
      (catalogue number 11 814 460 001 from Roche) and anti-H3 (code ab1791 from
      abcam). The membrane was then washed with \SI{1}{$\times$} TBST for 15 minutes 3 times
      before the secondary antibody incubation which occurred in \SI{5}{\percent} non-fat dry
      milk in \SI{1}{$\times$} TBST for 1 hour. The membrane was washed once more in the same
      conditions as before for the detection. All blocking, antibody incubation and
      washing steps occurred on a rocker.\todo{should list antibodies and their conditions on appendix}
      
      Detection was performed using the SuperSignal West Pico Chemiluminescent
      Substrate from Pierce, adding 1:1 of the solutions and allowing it to incubate
      with the membrane for 5 minutes. The membrane was exposed to x-ray films for
      10, 60, 5, 180 and 1800 seconds which were then developed.


\section{Tissue culture}
  \subsection{Cell culture}
    HeLa cells were supplied by Agnieszka Kaczmarczyk from National University
    of Ireland, Galway. They were maintained at \dc{37} and \pcent{5} CO$_2$ in \SI{10}{\cm}
    diameter plates with \SI{10}{\ml} of growth medium. Dulbecco's Phosphate Buffered
    Saline (DPBS) and trypsin--EDTA solutions were used, respectively, to wash
    and split the cells 1:10 once they reached a confluence of \SIrange{80}{90}{\percent}.

  \subsection{Cell stock storage}
    For long-term storage of HeLa cell lines, they were grown until they reached
    a confluence of \SIrange{80}{90}{\percent} and then trypsinized as usual. A volume of Freezing
    Media was added, equal to the volume of trypsin--EDTA, and \SI{2}{\ml} aliquots of
    cells transferred to cryotubes. Tubes were immediately wrapped in cotton and
    placed at \dc{-80}.

  \subsection{Whole cell extract}
    \label{sec:cell-extract}
    To obtain whole cell extracts, HeLa cells were trypsinized as usual. Growth
    medium added and the suspension was centrifuged at \SI{900}{\gn} for 10 minutes. Further
    steps were carried at \dc{4} and with previously chilled reagents. The supernatant
    was discarded and the pellet resuspended in \SI{500}{\ul} of chilled PBS before being
    sonicated 3 times at \pcent{40} amplitude for 10 seconds. Avoiding the foam formed at
    the top, \SI{300}{\ul} of suspension were transferred from the bottom of the tube to a
    new eppendorf and mixed with an equal volume of Laemmli buffer before being
    stored at \dc{-80}. \SI{2}{\ul} from the suspension was also transferred to a new tube
    for determination of protein concentration.

  \subsection{Genomic DNA extraction}
    To extract genomic DNA of HeLa cells, they were trypsinized as usual and
    growth medium was added in the end before counting with an hemocytometer.
    Cells were centrifuged at \SI{1500}{\gn} for 10 minutes at \dc{4}, the supernatant discarded
    and the pellet resuspended in TE buffer to achieve a desired concentration of
    \SI{4e7}{cells\per\ml}. 9 volumes of Genomic lysis buffer was added and the mixture
    incubated at \dc{37} for 90 minutes. Proteinase K was added to a final concentration of
    \SI{100}{\ug\per\ml} and the mixture incubated at 50 °C for 3 h and swirled every 20 min.
    DNA was then extracted by phenol:chloroform (\Sref{sec:phenol-extraction}) and purified by ethanol
    precipitation (\Sref{sec:ethanol-precipitation}).

  \subsection{Viable cells count}
    Trypan blue was used to assess the number of viable cells. After trypsination
    as usual, cells were diluted in growth medium, to a final concentration of \SI{2e5}{cells\per\ml}.
    To \SI{0.5}{\ml} of the cell suspension, was added \SI{0.1}{\ml} of \pcent{0.4} Trypan
    Blue Stain and the mixture left for 5 minutes at room temperature before counting
    the cells in an hemocytometer making a distinction between stained (non-viable)
    and non-stained (viable) cells.

  \subsection{Transfection of HeLa cells}
    Cells were transfected using Lipofectamine 2000, a cationic lipid reagent, from
    Invitrogen. Cells were trypsinized as usual on the day before transfection and
    replated on 6 well plates (surface area of \SI{9.5}{\square\cm\per well}) with \SI{2.5}{\ml} of growth
    medium so they would be \SI{90}{\percent} confluent on the following day. For each well,
    two tubes with \SI{250}{\ul} of transfection medium were prepared, one with \SI{7.5}{\ul}
    of Lipofectamine 2000 and another with \SI{3750}{\ng} of DNA (ethanol precipitation
    (\Sref{sec:ethanol-precipitation}) was used to increase the DNA concentration for values around \SI{500}{\ng\per\ul}
    before transfection). Both tubes were incubated at room temperature for 5 minutes,
    mixed together, and incubated again at room temperature for 20 minutes. Cells
    were washed with DPBS during this time and growth medium switched to \SI{2}{\ml}
    of transfection medium. The mixture was then added to the cells medium who
    were incubated at \dc{37} for 6 hours after which time it was switched back to
    \SI{0.5}{\ml} of growth medium.

  \subsection{Fixation and staining of HeLa cells}
    For microscopy visualization, cells grown directly on top of HCl washed coverslips since
    HeLa cells have difficulty attaching to glass. At least 24 hours passed
    between the plating and fixation. Growth medium was removed and the cells
    washed with PBS once before incubation with \SI{4}{\percent} formaldehyde in PBS
    for 4 minutes. The solution was removed and the cells washed with H$_2$O 2 more
    times, after which coverslips were removed from the wells and left to air dry.
    For each coverslip, \SI{2}{\ul} of SlowFade Light Antifade kit from Molecular Probes
    was used for mounting the coverslip on a microscope slide. DAPI was added
    to the mounting media when needed. Coverslips were then sealed with a 1:1
    mixture of clear nail polish and acetone and stored on a dark box at \dc{4}.


\section{Microscopy}
\section{Software used}
  %% we probably should use a script to get the actual version when building the document
  %% this versions numbers are the ones I'm using at the moment... 4 years ago, they didn't exist
  %% must redo the analysis and get the new version numbers.
  The European Molecular Biology Open Software Suite (EMBOSS) version 6.4.0
  was used for analysis of codon usage, RNA folding, sequence alignment, reading of abi
  files, search for restriction sites, prediction of molecular weight and
  other trivial tasks.

  The Perl language and the BioPerl module version 1.6.901 was also extensively used for automation of
  several tasks.
  
  GNU Octave version 3.6.2 and the image package version 2.0.0.
  
  ImageJ version 1.47h packaged though FIJI.

\section{Software developed}





  \section{Results}

    To investigate the challenges of performing FRAP in mammalian cells
    over the several hours required to measure histone mobility,
    we transfected HeLa cells with H2B-EGFP under
    the control of an EF-1\textalpha{} promoter
    and observed recovery in cells for 3.5 hours
    \frefp{fig:kill-frap:cell-movement}.

    \begin{sidewaysfigure}
      \centering
      \subbottom[Pre-bleach]{%
        \label{fig:kill-frap:first-frap-pre}%
        \begin{tikzpicture}[inner sep=0pt]
          \node at (0,0) {\includegraphics[width=6.1cm]%
                          {"results/first-frap-20"}};
          \node[text=white] at (-2.2,2.3) {\SI{10}{\um}};
          \draw[red,thick] (1.95,0.55) circle (5pt);
          \draw[red,thick] (1.20,-0.80) circle (5pt);
          \draw[red,thick] (-0.05,-1.20) circle (5pt);
          \draw[red,thick] (-0.80,-2.30) circle (5pt);
          \draw[red,thick] (-1.55,0.30) circle (5pt);
          \draw[red,thick] (-2.60,1.95) circle (5pt);
        \end{tikzpicture}%
      }
      \subbottom[Post-bleach]{%
        \label{fig:kill-frap:first-frap-post}%
        \begin{tikzpicture}[inner sep=0pt]
          \node at (0,0) {\includegraphics[width=6.1cm]%
                          {"results/first-frap-21"}};
        \end{tikzpicture}%
      }
      %% The time interval between frames was not always equal.  This
      %% was probably to "catch" the slow and fast exchanging
      %% components.  See the dv.log associated with the image.:
      %% Image 20.
      %%      Time:       Mon Dec 14 18:44:08 2009
      %%      Time Point: 32.458 secs
      %% Image 21.
      %%      Time:       Mon Dec 14 18:44:12 2009
      %%      Time Point: 36.715 secs
      %% Image 55.
      %%      Time:       Mon Dec 14 19:36:11 2009
      %%      Time Point: 3156.737 secs
      %% Image 65.
      %%      Time:       Mon Dec 14 20:01:11 2009
      %%      Time Point: 4656.740 secs
      %% Image 75.
      %%      Time:       Mon Dec 14 20:26:11 2009
      %%      Time Point: 6156.742 secs
      %% Image 85.
      %%      Time:       Mon Dec 14 20:51:11 2009
      %%      Time Point: 7656.748 secs
      \\[-1ex] % negative vertical space
      \subbottom[\SI{52}{\min}]{% (3156.7-36.715) / 60
        \label{fig:kill-frap:first-frap:52-min}%
        \begin{tikzpicture}[inner sep=0pt]
          \node at (0,0) {\includegraphics[width=6.1cm]%
                         {"results/first-frap-55"}};
          \draw[->,red,very thick] (1.6,1.3) -- (2.0, 1.3);
          \draw[->,red,very thick] (0.4,-0.4) -- (0.8, -0.4);
          \draw[->,red,very thick] (-0.5,0.35) -- (-0.1, 0.35);
          \draw[->,red,very thick] (-2.0,0.6) -- (-1.6, 0.6);
          \draw[->,red,very thick] (-2.8,2.2) -- (-2.4, 2.2);
          \draw[->,red,very thick] (-1.3,-2.0) -- (-0.9, -2.0);
        \end{tikzpicture}%
      }
      \subbottom[\SI{77}{\min}]{% (4656.7-36.715)/60
        \begin{tikzpicture}[inner sep=0pt]
          \node at (0,0) {\includegraphics[width=6.1cm]%
                          {"results/first-frap-65"}};
        \end{tikzpicture}%
      }
      \subbottom[\SI{102}{\min}]{% (6156.7-36.715)/60
        \label{fig:kill-frap:first-frap:102-min}%
        \begin{tikzpicture}[inner sep=0pt]
          \node at (0,0) {\includegraphics[width=6.1cm]%
                          {"results/first-frap-75"}};
        \end{tikzpicture}%
      }%
      \captionIntro{Circle FRAP of H2B--EGFP in HeLa}%
        {
          FRAP experiment performed in a widefield microscope with
          HeLa stable cell line expressing H2B--EGFP.
          \subcaptionref{fig:kill-frap:first-frap-pre}
          Last of the acquired images before the bleach event.  A
          total of 20 images with a time interval of \SI{1.7}{\sec}
          were acquired before bleaching. Circles show the location
          for the bleaching events.
          \subcaptionref{fig:kill-frap:first-frap-post}
          First image post the bleach event.
          \subcaptionref{fig:kill-frap:first-frap:52-min}%
          --\subcaptionref{fig:kill-frap:first-frap:102-min}
          Selected frames from FRAP experiment.  Arrows show location
          of the bleach spot.  Total time of FRAP
          experiment was \SI{3.5}{\hour} with variable time interval.
          Initially with \SI{15}{\sec} and \SI{2.5}{\min} at the end.
        }
      \label{fig:kill-frap:cell-movement}
    \end{sidewaysfigure}
    %% Could we show a panel B as a plot of
    %% the aspect ratio of the bleach spot and cell over time
    %% as an illustration of the changes.

    The images taken at \SI{25}{\min} intervals revealed considerable changes
    in position of the nuclei, arrangement of features within each nucleus,
    and shape of the bleach spot \frefp{fig:kill-frap:cell-movement}.
    The central requirement of FRAP is to accurately identify and
    quantitate the signal in the photobleached spot over time.
    This led us to pursue both cell biological approaches to minimise motility
    and computational approaches to track imaged regions.

  \subsection{Inhibition of cell motility}

    We first attempted to reduce motility by restricting the space available
    using cells at higher confluency for FRAP experiments.
    This was performed using a HeLa cell line stably expressing
    H4~R45H to ensure even
    tagging of histone fluorescence in all cells.
    This resulted in some decrease in movement of cells
    but did not achieve complete immobilisation
    \frefp{fig:kill-frap:confluent-hela}.
    In fact, nuclei frequently underwent considerable reshaping as cells
    apparently squeezed between their neighbours.

    \begin{figure}
      \centering
      \begin{tikzpicture}[inner sep=0pt]
        \node at (0,0) {\includegraphics[width=\linewidth]%
                        {"results/confluent-hela"}};
        %% See http://tex.stackexchange.com/a/64458/24374
        %% for the countouring of text (only works for pdf)
        \node at (-5.41,2.90) {\pgfsetstrokecolor{black}
                             \pgfsetfillcolor{white}
                             \pdfliteral{2 Tr}
                             \SI{10}{\um}};
      \end{tikzpicture}%
      %% We show only the outline of the nuclei so to not distract
      %% from the message which is how the nuclei kept moving.
      \captionIntro{Movement of confluent HeLa cells during a FRAP experiment}
        {
          HeLa-derived stable cell line expressing H4~R45H--YFP
          were observed in a confocal microscope over 8~hours with
          \SI{1}{\min} interval.  Figure shows only the nuclei outline
          in intervals of \SI{50}{\min}.
        }
      \label{fig:kill-frap:confluent-hela}
    \end{figure}

    %% Could we show panel B as a plot of
    %% the deviation of the cell or nuclear centroid in X and Y
    %% relative to the first frame as a representative example
    %% for the HeLa cells in the first and second figures.
    %% This needs to be in absolute micron units
    %% independent of the magnification.
    %% We could even show the distribution of aspect ratio deviations
    %% for a sample of individual cells as a box plot as panel 2C

    Fibroblast and epithelial cells display the property of contact
    inhibition of locomotion \citep{abercrombie1970contact}.
    In this cellular growth response, cells attempt to move in an
    opposite direction after contact with another.  As the number of
    cells increases and they become surrounded by neighbours, the
    available directions are reduced.
    Like most cancer cell lines, HeLa cells have
    lost the ability to activate contact inhibition
    \citep{stephenson1982locomotory},
    so we obtained a primary horse fibroblast cell line.
    Because primary cell line transfections typically have much lower
    efficiency than cancer cell lines, we transfected cells at
    \pcent{70} confluence and performed FRAP after 3 days.
    This timing enabled us to perform the transfection while cells
    were actively dividing, which increases the
    efficiency, and to perform the imaging once they reached confluence for
    reduced motility.

    Despite reaching confluence where contact inhibition
    of the fibroblasts was expected,
    we continued to observe movement of the transfected
    horse cells \frefp{fig:kill-frap:confluent-horse}.
    The characteristics of cell movement also
    differed dramatically from HeLa cells.
    The horse cell nuclei exhibit a helical motion
    about the vector of their movement
    in $x$ and $y$ axes \frefp{fig:kill-frap:confluent-horse},
    whereas HeLa nuclear motion was mostly
    restricted to the $z$ axis relative to
    the dish \frefp{fig:kill-frap:cell-movement}.

    \begin{figure}
      \centering
      \includegraphics[width=\textwidth]{results/confluent-horse.png}
      \captionIntro{Movement of confluent primary horse
                    cells during a FRAP experiment}
        {
          Immortalised horse cells were transfected with pBOS--H2B-EGFP
          with circle FRAP performed in a widefield microscope over 8~hours.
          Pre-bleach image at top left followed by
          image sequence at \SI{15}{\min} intervals.
        }
      \label{fig:kill-frap:confluent-horse}
    \end{figure}

    \subsection{Image-based tracking of cell movement}

    As an alternative strategy we implemented CropReg, a script
    to automate cell tracking of time series sequences.
    Using this image processing approach we were
    able to track individual HeLa cell nuclei
    throughout an entire sequence of FRAP images
    provided that nuclei did not overlap \frefp{fig:kill-frap:cropreg}.
    Although only a minority of cell image sequences satisfied this requirement
    throughout the full 8~hour duration of observations,
    it was possible to collect a sufficient number of
    cell observations for FRAP calculations.

    \begin{figure}
      \centering
      \includegraphics[width=\textwidth]{results/cropreg.png}
      \captionIntro{Automatic tracking and alignment of moving cells}
        {
         HeLa-derived stable cell line expressing H3--YFP
         with circle FRAP performed in a widefield microscope over 8~hours.
         Pre-bleach image at top left followed by
         image sequence at 20~min intervals.
         The top left of each image displays the cell tracked and transformed
         by automated cropping and registration.
        }
      \label{fig:kill-frap:cropreg}
    \end{figure}

  \subsection{Chromatin movement within nuclei}

    While performing the FRAP experiments, we observed movement
    of fluorescent chromatin features
    within cell nuclei that was supplementary to the overall
    motion of the cell itself \frefp{fig:kill-frap:frap-spot-movement}.
    This could not be accounted for by simple rotational movement of nuclei
    as rigid bodies around the $x$ or $y$ axis,
    and instead appeared to involve movement of
    individual regions within nuclei.
    Bleach spots also frequently showed elliptical or more complex distortions
    indicative of structural movements in the chromatin
    \frefp{fig:kill-frap:frap-spot-movement}.

    \begin{figure}
      \centering
      \begin{tikzpicture}[inner sep=0pt]
        \node at (0,0) {\includegraphics[width=\linewidth]%
                       {results/spot-distortion.png}};
        \node[text=white] at (-5.4,3.9) {\SI{5}{\um}};
        \draw[black,very thick] (-5.2,3.2) circle (7pt);
        \draw[->,black,very thick] (-2.1,2.5) -- (-2.1, 2.9);
        \draw[->,black,very thick] (1.0,2.5) -- (1.0, 2.9);
        \draw[->,black,very thick] (-1.9,-3.4) -- (-1.9, -3.0);
        \draw[->,black,very thick] (1.2,-3.4) -- (1.2, -3.0);
      \end{tikzpicture}
      \captionIntro{Movement of nuclear features and bleach spot distortion}
        {
          FRAP experiment performed in a confocal microscope with HeLa
          stable cell line expressing H4--YFP.  Circle FRAP was
          performed after 15 pre-bleach images and images acquired
          with \SI{60}{\sec} interval for a total of 4~hours.  Top two
          frames show the bleaching event, followed by the initial 3
          hours of recovery with \SI{20}{\min} interval.
        }
      \label{fig:kill-frap:frap-spot-movement}
    \end{figure}

    \subsection{Selection of \G1{} cells}

    One possible cause for changes in chromatin features that we observed
    is DNA replication and chromatin repackaging during S~phase.
    Furthermore, the doubling of histone content as a result of S~phase breaks
    a core assumption of FRAP that the system remains in equilibrium
    throughout the duration of the experiment.
    Therefore, the requirement to measure for a time period of 8~hours
    within a single cell cycle phase limits observations to \G1{}.

    To identify daughter cells that could be confidently assigned to early \G1{}
    because they had sufficient time to complete
    post-mitotic chromatin unpacking,
    cells in mitosis were selected and manually tracked for 4~hours.
    This selection was challenging because HeLa mitotic cells round up
    as spheres with only a weak connection to the growth surface
    causing them to exit the field of vision.
    Low laser power and a \SI{30}{\min} observation interval
    was used to minimise fluorophore damage.
    Since our system did not permit simultaneous
    Z-stack and time lapse imaging,
    and because cells in mitosis are in a separate focal plane,
    imaging was performed with a maximal pinhole sized focused
    between the growing and mitotic cell planes.
    The resulting blurred images were sufficient to visualise cells during
    the entire period required for selection.
    However, even after carefully selecting cells early \G1{},
    structural movements within the bleached region could
    still be observed (data not shown).

%      \begin{figure}
%        \centering
%        \missingfigure{Hela cells splitting}
%        \captionIntro{Picking cells at early G$_1$.}
%                     {We imaged cells that were entering mitosis and picked their
%                      daughter cells for the FRAP experiments. Because HeLa cells lift
%                      away from the dish during mitosis, opening the
%                      pinhole and set the Z-center in between the cell dividing plane
%                      and dish bottom was necessary. Ends up nothing being properly in focus but we
%                      can track things fine. Of course, some cells still floated away.}
%        %% TODO explicit parameters
%        \label{fig:kill-frap:picking-early-g1}
%      \end{figure}


    \subsection{Chromatin movement observed by photoactivation}

    Although we had surmounted the technical challenges of collecting
    overlaid images of nuclei for long time periods
    in \G1{} cells containing stably expressing
    core histones tagged with fluorescent reporters,
    we were concerned about the non-homogeneity of chromatin behaviour.

    To assess the extent of the chromatin motion,
    we performed photoactivation
    to track the movement of chromatin in the activated region alone.
    For this purpose we fused H2B to photoactivatable GFP as H2B--PAGFP.
    Since PAGFP cannot be easily detected before photoactivation,
    cells were co-transfected with mCherry--\textalpha--tubulin
    which localises exclusively to the cytoplasm
    and provides an outline of the nuclear region
    \frefp{fig:kill-frap:ifrap-pre}.

    Considerable non-homogenous movement of chromatin was clearly evident
    after activating and following H2B--PAGFP in
    \G1{} cells \frefp{fig:kill-frap:ifrap}.
    Instead of homogeneous diffusion of fluorescence,
    activated spots uncurled over time with individual channels of
    localised PAGFP appearing in the nuclei \frefp{fig:kill-frap:ifrap}.
    This finding suggests that quantitative FRAP is not tractable using
    a simple FRAP model based on homogenous non-directional diffusion.

    \begin{figure}
      \centering
      \subbottom[pre-activation]{
        \label{fig:kill-frap:ifrap-pre}
        \begin{tikzpicture}[inner sep=0pt]
          \node at (0,0) {\includegraphics[width=0.45\linewidth]%
                          {results/ifrap-pre.png}};
          \node[text=white] at (1.6,-1.6) {\SI{10}{\um}};
        \end{tikzpicture}
      }
      \hfill
      \subbottom[post-activation]{
        \label{fig:kill-frap:ifrap-post}
        \includegraphics[width=0.45\textwidth]
        {results/ifrap-post.png}
      }
      \subbottom[activated spot over time]{
        \label{fig:kill-frap:ifrap-timeframe}
        \begin{tikzpicture}[inner sep=0pt]
          \node at (0,0) {\includegraphics[width=\linewidth]%
                          {results/ifrap.png}};
          \node[text=white] at (-5.4,3.1) {\SI{2}{\um}};
        \end{tikzpicture}
      }
      %% Can you add an additional section in ifrap figure with 3-4
      %% still images of different very stark non-homogenous diffusion
      %% examples
      \captionIntro{Photoactivation experiment demonstrating
                    complex chromatin movement}
        {
          HeLa cells co-transfected with H2B--PAGFP and
          mCherry--\textalpha--tubulin.
          \subcaptionref{fig:kill-frap:ifrap-pre}
          Cell shown before photoactivation of H2B--PAGFP
          in a widefield microscope, demonstrating
          mCherry-bounded nuclear region.
          \subcaptionref{fig:kill-frap:ifrap-post}
          Same cell shown immediately after photoactivation
          showning circle photoactivated H2B--PAGFP nuclear spot.
          \subcaptionref{fig:kill-frap:ifrap-timeframe}
          Image sequence at 20~min intervals
          after photoactivation showing complex
          channelling of H2B--PAGFP diffusion.
        }
      \label{fig:kill-frap:ifrap}
    \end{figure}

  \section{Discussion}
  \subsection{The FRAP/FLAP model}
  \subsection{Development of MaryI}
  \subsection{HFD association rates}
  \subsection{HFD stoichiometry on the kinetochore}
  \subsection{Involvement of FACT in assembly of CENP-T/W complex}
  \subsection{Cell cycle}
    \subsubsection{Timing of assembly with respect to DNA replication}
    \subsubsection{Dynamics as a function of the cell cycle phase}
  \subsection{Requirements for CCAN in HFD CENP assembly}
    \subsubsection{CENP-T/W influence on CENP-S/X}

