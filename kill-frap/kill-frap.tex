\chapter{Quantitative dynamics of histones in the human cell nucleus}
\label{ch:kill-frap}

  %% \epigraph{Agora desenmerda-te.}{Portuguese ``saying''}

  %% chapter concept: in this chapter goes the whole kill FRAP project. I started
  %% positive that this should work and so should the text. We start by studying
  %% the technique and list the assumptions it requires. We do one experiment and
  %% face some problems. Each problem has its own section that ends in a solution.
  %% When the solution was not done, we could offer one. The last one is the cause
  %% that this is not possible and has no solution. Conclusion lists all problems
  %% and their solutions again.

  %% AF: Think this needs to be an abstract, summarising capter in 250 words
  %% Why would you have anything else than an abstract?

  \chapterprecis{
    To further the understanding of nucleosome structure and positioning
    in DNA, we aim to measure kinetic differences between wild type and
    mutant histones in live cells. Fluorescent Recovery After Photobleaching
    is a technique successfully used before to show the different kinetics and
    populations between different histone types. Using the same approach with
    our most disrupting histone mutant, we test the limits of FRAP when
    faced with extremely slow exchange ratios.
  }

  \section{Introduction}

  \subsection{Histone contribution to nucleosome dynamics}

    The building block of eukaryotic chromatin structure is the nucleosome, comprising
    \SI{147}{\bp} of DNA wrapped around an octamer of two copies of core histones H2A,
    H2B, H3, and H4 \citep{luger1997crystal}.
    Nucleosome core particles are arranged in a linear chain separated by DNA linkers, and
    can be further compacted into higher order chromatin structures.
    The role of chromatin not only achieves DNA compaction,
    but also provides a dynamic complex to mediate access to genetic
    information through its capability to undergo reconfiguration of structure. \addref{Flaus?}

    Local remodelling of chromatin can be achieved
    by changing nucleosome structure or altering its composition.
    This can be performed intrinsically through post-translational modification
    or incorporation of histone variants,
    causing nucleosomes to alter DNA sequence preferences or recruit other proteins.
    Alternatively, chaperones and ATP-dependent chromatin remodelling complexes
    can act extrinsically to alter nucleosome structure or position.

  \subsection{Histone SIN mutants}

    The archetype of ATP-dependent nucleosome remodelling is the SWI/SNF complex
    whose deficiency causes growth defects in yeast.
    This complex was identified independently through screens for
    mating type SWItching \citep{SWI-mutants}
    or Sucrose Non Fermentation \citep{SNF-mutants-original-discovery, SNF-mutants2}.

    A set of mutations were identified that compensate
    for the loss of the SWI/SNF complex
    that are collectively known as SIN mutations because they
    provide SWI/SNF INdependence  \addref{?}.
    A subset of these SIN mutations result in single amino-acid changes to core histones,
    providing a direct link between SWI/SNF and chromatin and suggesting
    that the mutated histone protein residues influence the same 
    nucleosome dynamic pathways leveraged by the enzyme and are therefore
    of major importance in the nucleosome structure.
    This predicted that the stability of SIN mutant containing nucleosomes would be affected in chromatin.
    The hypothesis has been tested \textit{in vitro}
    where it was observed that SIN mutant nucleosomes display higher 
    thermally driven nucleosome sliding mobility \citep{flaus2004sin}
    and that the mutated residues alter histone-DNA contacts in crystal structures \addref{muthurajan2004crystal}.

    However, the effect of histone protein SIN mutants
    in the more complex \textit{in vivo} chromatin environment 
    of mammailian cells has not been demonstrated. This is important to
    validate the functional significance of the residues
    and their implications for the basis of high conservation of 
    histone protein sequences in eukaryotes.

  \subsection{Fluorescence Recovery After Photobleaching}

    Fluorescence Recovery After Photobleaching (FRAP) is an optical technique
    that reveals the dynamics of fluorescently tagged molecules within live cells.
    Tagged molecules inside a small region are irreversibly photobleached by
    a high power focused laser beam and the recovery rate of fluorescence
    in the bleached area is measured. The recovery rate is interpreted as unbleached molecules
    from outside of the region at the time of photobleaching diffusing into the bleached area.
    It is assumed that this fluorescence recovery reflects natural protein movement.

    FRAP can be described by a simple chemical equilibrium:

    \begin{displaymath}
      F + S \overset{k_{on}}{\underset{k_{off}}{\rightleftharpoons}} FS
    \end{displaymath}

    where $F$ represents freely diffusing proteins, $S$ represents immobile vacant
    binding sites, and $FS$ the complex between the two, when the protein is bound
    to the binding site. The value of \Kon{} and \Koff{},
    are estimated from the rate at which photobleached $F$ is replaced in the $FS$ complex.

    Ongoing development of FRAP has led to increasingly complex models
    that are both more precise and accurate than simple inverse exponential decay \addref{?}.
    Despite their sophistication, these models require assumptions
    that are difficult to maintain over long experimental observation times.
    Firstly, equilibrium must be maintained throughout the entire experiment 
    so that both \Kon{} and \Koff{} remain constant.
    This also requires that concentrations of both $F$ and $S$ remain constant.
    Secondly, distribution of the fluorescently tagged molecule must mimic the endogenous protein.
    And finally, the binding sites must be part of a large, relatively immobile complex
    on the time and length scale of the recovery.

  \subsection{FRAP measurements of histones}

    FRAP has been extensively used to obtain qualitative and quantitative
    insight on the kinetic properties of proteins, including histones \citep{KimuraCook}.
    These results show extremely slow recovery rates
    of histones with residency half-lives longer than 8~hours.
    \todo{Much more detail on Kimura and Cook results}

  \subsection{Aims}

    In order to observe the implications for nucleosome structure and function for
    the histone SIN mutant H4~R45H that exhibited the
    highest increase in nucleosome mobility \textit{in vitro}
    we attempted to determine its exchange characteristics by FRAP.
    To achieve quantitative measurements we addressed the multiple technical challenges 
    of measuring subtle kinetic alterations of the nucleosome dynamics 
    over long time periods in live cells. This allowed us to define 
    the limitations of FRAP in observing molecules with extremely slow exchange rates.

  \section{Materials preparation}

  \subsection{Plasmid construction}
    \label{sec:kill-frap:cloning}

    Plasmids for the canonical histones H2B, H3, and H4,
    respectively pBOS--H2B--GFP \citep{KevinH2BGFP},
    pBOS--H3--EYFP.MC--N1, and pBOS--H4--ECFP.M--N1,
    were provided by Prof. Kevin Sullivan from National University of Ireland,
    Galway (NUIG). DNA sequencing identified the H3 plasmid as the HIST1H3B gene,
    encoding H3.1 from histone cluster 1; the H4 plasmid as either HIST1H4J or HIST1H4K;
    and the H2B plasmid similar to HIST1H2BJ but with missense mutations D25G and V118I.
    Plasmid pPAGFP--N1 and mCherry--\textalpha--tubulin were provided by Chelly van Vuuren
    and the HeLa cDNA library was a kind gift of Nadine Quinn \citep{NadineThesis}.
    Plasmid pMH3.2--614 including a mouse replication dependent histone~H3
    gene with upstream and downstream regulatory elements \citep{pMH3-plasmid},
    was provided by Prof. Kevin Sullivan.

    \paragraph{H2B--EGFP}
      The D25G and V118I mutations in pBOS--H2B--GFP were corrected by PCR mutagenesis
      using primers AFG114 and AFG115, and AFG112 and AFG113 respectively.
      The resulting product is H2B--EGFP, equivalent to the HIST1H2BJ gene product fusion.

    \paragraph{H4--ECFP R45H}
      The R45H mutation was applied to the pBOS--H4--ECFP.M--N1 by
      PCR mutagenesis using the primers AFG124 and AFG125. The codon
      \texttt{CAC} was selected for the histidine amino acid due to its
      higher codon usage in the human genome\citep{codon_usage}.

    \paragraph{H2A--EGFP}
      Plasmid pBOS--H2B--GFP was digested with KpnI and BamHI
      and the band corresponding to the linearised vector without the H2B sequence
      was purified by gel extraction. The HIST1H2AB sequence was amplified
      from HeLa genomic DNA with primers AFG116 and AFG118 and ligated into the vector.
      This coding sequence is equivalent to the H2A used in the previous
      \textit{in vitro} studies to be compared \citep{flaus2004sin}.
      HIST1H2AE has the same gene product but a lower codon adaptation index
      and more complex predicted 5' mRNA secondary structures.

    \paragraph{H2AX--EGFP and S139 mutants}
      Cloning of H2AX was similar to HIST1H2AB but used primers AFG130 and AFG131.
      Mutations to H2AX S139, an important site that is phosphorylated during
      DNA damage response, was performed at the same time of gene amplification
      since its location is close to the sequence 3' end. Primers AFG132, AFG133, and AFG134,
      were used with AFG130 to introduce mutations S139A, S139D, and S139E respectively.
      The mutation to alanine blocks, while mutation to aspartic and glutamic acid
      mimic phosphorylation. This strategy introduced an accidental frameshift mutation
      near the stop codon which was fixed by PCR mutagenesis using primers AFG400
      and AFG401.

    \paragraph{H2A.Z--EGFP}
      The H2A.Z sequence was amplified from an HeLa cDNA library
      using primers AFG121 and AFG122
      due to the existence of introns in the H2AFZ gene.
      Purification of the amplicon and ligation to the linearised pBOS--EGFP vector
      was identical to the preparation of the H2A--GFP plasmid.

    \paragraph{H4--EYFP}
      The plasmids pBOS--H3--EYFP.MC--N1 and pBOS--H4--ECFP.M--N1 were
      digested with the restriction enzymes BamHI and NotI. After agarose
      gel electrophoresis, the EYFP insert and pBOS--H4 vector were purified
      by gel extraction. The two DNA fragments were ligated to construct
      pBOS-H4-EYFP. The same strategy was used to construct the EYFP tagged
      H4~R45H mutant.

    \paragraph{H3--EYFP T45A and T45E}
      Mutations to H3 T45 were inserted into the pBOS--H3--EYFP.MC--N1 by
      PCR mutagenesis. The primers AFG151 and AFG152 were used for the
      T45E mutation, and AFG153 and AFG154 for T45A.

    \paragraph{H2B and H3 PAGFP}
      The PAGFP insert was prepared from pPAGFP--N1 by PCR using the
      primers AFG478 and AFG479, the amplicon purified by agarose gel
      extraction, digested with NotI and BamHI, and finally cleaned by
      PCR purification.
      Both pBOS--H2B--GFP and pBOS--H3--EYFP.MC--N1 plasmids were also
      digested with NotI and BamHI to excise their tags, and the vectors
      purified by agarose gel extraction. The insert was finally ligated
      into the two vectors for pBOS--H2B--PAGFP and pBOS--H3--PAGFP.
      This strategy introduces a Proline to Arginine mutation in the
      linker for the H2B plasmid.
      %% This mutation in the H2B linker (DPPVAT to DPRVAT) was on purpose.
      %% We could have easily avoid it but would cost us one extra primer
      %% and it shouldn't be making a difference.

    \paragraph{pMH2B--GFP and pMH3--EYFP}
      %% I'm actually not sure if Kevin gave me the pMH3.2--614 or the
      %% pCA-TAG plasmid. I did not have any plasmid map or sequence, only
      %% the very small Figure 5 of his paper PMID:9024683
      %% I couldn't even use any standard sequencing primer and by the time
      %% we cloned our genes there, we had already decided to kill the project
      %% so we never got to actually try these in human cells.
      Insertion of the H2B--GFP and H3--EYFP coding sequences into the
      pMH3.2--614 plasmid was performed by PCR amplification, bluent-end
      ligation of both vector and inserts due to the absence of restriction
      sites. pMH3.2--614 was amplified with primers AFG417 and AGF418 which
      create a linear sequence that only ignore the H3.2 coding sequence.
      The H2B--GFP coding sequence was generated with primers AFG419 and AFG420.
      Primers for the insert were phosphorylated by T4~PNK prior to the PCR
      since T4~PNK is more efficient on single stranded DNA. pM vector and
      H2B--GFP insert were purified by agarose gel extraction and ligated.
      Strategy for H3--EYFP was equivalent but using primers AFG424 and AFG420.

  \subsection{Cell lines}

    Transfection was always performed by lipofection~(\Sref{methods:lipofection}),
    both for transient and stable cell lines. For creation of stable cell lines,
    cells were trypsinized and split 1:20 on \SI{10}{\cm} dishes 24~hours after transfection.
    After another 24 hours, Blasticidin-S was added to the medium for
    a final concentration of \SI{3}{\ug\per\ml} as found by performing a
    Blasticidin-S kill-curve (\Sref{sec:methods:kill-curve}).
    Cell growth was followed and medium replaced when appropriate.
    As cell colonies started to be visible by the naked eye, approximately
    3~weeks after plating, these were screened by fluorescence microscopy.
    Positive colonies were aspirated and moved into 24-well plates with
    \SI{1}{\ml} disposable pipette tips, and the thinnest extremity removed.

    \begin{figure}
      \centering
      \includegraphics[width=\textwidth]{kill-frap/figs/facs-stable-cell-lines.pdf}
      \captionIntro{FACS sorting of mixed populations}
        {
          The multiple populations with differing intensity values were
          sorted by FACS. The full line represents the intensity profile
          of HeLa wild type cells; the dash-dot line, a mixed population
          of HeLa cells expressing H2B--EGFP; and the dotted line, the
          sorted population.
        }
      \label{fig:methods:facs}
    \end{figure}

    Populations with mixed levels of fluorescent intensity were frequently
    obtained while preparing stable cell lines. In such cases, cells with
    similar intensity of their corresponding fluorophore were FACS
    sorted (\fref{fig:methods:facs}).

    The following stable lines were prepared:

    \begin{itemize}
      \item HeLa H2B--EGFP
      \item HeLa H2B--EGFP D25G V118I
      \item HeLa H3--EYFP
      \item HeLa H3--EYFP T45A
      \item HeLa H3--EYFP T45E
      \item HeLa H4--EYFP
      \item HeLa H4--EYFP R45H
    \end{itemize}

  \subsection{Microscopy}

    Both stable and transiently cell lines were used in imaging as described.
    Transfection was performed 48~hours before imaging for transiently
    expressing cells.

    Confocal microscopy was performed with a Zeiss LSM510 Meta microscope
    using glass bottom LabTek II chambers. Wide-field fluorescence microscopy
    was performed with an Applied Precision DeltaVision Core system
    using \SI{35}{\mm} glass bottom MatTek dishes.

    In both cases, imaging was performed within an acrylic environmental
    chamber that fully enclosed the stage plate and microscope objectives.
    Temperature and CO$_2$ levels were maintained via separate units connected
    to the environmental chamber.

  \subsection{Computational analysis}

    Software used for image analysis and visualization was described in
    \Sref{sec:methods:software}. Original source code written in \textsc{Matlab} for a previously
    reported circle FRAP model \citep{mueller2008evidence} was kindly offered
    to us under the GNU General Public License (GPL) version~3 by the
    original authors. A port of this code for the GNU Octave language was
    prepared and made available under the same license.

  \section{Results}

    To investigate the challenges of performing FRAP in mammalian cells
    over the several hours required to measure histone mobility,
    we transfected HeLa cells with H2B-EGFP under
    the control of an EF-1\textalpha{} promoter
    and observed recovery in cells for 3.5 hours
    \frefp{fig:kill-frap:cell-movement}.

    \begin{sidewaysfigure}
      \centering
      \subbottom[Pre-bleach]{%
        \label{fig:kill-frap:first-frap-pre}%
        \begin{tikzpicture}[inner sep=0pt]
          \node at (0,0) {\includegraphics[width=6.1cm]%
                          {"results/first-frap-20"}};
          \node[text=white] at (-2.2,2.3) {\SI{10}{\um}};
          \draw[red,thick] (1.95,0.55) circle (5pt);
          \draw[red,thick] (1.20,-0.80) circle (5pt);
          \draw[red,thick] (-0.05,-1.20) circle (5pt);
          \draw[red,thick] (-0.80,-2.30) circle (5pt);
          \draw[red,thick] (-1.55,0.30) circle (5pt);
          \draw[red,thick] (-2.60,1.95) circle (5pt);
        \end{tikzpicture}%
      }
      \subbottom[Post-bleach]{%
        \label{fig:kill-frap:first-frap-post}%
        \begin{tikzpicture}[inner sep=0pt]
          \node at (0,0) {\includegraphics[width=6.1cm]%
                          {"results/first-frap-21"}};
        \end{tikzpicture}%
      }
      %% The time interval between frames was not always equal.  This
      %% was probably to "catch" the slow and fast exchanging
      %% components.  See the dv.log associated with the image.:
      %% Image 20.
      %%      Time:       Mon Dec 14 18:44:08 2009
      %%      Time Point: 32.458 secs
      %% Image 21.
      %%      Time:       Mon Dec 14 18:44:12 2009
      %%      Time Point: 36.715 secs
      %% Image 55.
      %%      Time:       Mon Dec 14 19:36:11 2009
      %%      Time Point: 3156.737 secs
      %% Image 65.
      %%      Time:       Mon Dec 14 20:01:11 2009
      %%      Time Point: 4656.740 secs
      %% Image 75.
      %%      Time:       Mon Dec 14 20:26:11 2009
      %%      Time Point: 6156.742 secs
      %% Image 85.
      %%      Time:       Mon Dec 14 20:51:11 2009
      %%      Time Point: 7656.748 secs
      \\[-1ex] % negative vertical space
      \subbottom[\SI{52}{\min}]{% (3156.7-36.715) / 60
        \label{fig:kill-frap:first-frap:52-min}%
        \begin{tikzpicture}[inner sep=0pt]
          \node at (0,0) {\includegraphics[width=6.1cm]%
                         {"results/first-frap-55"}};
          \draw[->,red,very thick] (1.6,1.3) -- (2.0, 1.3);
          \draw[->,red,very thick] (0.4,-0.4) -- (0.8, -0.4);
          \draw[->,red,very thick] (-0.5,0.35) -- (-0.1, 0.35);
          \draw[->,red,very thick] (-2.0,0.6) -- (-1.6, 0.6);
          \draw[->,red,very thick] (-2.8,2.2) -- (-2.4, 2.2);
          \draw[->,red,very thick] (-1.3,-2.0) -- (-0.9, -2.0);
        \end{tikzpicture}%
      }
      \subbottom[\SI{77}{\min}]{% (4656.7-36.715)/60
        \begin{tikzpicture}[inner sep=0pt]
          \node at (0,0) {\includegraphics[width=6.1cm]%
                          {"results/first-frap-65"}};
        \end{tikzpicture}%
      }
      \subbottom[\SI{102}{\min}]{% (6156.7-36.715)/60
        \label{fig:kill-frap:first-frap:102-min}%
        \begin{tikzpicture}[inner sep=0pt]
          \node at (0,0) {\includegraphics[width=6.1cm]%
                          {"results/first-frap-75"}};
        \end{tikzpicture}%
      }%
      \captionIntro{Circle FRAP of H2B--EGFP in HeLa}%
        {
          FRAP experiment performed in a widefield microscope with
          HeLa stable cell line expressing H2B--EGFP.
          \subcaptionref{fig:kill-frap:first-frap-pre}
          Last of the acquired images before the bleach event.  A
          total of 20 images with a time interval of \SI{1.7}{\sec}
          were acquired before bleaching. Circles show the location
          for the bleaching events.
          \subcaptionref{fig:kill-frap:first-frap-post}
          First image post the bleach event.
          \subcaptionref{fig:kill-frap:first-frap:52-min}%
          --\subcaptionref{fig:kill-frap:first-frap:102-min}
          Selected frames from FRAP experiment.  Arrows show location
          of the bleach spot.  Total time of FRAP
          experiment was \SI{3.5}{\hour} with variable time interval.
          Initially with \SI{15}{\sec} and \SI{2.5}{\min} at the end.
        }
      \label{fig:kill-frap:cell-movement}
    \end{sidewaysfigure}
    %% Could we show a panel B as a plot of
    %% the aspect ratio of the bleach spot and cell over time
    %% as an illustration of the changes.

    The images taken at \SI{25}{\min} intervals revealed considerable changes
    in position of the nuclei, arrangement of features within each nucleus,
    and shape of the bleach spot \frefp{fig:kill-frap:cell-movement}.
    The central requirement of FRAP is to accurately identify and
    quantitate the signal in the photobleached spot over time.
    This led us to pursue both cell biological approaches to minimise motility
    and computational approaches to track imaged regions.

  \subsection{Inhibition of cell motility}

    We first attempted to reduce motility by restricting the space available
    using cells at higher confluency for FRAP experiments.
    This was performed using a HeLa cell line stably expressing
    H4~R45H to ensure even
    tagging of histone fluorescence in all cells.
    This resulted in some decrease in movement of cells
    but did not achieve complete immobilisation
    \frefp{fig:kill-frap:confluent-hela}.
    In fact, nuclei frequently underwent considerable reshaping as cells
    apparently squeezed between their neighbours.

    \begin{figure}
      \centering
      \begin{tikzpicture}[inner sep=0pt]
        \node at (0,0) {\includegraphics[width=\linewidth]%
                        {"results/confluent-hela"}};
        %% See http://tex.stackexchange.com/a/64458/24374
        %% for the countouring of text (only works for pdf)
        \node at (-5.41,2.90) {\pgfsetstrokecolor{black}
                             \pgfsetfillcolor{white}
                             \pdfliteral{2 Tr}
                             \SI{10}{\um}};
      \end{tikzpicture}%
      %% We show only the outline of the nuclei so to not distract
      %% from the message which is how the nuclei kept moving.
      \captionIntro{Movement of confluent HeLa cells during a FRAP experiment}
        {
          HeLa-derived stable cell line expressing H4~R45H--YFP
          were observed in a confocal microscope over 8~hours with
          \SI{1}{\min} interval.  Figure shows only the nuclei outline
          in intervals of \SI{50}{\min}.
        }
      \label{fig:kill-frap:confluent-hela}
    \end{figure}

    %% Could we show panel B as a plot of
    %% the deviation of the cell or nuclear centroid in X and Y
    %% relative to the first frame as a representative example
    %% for the HeLa cells in the first and second figures.
    %% This needs to be in absolute micron units
    %% independent of the magnification.
    %% We could even show the distribution of aspect ratio deviations
    %% for a sample of individual cells as a box plot as panel 2C

    Fibroblast and epithelial cells display the property of contact
    inhibition of locomotion \citep{abercrombie1970contact}.
    In this cellular growth response, cells attempt to move in an
    opposite direction after contact with another.  As the number of
    cells increases and they become surrounded by neighbours, the
    available directions are reduced.
    Like most cancer cell lines, HeLa cells have
    lost the ability to activate contact inhibition
    \citep{stephenson1982locomotory},
    so we obtained a primary horse fibroblast cell line.
    Because primary cell line transfections typically have much lower
    efficiency than cancer cell lines, we transfected cells at
    \pcent{70} confluence and performed FRAP after 3 days.
    This timing enabled us to perform the transfection while cells
    were actively dividing, which increases the
    efficiency, and to perform the imaging once they reached confluence for
    reduced motility.

    Despite reaching confluence where contact inhibition
    of the fibroblasts was expected,
    we continued to observe movement of the transfected
    horse cells \frefp{fig:kill-frap:confluent-horse}.
    The characteristics of cell movement also
    differed dramatically from HeLa cells.
    The horse cell nuclei exhibit a helical motion
    about the vector of their movement
    in $x$ and $y$ axes \frefp{fig:kill-frap:confluent-horse},
    whereas HeLa nuclear motion was mostly
    restricted to the $z$ axis relative to
    the dish \frefp{fig:kill-frap:cell-movement}.

    \begin{figure}
      \centering
      \includegraphics[width=\textwidth]{results/confluent-horse.png}
      \captionIntro{Movement of confluent primary horse
                    cells during a FRAP experiment}
        {
          Immortalised horse cells were transfected with pBOS--H2B-EGFP
          with circle FRAP performed in a widefield microscope over 8~hours.
          Pre-bleach image at top left followed by
          image sequence at \SI{15}{\min} intervals.
        }
      \label{fig:kill-frap:confluent-horse}
    \end{figure}

    \subsection{Image-based tracking of cell movement}

    As an alternative strategy we implemented CropReg, a script
    to automate cell tracking of time series sequences.
    Using this image processing approach we were
    able to track individual HeLa cell nuclei
    throughout an entire sequence of FRAP images
    provided that nuclei did not overlap \frefp{fig:kill-frap:cropreg}.
    Although only a minority of cell image sequences satisfied this requirement
    throughout the full 8~hour duration of observations,
    it was possible to collect a sufficient number of
    cell observations for FRAP calculations.

    \begin{figure}
      \centering
      \includegraphics[width=\textwidth]{results/cropreg.png}
      \captionIntro{Automatic tracking and alignment of moving cells}
        {
         HeLa-derived stable cell line expressing H3--YFP
         with circle FRAP performed in a widefield microscope over 8~hours.
         Pre-bleach image at top left followed by
         image sequence at 20~min intervals.
         The top left of each image displays the cell tracked and transformed
         by automated cropping and registration.
        }
      \label{fig:kill-frap:cropreg}
    \end{figure}

  \subsection{Chromatin movement within nuclei}

    While performing the FRAP experiments, we observed movement
    of fluorescent chromatin features
    within cell nuclei that was supplementary to the overall
    motion of the cell itself \frefp{fig:kill-frap:frap-spot-movement}.
    This could not be accounted for by simple rotational movement of nuclei
    as rigid bodies around the $x$ or $y$ axis,
    and instead appeared to involve movement of
    individual regions within nuclei.
    Bleach spots also frequently showed elliptical or more complex distortions
    indicative of structural movements in the chromatin
    \frefp{fig:kill-frap:frap-spot-movement}.

    \begin{figure}
      \centering
      \begin{tikzpicture}[inner sep=0pt]
        \node at (0,0) {\includegraphics[width=\linewidth]%
                       {results/spot-distortion.png}};
        \node[text=white] at (-5.4,3.9) {\SI{5}{\um}};
        \draw[black,very thick] (-5.2,3.2) circle (7pt);
        \draw[->,black,very thick] (-2.1,2.5) -- (-2.1, 2.9);
        \draw[->,black,very thick] (1.0,2.5) -- (1.0, 2.9);
        \draw[->,black,very thick] (-1.9,-3.4) -- (-1.9, -3.0);
        \draw[->,black,very thick] (1.2,-3.4) -- (1.2, -3.0);
      \end{tikzpicture}
      \captionIntro{Movement of nuclear features and bleach spot distortion}
        {
          FRAP experiment performed in a confocal microscope with HeLa
          stable cell line expressing H4--YFP.  Circle FRAP was
          performed after 15 pre-bleach images and images acquired
          with \SI{60}{\sec} interval for a total of 4~hours.  Top two
          frames show the bleaching event, followed by the initial 3
          hours of recovery with \SI{20}{\min} interval.
        }
      \label{fig:kill-frap:frap-spot-movement}
    \end{figure}

    \subsection{Selection of \G1{} cells}

    One possible cause for changes in chromatin features that we observed
    is DNA replication and chromatin repackaging during S~phase.
    Furthermore, the doubling of histone content as a result of S~phase breaks
    a core assumption of FRAP that the system remains in equilibrium
    throughout the duration of the experiment.
    Therefore, the requirement to measure for a time period of 8~hours
    within a single cell cycle phase limits observations to \G1{}.

    To identify daughter cells that could be confidently assigned to early \G1{}
    because they had sufficient time to complete
    post-mitotic chromatin unpacking,
    cells in mitosis were selected and manually tracked for 4~hours.
    This selection was challenging because HeLa mitotic cells round up
    as spheres with only a weak connection to the growth surface
    causing them to exit the field of vision.
    Low laser power and a \SI{30}{\min} observation interval
    was used to minimise fluorophore damage.
    Since our system did not permit simultaneous
    Z-stack and time lapse imaging,
    and because cells in mitosis are in a separate focal plane,
    imaging was performed with a maximal pinhole sized focused
    between the growing and mitotic cell planes.
    The resulting blurred images were sufficient to visualise cells during
    the entire period required for selection.
    However, even after carefully selecting cells early \G1{},
    structural movements within the bleached region could
    still be observed (data not shown).

%      \begin{figure}
%        \centering
%        \missingfigure{Hela cells splitting}
%        \captionIntro{Picking cells at early G$_1$.}
%                     {We imaged cells that were entering mitosis and picked their
%                      daughter cells for the FRAP experiments. Because HeLa cells lift
%                      away from the dish during mitosis, opening the
%                      pinhole and set the Z-center in between the cell dividing plane
%                      and dish bottom was necessary. Ends up nothing being properly in focus but we
%                      can track things fine. Of course, some cells still floated away.}
%        %% TODO explicit parameters
%        \label{fig:kill-frap:picking-early-g1}
%      \end{figure}


    \subsection{Chromatin movement observed by photoactivation}

    Although we had surmounted the technical challenges of collecting
    overlaid images of nuclei for long time periods
    in \G1{} cells containing stably expressing
    core histones tagged with fluorescent reporters,
    we were concerned about the non-homogeneity of chromatin behaviour.

    To assess the extent of the chromatin motion,
    we performed photoactivation
    to track the movement of chromatin in the activated region alone.
    For this purpose we fused H2B to photoactivatable GFP as H2B--PAGFP.
    Since PAGFP cannot be easily detected before photoactivation,
    cells were co-transfected with mCherry--\textalpha--tubulin
    which localises exclusively to the cytoplasm
    and provides an outline of the nuclear region
    \frefp{fig:kill-frap:ifrap-pre}.

    Considerable non-homogenous movement of chromatin was clearly evident
    after activating and following H2B--PAGFP in
    \G1{} cells \frefp{fig:kill-frap:ifrap}.
    Instead of homogeneous diffusion of fluorescence,
    activated spots uncurled over time with individual channels of
    localised PAGFP appearing in the nuclei \frefp{fig:kill-frap:ifrap}.
    This finding suggests that quantitative FRAP is not tractable using
    a simple FRAP model based on homogenous non-directional diffusion.

    \begin{figure}
      \centering
      \subbottom[pre-activation]{
        \label{fig:kill-frap:ifrap-pre}
        \begin{tikzpicture}[inner sep=0pt]
          \node at (0,0) {\includegraphics[width=0.45\linewidth]%
                          {results/ifrap-pre.png}};
          \node[text=white] at (1.6,-1.6) {\SI{10}{\um}};
        \end{tikzpicture}
      }
      \hfill
      \subbottom[post-activation]{
        \label{fig:kill-frap:ifrap-post}
        \includegraphics[width=0.45\textwidth]
        {results/ifrap-post.png}
      }
      \subbottom[activated spot over time]{
        \label{fig:kill-frap:ifrap-timeframe}
        \begin{tikzpicture}[inner sep=0pt]
          \node at (0,0) {\includegraphics[width=\linewidth]%
                          {results/ifrap.png}};
          \node[text=white] at (-5.4,3.1) {\SI{2}{\um}};
        \end{tikzpicture}
      }
      %% Can you add an additional section in ifrap figure with 3-4
      %% still images of different very stark non-homogenous diffusion
      %% examples
      \captionIntro{Photoactivation experiment demonstrating
                    complex chromatin movement}
        {
          HeLa cells co-transfected with H2B--PAGFP and
          mCherry--\textalpha--tubulin.
          \subcaptionref{fig:kill-frap:ifrap-pre}
          Cell shown before photoactivation of H2B--PAGFP
          in a widefield microscope, demonstrating
          mCherry-bounded nuclear region.
          \subcaptionref{fig:kill-frap:ifrap-post}
          Same cell shown immediately after photoactivation
          showning circle photoactivated H2B--PAGFP nuclear spot.
          \subcaptionref{fig:kill-frap:ifrap-timeframe}
          Image sequence at 20~min intervals
          after photoactivation showing complex
          channelling of H2B--PAGFP diffusion.
        }
      \label{fig:kill-frap:ifrap}
    \end{figure}

  \section{Discussion}

    We wished to quantitatively determine the
    effect on human chromatin dynamics
    of SIN mutations in core histones H3 and H4 known
    to be destabilising \textit{in vitro} and to affect
    cell growth in \species{S. cerevisiae}.
    We set out to use a previously reported circle FRAP model
    which accounts for multiple factors in a typical
    FRAP modelling \citep{mcnally-frap-code}.

    However, FRAP recovery is incomplete even
    after 8~hours for core histones \citep{KimuraCook}.
    This led us to address a series of technical challenges in
    collecting valid quantitative recovery data over extended time periods.

% Cell movement

    The first problem faced was cell motility, which
    is an expected property of actively dividing cells.
    We attempted to reduce motility by taking advantage of
    the fact that many primary cells display contact inhibition of
    locomotion and proliferation when they reach high densities.
    This contact inhibition is a natural mechanism
    that controls cellular growth in
    multicellular organisms, and results in a stop in proliferation
    with the formation of a monolayer of healthy cells in tissue culture.

    However, the approach has disadvantages including
    increased cell handling and reduced transfection efficiency.
    The potential inability to compare results with published data for
    immortalised cell lines such as HeLa is also undesirable.

    Despite achieving a monolayer of healthy cells that
    could be maintained stably over 2 weeks,
    individual transfected primary horse fibroblasts still showed motility
    despite exhibiting overall characteristics of contact inhibition.
    Furthermore, nuclei in these cells displayed a helical motion
    on the direction of cell movement \frefp{fig:kill-frap:confluent-horse}.

    The possibility of chemically inhibiting
    cells to reduce motion was considered
    since previous FRAP experiments with core
    histones were performed using multiple inhibitors
    of protein synthesis \citep{KimuraCook}. However, these studies revealed
    inhibitor-dependent variations in kinetics and the authors qualified
    their conclusions about the absolute accuracy
    of the histone exchange parameters measured.

    To better address the problem of cell motility we
    instead developed a computational approach
    by writing the program CropReg for cell tracking by normalised
    cross-correlation template matching.
    Using automated analysis enabled us to process
    the large numbers of cell images
    required to provide statistically valid
    quantitative measurements of core histone exchange.

% Compositional changes

    The second challenge to measuring core histone exchange by FRAP is that
    a chemical equilibrium is required between freely diffusing proteins and
    formation of a complex. Although absolute equilibrium is unlikely
    in the dynamic cell environment undergoing
    complex transcriptional and translation responses anyway,
    DNA replication involving polymerase passage
    and repackaging of the duplicated genome
    in S~phase will certainly unbalance any equilibrium.

    Chromosome compaction in mitosis also generates a chromatin environment
    that is distinct from interphase.
    This limits FRAP experiment to either \G1{} or \G2{} phases.
    The HeLa cell cycle has a typical \G1{} phase of 11.7~hours
    and a \G2{} phase of 3~hours \citep{HeLaCellCycle}
    so the extended time periods needed for FRAP of core histones requires
    starting FRAP early in \G1{} \frefp{fig:kill-frap:cell-cycle}.

    Post-mitotic chromosomes take approximately 2~hours
    to migrate within the nucleus
    and rebuild the interphase nuclear architecture during early \G1{}
    \citep{visualizationG1chromosomes,earlyg1position,RelativeChromosomePosition}.
    This defines the window for extended FRAP experiments
    from approximately 3 to 11 hours after mitosis
    in HeLa cells, although cells lines with even longer
    \G1{} phase could also be used \citep{PancreaticCells}.

    We wished to avoid the use of drugs for cell
    cycle arrest since this has been
    shown to influence FRAP results \citep{KimuraCook}.
    We also discounted serum starvation to move cells into the
    quiescent \G0{} phase since this could affect
    the relevance of measuring core histone
    exchange \citep{SerumStarvation}.

      \begin{figure}
        \centering
        %% based on original code from Robert Vollmert
        %% http://www.texample.net/tikz/examples/pie-chart/
        \newcommand{\slice}[4]{
          \pgfmathparse{0.5*#1+0.5*#2}
          \let\midangle\pgfmathresult

          % slice
          \draw[thick,fill=black!10] (0,0) -- (#1:1) arc (#1:#2:1) -- cycle;

          % outer label
          \node[label=\midangle:#4] at (\midangle:1) {};

          % inner label
          \pgfmathparse{min((#2-#1-10)/110*(-0.3),0)}
          \let\temp\pgfmathresult
          \pgfmathparse{max(\temp,-0.5) + 0.8}
          \let\innerpos\pgfmathresult
          \node at (\midangle:\innerpos) {#3};
        }
        \begin{tikzpicture}[scale=3]
          \newcounter{a}
          \newcounter{b}
          %% Total cell cycle is 24.5 hours, G1 is 11.7h, S is 8.8h,
          %% G2 is 3h, M is 1h. The problem is that the counters can't handle
          %% decimal places so we have a variable with the actual time for
          %% the text, and another one times 10 to calculate the angle.
          \foreach \p/\t/\l in {117/11.7/\G1, 9/0.9/M,
                                31/3.1/\G2, 88/8.8/S}
            {
              \setcounter{a}{\value{b}}
              \addtocounter{b}{\p}
              \slice{36*\thea/24.5} % we multiply by 36 instead of 360 because
                    {36*\theb/24.5} % the time is already times 10
                    {\l}{\t{} hours}
            }
        \end{tikzpicture}
        \captionIntro{HeLa cell cycle phases and timing}
          {
            Under optimal growth conditions the HeLa cell has a median
            doubling time of 24~hours, with \G1~and
            S~phases of 11.7 and 8.8 hours respectively \citep{HeLaCellCycle}.
          }
        \label{fig:kill-frap:cell-cycle}
      \end{figure}

    Instead, we developed a procedure to track
    progression of cells manually during mitosis
    where visual identification of the cell cycle is possible.
    This allowed us to minimise the variations of normal cell growth
    and to identify individual cells exactly 3 hours after start of \G1{}.
    This has the added advantage of allowing time for maturation of GFP
    expressed during the establishment of interphase.
    The time interval between images during manual selection was increased and
    both resolution and laser intensity were reduced
    to minimise phototoxicity or bleaching.
    This resulted in a set of selected early \G1{}
    cells suitable for FRAP experiments.

    The fluorescently tagged histone proteins are constitutively expressed
    under the control of an EF-1\textalpha{} promoter,
    so they lack the 3' regulatory features of native histone genes.
    This regulation does not follow the normal
    expression program of a histone gene
    and could affect the distribution of the histone in chromatin.
    Constant expression of tagged histones by a strong constitutive promoter
    will enrich them in the \G1{} and early S~phase pools
    making subsequent incorporation in euchromatin more likely,
    relative to mid-late S~phase where heterochromatic sequences
    are replicated and packaged \citep{DNA-replication-timing}.

    A more realistic tagged histone expression profile could be achieved using
    flanking regulatory regions from native histone genes,
    as demonstrated for H3 and CENP--A \citep{pMH3-plasmid,Kevin-pCA-TAG}.
    Another potential solution is to insert GFP
    in-frame into the native gene locus by genome engineering,
    although the redundancy between the multiple canonical histone genes means
    that identifying the most appropriate isoform to target
    could introduce complexities.

    Protein synthesis inhibitors were used by \citet{KimuraCook}
    to address this issue,
    but this has the disadvantage of potentially affecting
    many other processes as discussed above.

      %% TODO: would be cool to create this figure
%      \begin{figure}
%        \centering
%        \missingfigure{a schematic of cell cycle, soluble pool}
%        \captionIntro{Distribution of tagged and endogenous histones during cell cycle}
%                     {
%                       This would be at least 3 different subplots. The first
%                       and the second are like the ones in Fig 7A of Kimura and
%                       Cook paper. The third one would show the ratio of each
%                       histone over time, i.e., 100\% tagged during all cell
%                       cell cycle and some endogenous during S phase. In this
%                       plots, also note where euchromatin and heterochromatin
%                       are replicated.
%                     }.
%        \label{fig:kill-frap:messy-histone-expression}
%      \end{figure}

% Movement of the reference

    The final challenge to measuring core histone
    exchange by FRAP that we identified
    was non-homogenous regional movement of chromatin itself.

    One possible cause for this movement is chromatin repackaging
    during DNA replication which we addressed by selecting cells at
    early \G1{} phase.
    One other cause is chromatin remodelling as part of a DNA damage
    response caused by the FRAP photobleaching event itself.
    The phototoxicity effects of a FRAP experiment are often dismissed
    on the basis that the photobleaching event of a typical FRAP
    experiment does not affect cell viability
    \citep{kruhlak2000reduced, KimuraCook, carrero2003using} but the
    DNA damage that such experiment may introduce, and the effect that
    the repair response of such damage may have on chromatin
    reconfiguration, has not yet been addressed.
    Still, the use of laser beams of different wavelength and longer
    exposure times are often used to introduce both single and double
    DNA strand breaks on which FRAP experiments are then performed
    \citep{stixova2014advanced, mari2006dynamic, kim2002specific}.

    Independently of the cause for the chromatin movement,
    it undermines the assumption of FRAP analysis that binding sites
    remain immobile throughout the FRAP experiment.
    This assumption is required to interpret recovery
    as the rate of movement of freely diffusing unbleached molecules into the
    bleached area which allows the kinetic
    rates \kon{} and \koff{} to be estimated.
    If chromatin binding sites also move then the recovery curve becomes a
    much more complex function of both binding site movement and free diffusion.

    Chromatin movement is recognisable
    by changes in the intra-nuclear features of the fluorescent chromatin
    and by changes in the circular bleach spot.
    Although some of these effects are subtle when observed by photobleaching,
    the photoactivation of an equal circular spot demonstrates
    clear non-homogenous reshaping of chromatin.
    Equivalent chromatin movement has also been reported
    for H4--PAGFP in strip photoactivation \cite{H4PAGFP-chromatin-movement}.

    The movements we observed were in the
    range of \SI{4}{\um}, which is double the size of the bleach spot,
    and exhibited complicated shapes reminiscent of channelling.
    This is consistent with chromosome distribution in nuclei that is
    territorial on the scale of \SI{5}{\um} \citep{sun2000size}
    separated by interchromosomal channels of
    \SIrange{10}{100}{\nm} \citep{gorisch2005histone}.

    The clarity of H2B--GFP imaging by photoactivation
    suggests the opportunity to analyse the
    paths taken by diffusing core histones.
    For example, simultaneous use of combined
    photoactivation and photobleaching
    of complementary dimer and tetramer histones could
    enable relative diffusion rates and paths to be determined.
    Alternatively, an enzymatic mechanism to incorporate a
    complementary photo-differented label
    into DNA  would facilitate masking for
    the original location at the same time as tracking the histone diffusion
    and enable quantitative FRAP.
    Nevertheless, it is important to recognise
    that such experiments would test the
    resolution and sensitivity of microscopes.

\section{Conclusion}

    Since its inception over 30 years ago, FRAP has been continuously
    improved
    through technical capabilities of light microscopy
    and sensitive kinetic models that are now able to take into account
    an increasing number of biophysical features such as container size,
    non-homogeneous distribution of fluorescence, and profile of bleach spot.

    Despite these advances, the ability to perform FRAP
    over extended time periods of several hours for highly stable complexes
    such as core histones is limited by the dynamic nature of the cell.

    We overcame the challenges of cell motility
    and selection of cells in \G1{} phase,
    but were not able to develop a method to adjust
    for changes in chromatin structure within the cell nucleus.
    While a photobleached spot appears stable and
    can be tracked over several hours,
    small natural disturbances and non-homogeneous diffusion
    impact on photorecovery
    and estimation of kinetic parameters.
    We find that FRAP is suitable for semi-quantitative
    estimates of slowly diffusing molecules
    but not for the precise quantitative comparisons
    required to compare core histone mutations.

    Ultimately, the issue of long observation times stems from the
    requirements of FRAP models to achieve full recovery of the mobile
    populations being measured.  Alternative techniques capable to
    detect the motion of isolated molecules would reduce the
    requirement for observation times and overcome the issue of
    chromatin movement.  However, the high concentration of histones
    in the nucleus makes it challenging to observe individual
    molecules as required for approaches such as single particle
    tracking.

    Fluorescence Correlation Spectroscopy (FCS) is a fluorescent
    microscopy technique providing estimates of dynamic parameters by
    using fluorescence fluctuations in a femtolitre volume.  A
    variation of FCS named Raster Scan Image Correlation Spectroscopy
    (RICS) has recently been developed that uses larger image regions
    such as an entire cell nuclei, and accounts for the spatial
    variation of dynamics \citep{digman2005rics}.  This technique
    could potentially provide not only a method to compare core
    histone mutations \textit{in vivo} but also their effect in
    different chromatin domains.

    Overall, the current period of rapid technological advances in
    cell biology research means that new techniques such as RICS offer
    hope to address specific challenges such as histone mobility for
    which existing approaches such as FRAP are poorly suited.

%  Single-molecule imaging of histones for short period of times in
%  live cells has recently been reported using super-resolution
%  imaging\addref[nature methods 7(9):717-719, 2010 and nature methods
%  8(1):7-9, 2011].

%  Also, use of PAGFP has been used to measure dynamics of H4 over
%  \SI{90}{\ms} reporting differences between interphase chromatin and
%  mitotic chromosomes\addref[Saera Hihara et al 2012].  However, the
%  difference between these two phases is the highest and might not be
%  comparable to the difference between histones variants\todo{study
%  this. Someone must have measured this}.

