\section{Results}

    To investigate the challenges of performing FRAP in mammalian cells
    over the several hours required to measure histone mobility,
    we transfected HeLa cells with H2B-EGFP under
    the control of an EF-1\textalpha{} promoter
    and observed recovery in cells for 3.5 hours
    \frefp{fig:kill-frap:cell-movement}.

    \begin{sidewaysfigure}
      \centering
      \subbottom[Pre-bleach]{%
        \label{fig:kill-frap:first-frap-pre}%
        \begin{tikzpicture}[inner sep=0pt]
          \node at (0,0) {\includegraphics[width=6.1cm]%
                          {"results/first-frap-20"}};
          \node[text=white] at (-2.2,2.3) {\SI{10}{\um}};
          \draw[red,thick] (1.95,0.55) circle (5pt);
          \draw[red,thick] (1.20,-0.80) circle (5pt);
          \draw[red,thick] (-0.05,-1.20) circle (5pt);
          \draw[red,thick] (-0.80,-2.30) circle (5pt);
          \draw[red,thick] (-1.55,0.30) circle (5pt);
          \draw[red,thick] (-2.60,1.95) circle (5pt);
        \end{tikzpicture}%
      }
      \subbottom[Post-bleach]{%
        \label{fig:kill-frap:first-frap-post}%
        \begin{tikzpicture}[inner sep=0pt]
          \node at (0,0) {\includegraphics[width=6.1cm]%
                          {"results/first-frap-21"}};
        \end{tikzpicture}%
      }
      %% The time interval between frames was not always equal.  This
      %% was probably to "catch" the slow and fast exchanging
      %% components.  See the dv.log associated with the image.:
      %% Image 20.
      %%      Time:       Mon Dec 14 18:44:08 2009
      %%      Time Point: 32.458 secs
      %% Image 21.
      %%      Time:       Mon Dec 14 18:44:12 2009
      %%      Time Point: 36.715 secs
      %% Image 55.
      %%      Time:       Mon Dec 14 19:36:11 2009
      %%      Time Point: 3156.737 secs
      %% Image 65.
      %%      Time:       Mon Dec 14 20:01:11 2009
      %%      Time Point: 4656.740 secs
      %% Image 75.
      %%      Time:       Mon Dec 14 20:26:11 2009
      %%      Time Point: 6156.742 secs
      %% Image 85.
      %%      Time:       Mon Dec 14 20:51:11 2009
      %%      Time Point: 7656.748 secs
      \\[-1ex] % negative vertical space
      \subbottom[\SI{52}{\min}]{% (3156.7-36.715) / 60
        \label{fig:kill-frap:first-frap:52-min}%
        \begin{tikzpicture}[inner sep=0pt]
          \node at (0,0) {\includegraphics[width=6.1cm]%
                         {"results/first-frap-55"}};
          \draw[->,red,very thick] (1.6,1.3) -- (2.0, 1.3);
          \draw[->,red,very thick] (0.4,-0.4) -- (0.8, -0.4);
          \draw[->,red,very thick] (-0.5,0.35) -- (-0.1, 0.35);
          \draw[->,red,very thick] (-2.0,0.6) -- (-1.6, 0.6);
          \draw[->,red,very thick] (-2.8,2.2) -- (-2.4, 2.2);
          \draw[->,red,very thick] (-1.3,-2.0) -- (-0.9, -2.0);
        \end{tikzpicture}%
      }
      \subbottom[\SI{77}{\min}]{% (4656.7-36.715)/60
        \begin{tikzpicture}[inner sep=0pt]
          \node at (0,0) {\includegraphics[width=6.1cm]%
                          {"results/first-frap-65"}};
        \end{tikzpicture}%
      }
      \subbottom[\SI{102}{\min}]{% (6156.7-36.715)/60
        \label{fig:kill-frap:first-frap:102-min}%
        \begin{tikzpicture}[inner sep=0pt]
          \node at (0,0) {\includegraphics[width=6.1cm]%
                          {"results/first-frap-75"}};
        \end{tikzpicture}%
      }%
      \captionIntro{Circle FRAP of H2B--EGFP in HeLa}%
        {
          FRAP experiment performed in a widefield microscope with
          HeLa stable cell line expressing H2B--EGFP.
          \subcaptionref{fig:kill-frap:first-frap-pre}
          Last of the acquired images before the bleach event.  A
          total of 20 images with a time interval of \SI{1.7}{\sec}
          were acquired before bleaching. Circles show the location
          for the bleaching events.
          \subcaptionref{fig:kill-frap:first-frap-post}
          First image post the bleach event.
          \subcaptionref{fig:kill-frap:first-frap:52-min}%
          --\subcaptionref{fig:kill-frap:first-frap:102-min}
          Selected frames from FRAP experiment.  Arrows show location
          of the bleach spot.  Total time of FRAP
          experiment was \SI{3.5}{\hour} with variable time interval.
          Initially with \SI{15}{\sec} and \SI{2.5}{\min} at the end.
        }
      \label{fig:kill-frap:cell-movement}
    \end{sidewaysfigure}
    %% Could we show a panel B as a plot of
    %% the aspect ratio of the bleach spot and cell over time
    %% as an illustration of the changes.

    The images taken at \SI{25}{\min} intervals revealed considerable changes
    in position of the nuclei, arrangement of features within each nucleus,
    and shape of the bleach spot \frefp{fig:kill-frap:cell-movement}.
    The central requirement of FRAP is to accurately identify and
    quantitate the signal in the photobleached spot over time.
    This led us to pursue both cell biological approaches to minimise motility
    and computational approaches to track imaged regions.

  \subsection{Inhibition of cell motility}

    We first attempted to reduce motility by restricting the space available
    using cells at higher confluency for FRAP experiments.
    This was performed using a HeLa cell line stably expressing
    H4~R45H to ensure even
    tagging of histone fluorescence in all cells.
    This resulted in some decrease in movement of cells
    but did not achieve complete immobilisation
    \frefp{fig:kill-frap:confluent-hela}.
    In fact, nuclei frequently underwent considerable reshaping as cells
    apparently squeezed between their neighbours.

    \begin{figure}
      \centering
      \begin{tikzpicture}[inner sep=0pt]
        \node at (0,0) {\includegraphics[width=\linewidth]%
                        {"results/confluent-hela"}};
        \node at (-5.90,3.22) {\SI{10}{\um}};
      \end{tikzpicture}%
      %% We show only the outline of the nuclei so to not distract
      %% from the message which is how the nuclei kept moving.
      \captionIntro{Movement of confluent HeLa cells during a FRAP experiment}
        {
          HeLa-derived stable cell line expressing H4~R45H--YFP
          were observed in a confocal microscope over 8~hours with
          \SI{1}{\min} interval.  Figure shows only the nuclei outline
          in intervals of \SI{50}{\min}.
        }
      \label{fig:kill-frap:confluent-hela}
    \end{figure}

    %% Could we show panel B as a plot of
    %% the deviation of the cell or nuclear centroid in X and Y
    %% relative to the first frame as a representative example
    %% for the HeLa cells in the first and second figures.
    %% This needs to be in absolute micron units
    %% independent of the magnification.
    %% We could even show the distribution of aspect ratio deviations
    %% for a sample of individual cells as a box plot as panel 2C

    Fibroblast and epithelial cells display the property of contact
    inhibition of locomotion \citep{abercrombie1970contact}.
    In this cellular growth response, cells attempt to move in an
    opposite direction after contact with another.  As the number of
    cells increases and they become surrounded by neighbours, the
    available directions are reduced.
    Like most cancer cell lines, HeLa cells have
    lost the ability to activate contact inhibition
    \citep{stephenson1982locomotory},
    so we obtained a primary horse fibroblast cell line.
    Because primary cell line transfections typically have much lower
    efficiency than cancer cell lines, we transfected cells at
    \pcent{70} confluence and performed FRAP after 3 days.
    This timing enabled us to perform the transfection while cells
    were actively dividing, which increases the
    efficiency, and to perform the imaging once they reached confluence for
    reduced motility.

    Despite reaching confluence where contact inhibition
    of the fibroblasts was expected,
    we continued to observe movement of the transfected
    horse cells \frefp{fig:kill-frap:confluent-horse}.
    The characteristics of cell movement also
    differed dramatically from HeLa cells.
    The horse cell nuclei exhibit a helical motion
    about the vector of their movement
    in $x$ and $y$ axes \frefp{fig:kill-frap:confluent-horse},
    whereas HeLa nuclear motion was mostly
    restricted to the $z$ axis relative to
    the dish \frefp{fig:kill-frap:cell-movement}.

    \begin{figure}
      \centering
      \begin{tikzpicture}[inner sep=0pt]
        \node at (0,0) {\includegraphics[width=\textwidth]%
                        {"results/confluent-horse"}};
        \node[text=white] at (-5.70,3.25) {\SI{10}{\um}};
        \draw[red,thick] (-5.15,1.45) circle (5pt);
      \end{tikzpicture}%
      \captionIntro{Movement of confluent primary horse
                    cells during a circle FRAP experiment}
        {
          Primary horse fibroblast transiently expressing H2B-EGFP
          imaged in a widefield microscope.  Red circle on the first
          image shows the location of the bleaching event.  Following
          images shows the nuclei movement at \SI{15}{\min} intervals
          for a total of 2 hours.
        }
      \label{fig:kill-frap:confluent-horse}
    \end{figure}

    \subsection{Image-based tracking of cell movement}

    As an alternative strategy we implemented CropReg, a script
    to automate cell tracking of time series sequences.
    Using this image processing approach we were
    able to track individual HeLa cell nuclei
    throughout an entire sequence of FRAP images
    provided that nuclei did not overlap \frefp{fig:kill-frap:cropreg}.
    Although only a minority of cell image sequences satisfied this requirement
    throughout the full 8~hour duration of observations,
    it was possible to collect a sufficient number of
    cell observations for FRAP calculations.

    \begin{figure}
      \centering
      \begin{tikzpicture}[inner sep=0pt]
        \node at (0,0) {\includegraphics[width=\linewidth]%
                       {results/cropreg.png}};
        \node[text=white] at (-3.1,0.55) {\SI{20}{\um}};
        \draw[->,white,very thick] (-3.4,3.4) -- (-3.4, 2.9);
      \end{tikzpicture}
      \captionIntro{Automatic tracking and alignment of moving cells}
        {
         HeLa-derived stable cell line expressing H3--EYFP
         observed in a widefield microscope.  Imaging was performed
         for a total of 6~hours with 10~min imaging interval.
         White arrow on the first panel corresponds to the cell that
         is being automatically tracked and aligned, and is inset on
         the top left corner of each panel.
        }
      \label{fig:kill-frap:cropreg}
    \end{figure}

  \subsection{Chromatin movement within nuclei}

    While performing the FRAP experiments, we observed movement
    of fluorescent chromatin features
    within cell nuclei that was supplementary to the overall
    motion of the cell itself \frefp{fig:kill-frap:frap-spot-movement}.
    This could not be accounted for by simple rotational movement of nuclei
    as rigid bodies around the $x$ or $y$ axis,
    and instead appeared to involve movement of
    individual regions within nuclei.
    Bleach spots also frequently showed elliptical or more complex distortions
    indicative of structural movements in the chromatin
    \frefp{fig:kill-frap:frap-spot-movement}.

    \begin{figure}
      \centering
      \begin{tikzpicture}[inner sep=0pt]
        \node at (0,0) {\includegraphics[width=\linewidth]%
                       {results/spot-distortion.png}};
        \node[text=white] at (-6.05,4.4) {\SI{5}{\um}};
        \draw[black,very thick] (-5.8,3.5) circle (7pt);
        \draw[->,black,very thick] (-2.35,2.7) -- (-2.35, 3.1);
        \draw[->,black,very thick] (1.0,2.7) -- (1.0, 3.1);
        \draw[->,black,very thick] (-2.0,-4.0) -- (-2.0, -3.6);
        \draw[->,black,very thick] (1.2,-4.0) -- (1.2, -3.6);
      \end{tikzpicture}
      \captionIntro{Movement of nuclear features and bleach spot distortion}
        {
          FRAP experiment performed in a confocal microscope with HeLa
          stable cell line expressing H4--YFP.  Circle FRAP was
          performed after 15 pre-bleach images and images acquired
          with \SI{60}{\sec} interval for a total of 4~hours.  Top two
          frames show the bleaching event, followed by the initial 3
          hours of recovery with \SI{20}{\min} interval.
        }
      \label{fig:kill-frap:frap-spot-movement}
    \end{figure}

    \subsection{Selection of \G1{} cells}

    One possible cause for changes in chromatin features that we observed
    is DNA replication and chromatin repackaging during S~phase.
    Furthermore, the doubling of histone content as a result of S~phase breaks
    a core assumption of FRAP that the system remains in equilibrium
    throughout the duration of the experiment.
    Therefore, the requirement to measure for a time period of 8~hours
    within a single cell cycle phase limits observations to \G1{}.

    To identify daughter cells that could be confidently assigned to early \G1{}
    because they had sufficient time to complete
    post-mitotic chromatin unpacking,
    cells in mitosis were selected and manually tracked for 4~hours.
    This selection was challenging because HeLa mitotic cells round up
    as spheres with only a weak connection to the growth surface
    causing them to exit the field of vision.
    Low laser power and a \SI{30}{\min} observation interval
    was used to minimise fluorophore damage.
    Since our system did not permit simultaneous
    Z-stack and time lapse imaging,
    and because cells in mitosis are in a separate focal plane,
    imaging was performed with a maximal pinhole sized focused
    between the growing and mitotic cell planes.
    The resulting blurred images were sufficient to visualise cells during
    the entire period required for selection.
    However, even after carefully selecting cells early \G1{},
    structural movements within the bleached region could
    still be observed (data not shown).

%      \begin{figure}
%        \centering
%        \missingfigure{Hela cells splitting}
%        \captionIntro{Picking cells at early G$_1$.}
%                     {We imaged cells that were entering mitosis and picked their
%                      daughter cells for the FRAP experiments. Because HeLa cells lift
%                      away from the dish during mitosis, opening the
%                      pinhole and set the Z-center in between the cell dividing plane
%                      and dish bottom was necessary. Ends up nothing being properly in focus but we
%                      can track things fine. Of course, some cells still floated away.}
%        %% TODO explicit parameters
%        \label{fig:kill-frap:picking-early-g1}
%      \end{figure}


    \subsection{Chromatin movement observed by photoactivation}

    Although we had surmounted the technical challenges of collecting
    overlaid images of nuclei for long time periods
    in \G1{} cells containing stably expressing
    core histones tagged with fluorescent reporters,
    we were concerned about the non-homogeneity of chromatin behaviour.

    To assess the extent of the chromatin motion,
    we performed photoactivation
    to track the movement of chromatin in the activated region alone.
    For this purpose we fused H2B to photoactivatable GFP as H2B--PAGFP.
    Since PAGFP cannot be easily detected before photoactivation,
    cells were co-transfected with mCherry--\textalpha--tubulin
    which localises exclusively to the cytoplasm
    and provides an outline of the nuclear region
    \frefp{fig:kill-frap:ifrap-pre}.

    Considerable non-homogenous movement of chromatin was clearly evident
    after activating and following H2B--PAGFP in
    \G1{} cells \frefp{fig:kill-frap:ifrap}.
    Instead of homogeneous diffusion of fluorescence,
    activated spots uncurled over time with individual channels of
    localised PAGFP appearing in the nuclei \frefp{fig:kill-frap:ifrap}.
    This finding suggests that quantitative FRAP is not tractable using
    a simple FRAP model based on homogenous non-directional diffusion.

    \begin{figure}
      \centering
      \subbottom[pre-activation]{
        \label{fig:kill-frap:ifrap-pre}
        \begin{tikzpicture}[inner sep=0pt]
          \node at (0,0) {\includegraphics[width=0.45\linewidth]%
                          {results/ifrap-pre.png}};
          \node[text=white] at (1.8,-1.9) {\SI{10}{\um}};
        \end{tikzpicture}
      }
      \hfill
      \subbottom[post-activation]{
        \label{fig:kill-frap:ifrap-post}
        \includegraphics[width=0.45\textwidth]
        {results/ifrap-post.png}
      }
      \subbottom[activated spot over time]{
        \label{fig:kill-frap:ifrap-timeframe}
        \begin{tikzpicture}[inner sep=0pt]
          \node at (0,0) {\includegraphics[width=\linewidth]%
                          {results/ifrap.png}};
          \node[text=white] at (-6.0,3.5) {\SI{2}{\um}};
        \end{tikzpicture}
      }
      %% Can you add an additional section in ifrap figure with 3-4
      %% still images of different very stark non-homogenous diffusion
      %% examples
      \captionIntro{Photoactivation experiment demonstrating
                    complex chromatin movement}
        {
          HeLa cells co-transfected with H2B--PAGFP and
          mCherry--\textalpha--tubulin.
          \subcaptionref{fig:kill-frap:ifrap-pre}
          Cell shown before photoactivation of H2B--PAGFP
          in a widefield microscope, demonstrating
          mCherry-bounded nuclear region.
          \subcaptionref{fig:kill-frap:ifrap-post}
          Same cell shown immediately after photoactivation
          showning circle photoactivated H2B--PAGFP nuclear spot.
          \subcaptionref{fig:kill-frap:ifrap-timeframe}
          Image sequence at 20~min intervals
          after photoactivation showing complex
          channelling of H2B--PAGFP diffusion.
        }
      \label{fig:kill-frap:ifrap}
    \end{figure}
