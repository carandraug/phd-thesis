\section{Materials and Methods}

  HeLa cells were transformed by lipofection
  \Srefp{methods:lipofection}.  For the generation of stable cell
  lines, cells were split to a low confluence and treated with
  \SI{3}{\ug\per\ml} Blasticidin-S.  Colonies were screened by
  fluorescence microscopy, and positive individual colonies were
  selected for further growth.  High fluorescence cells were sorted by
  FACS to generate an homogeneous cell line.  For transient expressing
  cells, transfection was performed 48h before imaging.

  Confocal microscopy was performed in a Zeiss LSM510 Meta microscope
  using glass bottom LabTek~II chambers.  Wide-field fluorescence
  microscopy was performed with an Applied Precision DeltaVision Core
  system using \SI{35}{\mm} glass bottom MatTek dishes.  In both
  cases, imaging was performed within an acrylic environmental chamber
  that fully enclosed the stage plate and microscope objectives.  A
  temperature of \dc{37} and \pcent{5} CO$_2$ levels were maintained
  via separate units connected to the environmental chamber.

  %% See the CropReg.m script for more details.  For example, only the
  %% region surrounding the original position was used for performance
  %% and robustness.
  Cell movement between time frames was performed by consecutive
  template-based registration by normalised cross-correlation.  Nuclei
  of interest were identified on the first time frame and used as
  template on the subsequent image.  To correct for rotational
  movement around the $z$ dimension along the optical axis, registered
  frames were aligned by rigid body geometric transformation in the
  ImageJ \citep{imagej1} plugin StackReg \citep{stackreg}.

  Automatic extraction and processing of FRAP recovery curves was
  performed with the GNU Octave programming language \citep{octave}
  and the Octave Forge image package.  Source code written in Matlab
  for a previously reported circle FRAP model
  \citep{mcnally-frap-code} was kindly offered by the original authors
  and ported to Octave.  A new program to perform this analysis steps
  with several options, \command{frapinator}, was released in a new
  FRAP Octave package, including all individual functions for images
  pre-processing and FRAP fitting \Srefp{sec:software:octave-frap}.
