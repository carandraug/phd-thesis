\section{Introduction}

    The chromatin packaging of eukaryote genomes
    compacts very large lengths of DNA into the
    microscopic cell nucleus, facilitates chromosomal movements
    during cell division, and provides a substrate for molecular
    mechanisms acting on the genome.

    The building block of eukaryotic chromatin is the
    nucleosome structure, comprising \SI{147}{\bp} of DNA wrapped around an
    octamer of two copies each of core histones
    H2A, H2B, H3, and H4 \citep{Luger1997structure}.
    In this structure, two H2A/H2B dimers flank a central H3/H4 tetramer.
    Nucleosome core particles are arranged
    in a linear chain separated by DNA linkers, and can be further
    compacted into higher order chromatin structures.

    Chromatin functionality at the molecular and cellular levels
    requires the capability for dynamic rearrangement.  Chromatin
    structure can be modulated through nucleosomes by changing the
    arrangement of histones and DNA in a process known as remodelling
    \citep{flaus2011mechanisms}, or by altering histone chemical
    composition through post-translational modification
    \citep{bannister2011ptm-review}, or exchange of histone variants
    \citep{TalbertHenikoff2010}.

    A large amount of information has been accumulated about the static
    structure of the nucleosome at atomic resolution
    \citep{mcginty2014histone}, and about the arrangement of polymeric
    chromatin \citep{kuznetsova2016chromatin}.  However, the
    mechanisms for dynamic rearrangement of chromatin are much less
    well understood or integrated between the molecular and polymer
    levels \citep{andrews2011nucleosome}.

  \subsection{Nucleosome dynamics and histone SIN mutants}

    The archetype of ATP-dependent nucleosome remodelling enzymes is
    the SWI/SNF complex, which was identified in screens for mating
    type SWItching \citep{SWI-mutants} and Sucrose Non Fermentation
    \citep{SNF-mutants-original-discovery, SNF-mutants2} in
    \species{Saccharomyces cerevisiae}.

    Mutations were subsequently identified that compensate for the
    loss of the SWI/SNF complex and these are collectively known as SIN
    mutations because they provide SWI/SNF INdependence
    \citep{kruger1995amino}.  A subset of SIN mutants are single
    amino-acid changes in core histones H3 and H4, providing a direct
    link between SWI/SNF and chromatin.  This also suggests the
    mutated residues in the histone proteins influence the same
    pathways for nucleosome dynamics that are leveraged
    by the remodelling enzyme, and that
    these residues are significant for nucleosome stability.

    The prediction that the stability of SIN mutant containing
    nucleosomes is affected in chromatin has been tested \textit{in vitro},
    where it was observed that SIN mutant nucleosomes display higher
    thermally driven nucleosome sliding mobility \citep{flaus2004sin}
    and that the mutated residues affect histone-DNA contacts in
    crystal structures \citep{muthurajan2004crystal}.

    However, the effect of histone protein SIN mutants
    in the more complex \textit{in vivo} chromatin environment
    of mammalian cells has not been demonstrated.
    Validating the functional significance of these residues
    is important for understanding nucleosome dynamics and
    explaining the high degree of conservation of
    histone protein sequences in eukaryotes.

  \subsection{Fluorescence Recovery After Photobleaching}

    Fluorescence Recovery After Photobleaching (FRAP) is an optical technique
    that can be used to report the dynamics of fluorescently
    tagged molecules within live cells.
    Tagged molecules inside a small region are irreversibly photobleached by
    a focused high power laser beam and the recovery rate of fluorescence
    in the bleached area is measured. The recovery rate is interpreted
    as unbleached fluorescing molecules from outside the region
    at the time of photobleaching diffusing into the bleached area.
    It is assumed that this fluorescence recovery
    reflects natural protein movement.

    A simple chemical equilibrium underlies the model for FRAP for a
    molecule with a single binding reaction:

    \begin{displaymath}
      F + S \overset{k_{on}}{\underset{k_{off}}{\rightleftharpoons}} FS
    \end{displaymath}

    where $F$ represents freely diffusing proteins,
    $S$ represents immobile vacant binding sites,
    and $FS$ is the complex between the two when the
    proteins are bound to the sites.
    The value of \kon{} and \koff{},
    are estimated from the rate at which photobleached $F$
    is replaced in the $FS$ complex.
    However, complexity is added to this simple model if
    diffusion and space are considered.

    Ongoing development of FRAP has led to increasingly complex models
    with more precision and accuracy than the inverse of an
    exponential decay.  For example, $1 - Ae^{-k_{off}t}$ where
    $t$ is time, \koff{} is the dissocation constant, and $A$ is the
    mobile fraction which can be used to estimate the association
    constant \kon{} \citep{mcnally-frap-2010}.
    Despite the sophistication of current models, FRAP requires assumptions
    that are difficult to maintain over long experimental observation times.
    Firstly, equilibrium must be maintained throughout the entire experiment
    so that both \kon{} and \koff{} remain constant.
    This also requires that concentrations of both $F$ and $S$ remain constant.
    Secondly, distribution of the fluorescently tagged molecules
    must mimic the endogenous protein.
    And finally, the binding sites must be part of a large,
    relatively immobile complex
    on the time and length scale of the recovery.
    In addition, different FRAP models will have further constraints
    based on the assumptions involved in its design \citep{mcnally-frap-2010}.

  \subsection{FRAP measurements of histones}

    FRAP has been extensively used to obtain qualitative and
    quantitative insight into the kinetic properties of chromatin bound
    proteins \citep{phair2000high, essers2005nuclear, agresti2005gr}.
    These rely on the established assumption that chromatin is
    relatively immobile in the interphase nucleus
    \citep{abney1997chromatin} since most proteins show recovery
    on the scale of seconds to minutes.
    H2B--GFP \citep{KevinH2BGFP} has become
    the standard reference for the immobile fraction in these
    FRAP experiments \citep{dey2000bromodomain, kuipers2011highly,
    jullien2016chromatibody}.

    However, the dynamics of the core histones themselves
    was measured by FRAP in a seminal
    and widely cited study by \citet{KimuraCook}. Multiple H2B
    populations were delineated with distinctive exchange rates.  Some
    \pcent{3} of H2B had a rapid recovery within minutes, \pcent{40}
    had slow recovery with \halflife[] of \SI{130}{\minute},
    and over \pcent{50} of H2B molecules had a
    very slow recovery with \halflife[] of
    over 8 hours that was considered to be effectively immobile.

    In contrast to H2B which is a histone dimer component, the tetramer
    histones H3 and H4 were found to have even slower mobility.  There
    were no rapid populations, with only slow and very slow populations
    of \SIrange{16}{22}{\percent} and \SIrange{62}{68}{\percent} being
    identified respectively.

    In combination with additional heterokaryon data, Kimura and Cook
    interpreted the rapid, slow, and very slow by exchanging H2B
    populations as correlating with transcription units, euchromatin,
    and heterochromatin respectively.  They assigned over \pcent{80}
    of H3 and H4 as immobile, whereas the remaining
    \SI{\approx 20}{\percent} was suggested to be mobilised by
    remodelling.  This latter small but significant
    slowly exchanging fraction of histones
    provides an opportunity to observe the dynamics of
    tetramer histones such as SIN point mutants in live mammalian cells.

    %% Their mobile fraction was odd.  The claim is that it represents
    %% the freely moving histones, but the text says that it was the
    %% recovery before the first post-bleach image could be acquired.
    %% This sounds like they assumed complete photobleaching.
    %% However, it can be that this is actually the unbleached
    %% fraction or some other background.

  \subsection{Aims and objectives}

    We aimed to develop a technique capable of estimate the \textit{in
      vivo} effect of histone sequences on nucleosome dynamics and
    validate previous \textit{in vitro} studies.  We choose FRAP
    due to its previous use on comparing exchange between different
    histone types \citep{KimuraCook}.  As a positive control, we
    choose the H4 mutant R45H which has exhibited the highest
    increase in nucleosome mobility \textit{in vitro} studies
    \citep{flaus2004sin}.

    In attempting to achieve quantitative measurements
    we encountered multiple technical challenges associated with
    measuring subtle kinetic alterations in nucleosome dynamics
    over long time periods in live cells.
    This required us to define the limitations of FRAP for observing molecules
    such as core histones with extremely slow exchange rates.
