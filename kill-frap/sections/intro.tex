\section{Introduction}

  \subsection{Histone contribution to nucleosome dynamics}

    The building block of eukaryotic chromatin structure is the nucleosome, comprising
    \SI{147}{\bp} of DNA wrapped around an octamer of two copies of core histones H2A,
    H2B, H3, and H4 \citep{luger1997crystal}.
    Nucleosomes are arranged in a linear chain separated by DNA linkers, and
    can be further compacted into higher order chromatin structures.
    The role of chromatin not only encompasses DNA compaction,
    but also involves providing a dynamic complex to mediate access to genetic
    information by a capability to undergo reconfiguration of structure. \addref{Flaus?}

    Local reconfiguration of chromatin can be achieved
    by changing nucleosome structure or altering its composition.
    This can be performed through post-translational modifications
    or incorporation of histone variants,
    causing nucleosomes to alter DNA sequence preferences or recruit other proteins.
    Alternatively, chaperones and ATP-dependent chromatin remodelling complexes
    act extrinsically in nucleosome position.

    An important example of nucleosome remodelling is the SWI/SNF~complex
    whose deficiency causes growth defects in yeast.
    This complex was identified independently through screens for
    mating type SWIthching \citep{SWI-mutants}
    or Sucrose Non Fermentation \citep{SNF-mutants-original-discovery, SNF-mutants2}.

    A set of mutations have been uncovered that compensate
    for the loss of the SWI/SNF complex
    that are collectively known as SIN mutations because they
    provide SWI/SNF INdependence  \addref{?}.
    A subset of these are single amino-acid changes in core histones,
    providing a direct link between SWI/SNF and chromatin and suggesting
    that the histone mutation sites are involved in nucleosome dynamics and therefore
    of major importance in the nucleosome structure. This hypothesis has been tested \textit{in vitro}
    where SIN mutant nucleosomes display higher thermal mobility \citep{flaus2004sin}
    and observed in crystal structures of the mutants \addref{muthurajan2004crystal}.

    A complementary demostration of the effect of histone SIN mutants
    in the more complex \textit{in vivo} chromatin environment would
    validate the functional significance of these residues
    and contribute to the understanding of conservation of histone sequences

  \subsection{FRAP}

    Fluorescence Recovery After Photobleaching (FRAP) is an optical technique
    that reveals the dynamics of fluorescently tagged molecules within live cells.
    Tagged molecules inside a small region are irreversibly photobleached by
    action of a high-power focused laser beam and the recovery rate of fluorescence
    is measured. The recovery rate is interpreted as unbleached molecules
    from outside of the region at the time of photobleaching diffusing into the bleached area.
    It is assumed that this fluorescence recovery reflects natural protein movement.

    FRAP can be described by a simple chemical equilibrium:

    \begin{displaymath}
      F + S \overset{K_{on}}{\underset{K_{off}}{\rightleftharpoons}} FS
    \end{displaymath}

    where $F$ represents freely diffusing proteins, $S$ represents immobile vacant
    binding sites, and $FS$ the complex between the two, when the protein is bound
    to the binding site. This technique measures the value of \Kon{} and \Koff{},
    the association and dissociation constants by observing how fast a
    small population of photobleached $F$ in the $FS$ complex is replaced.

    This technique has been extensively used to obtain qualitative and quantitative
    insight on the kinetic properties of proteins, including histones \citep{KimuraCook}.
    These results show extremely slow recovery rates
    with histone residency half-lives longer than 8~hours.

    %% FIXME Much more descriptive on Kimura and Cook results

    Continuous development of FRAP has led to increasingly complex models
    that are both more precise and accurate than simple inverse exponential decay \addref{?}.
    Despite their sophistication, these models require certain important assumptions
    that are difficult to maintain for long experimental observation times:

    \begin{itemize}
      \item equilibrium is maintained throughout the entire experiment in
            order that both \Kon{} and \Koff{} remain constant,
            including that concentrations of both $F$ and $S$ remain constant;
      \item distribution of tagged molecule mimics the endogenous protein;
      \item the binding sites are part of a large, relatively immobile complex,
            on the time and length scale of the recovery.
    \end{itemize}

  \subsection{Aims}

    We wished to observe the structure--function relationship of the nucleosome
    using the histone SIN mutant H4~R45H that exhibited the
    highest increase in nucleosome mobility \textit{in vitro}.
    This led us to investigate the technical challenges of measuring subtle kinetic
    alterations of the nucleosome dynamics over long time periods in live cells.
