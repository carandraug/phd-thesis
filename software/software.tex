\chapter{Software Tools for Image and Sequence Analysis}

\todo[inline]{Add one page general background to broad needs and
  motivations of software.  Two sections on: Why software development
  is essential for cell biology (1 page), and Why FOSS is important for
  researchers (half page)}

\section{GNU Octave}

The GNU Octave programming language was heavily used for the analysis
of microscope images throughout this work.  Several features of the
language and its supportive community make it an attractive choice
for qualitative microscopy.

\todo[inline]{Add 2 sentences with a general description of Octave
  capabilities}

Biological microscope images have a varying number of dimensions.
In addition to the 2 dimensional plane of a standard image, microscope
outputs typically include any combination of: $z$ dimension along
the optical axis for a volume image; time dimension for time-lapse
experiments; and wavelength for multi-channel experiments.  Recent microscopy
techniques generate images with an even higher number of dimensions
such as angle, phase, and lifetime.  This varying dimensionality adds
complexity and makes it more challenging to write generalized subroutines for
microscope image analysis.

GNU Octave is an array programming language primarily intended for
numerical computations \citep{octave}.  Array programming languages
are high level programming languages where the primary data types and
operators \todo{explain data types and operators}
are generalised to multidimensional arrays.  This provides
an abstraction layer that is useful when writing code for an arbitrary
number of dimensions.

%% Doesn't seem to be any scientific study about REPL as an advantage
%% for prototyping.  Similarly, none about being better for
%% exploratory data analysis but I can't imagine anyone disagreeing
%% with it.
Octave also has a read–eval–print loop (REPL), or interactive top
level, like Lisp machines, IPython, and Unix shells, which reduces the
feedback time and provides an efficient
environment for exploratory data analysis.

%% No citation for this, only the anecdotal evidence of being the
%% project leader of octave Forge for 6 years.
The project has a large and active community
mainly composed of
scientists and engineers as a large support group of
specialists in numerical computations

GNU Octave and Octave Forge packages are free software which allows
the study, modification, and distribution of modified source.
We made extensive
use of this feature and have contributed to improving Octave and its
packages for the needs of quantitative microscopy.

In addition, Octave has
packages that focus on specific applications such as control
systems \citep{octave-control}, time-frequency analysis
\citep{octave-ltfat}, or level sets \citep{octave-level-set}.  The
Octave Forge project hosts
Octave packages and provides a collaborative environment for their
development.  The Octave Forge image package
provides a number of functions for image processing such as geometric
transformations, mathematical morphology, image registration, and
noise reduction.

%% These were actually lsm files but those are actually TIFFs.
While Octave is well suited to image
processing, we also identified several problems always related with
the large image size or number of dimensions.  Our microscopy images
were several megabytes in size.  For example, the single cell
FRAP experiments for histone dynamics generating TIFF files of 449
These images had a field of view of 300
by 300 pixels and 2500 time frames
on acquisition by an 8-bit camera.
Such files sizes are typical in the field of microscopy.

\subsection{Reading and Writing of Image Files}

%% Number of 88 major image formats taken from GraphicsMagick
%% documentation on February 2017.
Octave uses the GraphicsMagick C++ library for the reading and writing
of image files which provides a common API \todo{define API}
to a varied collection of
image format specific libraries covering almost
90 major image formats.  In Octave, this functionality is
provided via the functions \command{imfinfo}, \command{imread}, and
\command{imwrite}.

These three functions were rewritten with the aim of achieving
reduced memory usage, increased performance, improved
interface for multidimensional images, and new image types.
New options were added to read and write a series of image planes
with in a single function call, to read specific regions of interest in
individual planes, to read and write images
with floating point precision, and to append additional planes to existing image
files.

As part of this rewrite, existing features were improved such as
support for transparency and
indexed images, support for the CMYK colour model, EXIF
and GPS metadata, reading and writing of animations in GIF images, and
control over the image compression type.

To support this development, a system of configurable hooks \todo{explain hooks}
to the image IO \todo{define IO}
functions were added based on individual file formats.
This enables plugin \todo{explain plugin} support for new image file formats
without addition of new format specific functions, and improved access
to the existing formats.  For example, it is now possible to read
microscope specific metadata from \command{imfinfo}.
This system is available via the function \command{imformats}.

Changes extended to other functions of Octave related to images
including overhaul of functions involved in conversion between color
models, grayscale images, and indexed images to support integer and
floating point data types, and multiple dimensions.

\todo[inline]{list all functions added or upgraded in  table These functions
are: \command{gray2ind}, \command{hsv2rgb}, \command{ind2gray},
\command{ind2rgb}, \command{ntsc2rgb}, \command{rgb2hsv},
\command{rgb2ind}, and \command{rgb2ntsc}}

All changes were released with Octave version 3.8.0.


\subsubsection{Bio-Formats and Octave Java interface}

%% We kind of managed to read the pixel data.  This also required
%% knowing that LSM interleaves the actual pixel data with thumbnails
%% which need to be skipped.
While GraphicsMagick provides support to read many image formats, its
support for scientific microscope image formats is limited.
For example, Zeiss confocal microscopes save
images in the LSM file format which is a proprietary extension of TIFF.
Although this reads pixel data,
the file metadata such as pixel size, time interval, or region of
bleaching event cannot be retrieved.

Bio-Formats is a free software library for reading and writing image
data with a strong focus on microscopy image file formats
\citep{bioformats}.  It is written in the Java programming language and
used by other programs in the field of microscope image analysis such
as CellProfiler \citep{cellprofiler}, ImageJ \cite{imagej2}, and OMERO
\citep{omero}.

%% Before version 3.8.0, there was a separate Octave Forge package
%% that added the java interface.  Actually, Octave version 3.8.0
%% pretty much only merged the java package into it.
\todo{explain interface in 1 sentence}
Octave has a native Java interface that should enable integration with
Bio-Formats.
We identified a series of problems in the Octave Java interface that
could be solved either by improving Octave or Bio-Formats.

In Octave, we rewrote the wrapping and unwrapping of Java objects \todo{explain}
into Octave data types.  This enabled Octave to automatically
convert Java objects into native Octave data types.
Support for arrays of Java arrays was not enabled
since we did not require it and Octave still lacks
support for it.

In Bio-Formats, we modified the code to use a Java interface that is
both Octave and Matlab compatible.
This change considerably  simplified the packaging of
Bio-Formats for Matlab and Octave so that they are now
effectively the same code.
Finally, we configured the Bio-Formats build system \todo{explain}
to routinely prepare an Octave package as part of its releases.

All changes were released with Octave since version 4.0.0 and
Bio-Formats since version 5.1.2

\section{BioPerl}

\section{SCons}
