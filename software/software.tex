\chapter{Developed software}

\section{GNU Octave}

The GNU Octave programming language was heavily used for the analysis
of microscope images throughout this studies as several of the
language and features as well as its community makes it an attractive
choice for this problems.

Biological microscope images have a varying number of dimensions.
Other than the 2 dimensional plane of a standard image, microscope
images will typically include any combination of: $z$ dimension along
the optical axis for a volume image; time dimension for time-lapse
experiments; and wavelength for multi-channels.  Recent microscopy
techniques generate images with an even higher number of dimensions
such as angle, phase, and lifetime.  This varying dimensionality adds
complexity and makes it harder to write generalized subroutines for
microscope image analysis.

GNU Octave is an array programming language, primarily intended for
numerical computations \citep{octave}.  Array programming languages
are high level programming languages where the primary data types and
operators are generalised to multidimensional arrays.  This provides
an abstraction layer that is useful when writing code for an arbitrary
number of dimensions.

In addition to being an array programming language, Octave has several
packages that focus on problems of a specific domain such as control
systems \citep{octave-control}, time-frequency analysis
\citep{octave-ltfat}, or level-set \citep{octave-level-set}.  The
Octave Forge project is a separate project that provides a collection
of such Octave packages and a collaborative environment for their
development.  This includes the Octave Forge image package which
provides several functions for image processing such as geometric
transformations, mathematical morphology, image registration, and
noise reduction.

%% Doesn't seem to be any scientific study about REPL as an advantage
%% for prototyping.  Similarly, none about being better for
%% exploratory data analysis but I can't imagine anyone disagreeing
%% with it.
Octave also has a read–eval–print loop (REPL) or an interactive top
level, like Lisp machines, IPython, and Unix shells, which reduces the
feedback time and creates a good\todo{I probably can't say this word
  without reference} environment for exploratory data analysis.

%% No citation for this, only the anecdotal evidence of being the
%% project leader for 6 years.
The project has a large and activity community and being primarily
aimed at numerical computations, the community is mainly composed of
scientists and engineers which provide a large support group of
specialists.

Finally, GNU Octave and Octave Forge packages are free software.  This
allows one to study the existing source, as well as modifying it for
ones purpose and distributing the modified source.  We made extensive
use of this feature and have constantly improved Octave and its
packages to best fit our needs.  Octave is also a superset of the
Matlab language and while Matlab itself is a proprietary language,
code written for it is often free software which we were able to
improve and use.

%% These were actually lsm files but those are actually TIFFs.
While we found the Octave language to be well suited for image
processing, we also identified several problems always related with
the large image size or number of dimensions.  Our microscopy images
were several megabytes in size --- the single cell and half-nuclear
FRAP experiments for histone dynamics generated TIFF files of 449 and
512 megabytes respectively.  These images had a field of view of 300
by 300 pixels and 2500 time frames for the single cell images, and
1024 by 1024 pixels and 466 time points for the half-nuclear FRAP
images, both being acquired by an 8-bit camera.  While these are large
image files, they are not atypical in the field of microscopy.

\subsection{Reading and Writing of Image Files}

%% Number of 88 major image formats taken from GraphicsMagick
%% documentation on February 2017.
Octave uses the GraphicsMagick C++ library for the reading and writing
of image files which provides a common API to a varied collection of
image format specific libraries like libtiff, libpng, and libjpeg.  As
of version 1.3.25, released on September of 2016, it gives support for
over 88 major image formats.  In Octave, this functionality is
provided via the functions \command{imfinfo}, \command{imread}, and
\command{imwrite} which is part of Octave since version 3.2.0.

We rewrote this three functions with the aim of supporting more
options for reduced memory usage, increased performance, improved
usage for multidimensional images, and new image types.  With this
aim, we added new options to read and write a series of image planes
with a single function call, to read specific regions of interested in
individual planes, added support for reading and writing of images
with floating point precision, and to append planes to existing image
files.

As part of this rewrite, we also improved existing features, and
introduced several new features that do not have immediate usage in
the field of microscopy.  We improved support for transparency as well
as for indexed images, added support for the CMYK colour model, EXIF
and GPS metadata, reading and writing of animations in GIF images, and
control over the image compression type.

To support this development, we added a system of configurable hooks
to the image IO functions based on individual file formats.  It makes
it not only possible to plugin support of new image file formats
without addition of new format specific functions, but also to hook
into the existing formats.  For example, one could now hook reading of
microscope specific metadata into \command{imfinfo}.  We tested and
improved this system while improving the image IO functions
ourselves.  This system is available via the function \command{imformats}.

This changes extended to other functions of Octave related to images
and we overhauled all the images involved in conversion between color
models, grayscale images, and indexed images to support integer and
floating point data types, as well as N dimensions.  These functions
are: \command{gray2ind}, \command{hsv2rgb}, \command{ind2gray},
\command{ind2rgb}, \command{ntsc2rgb}, \command{rgb2hsv},
\command{rgb2ind}, and \command{rgb2ntsc}.

All of this changes were released with Octave version 3.8.0.


\subsubsection{Bio-Formats and Octave Java interface}

%% We kind of managed to read the pixel data.  This also required
%% knowing that LSM interleaves the actual pixel data with thumbnails
%% which need to be skipped.
While GraphicsMagick provides support to read many image formats, its
support for scientific microscope image formats is smaller and
accidental.  For example, the confocal microscopes we used all saved
their images in the LSM file format which is an extension of TIFF and
so we were able to read the pixel data.  However, this retrieved none
of the file metadata such as pixel size, time interval, or region of
bleaching event.

Bio-Formats is a free software library for reading and writing image
data with a strong focus on microscopy image file formats
\citep{bioformats}, written in the Java programming language.  It is
used by other programs in the field of microscope image analysis such
as CellProfiler \citep{cellprofiler}, ImageJ \cite{imagej2}, and OMERO
\citep{omero}, all of which are also free software.

%% Before version 3.8.0, there was a separate Octave Forge package
%% that added the java interface.  Actually, Octave version 3.8.0
%% pretty much only merged the java package into it.
Octave has a native Java interface since version 3.8.0 which would
ease development of any interface between Bio-Formats and Octave.
However, Bio-Formats already included an interface to Matlab which
should have worked in Octave.  Instead, we identified a series of
problems in the Octave Java interface.  These were separated into two
groups, one was addressed in Octave while the other in Bio-Formats.

In Octave, we rewrote the wrapping and unwrapping of Java objects into
Octave data types.  Specifically, we changed Octave to automatically
convert Java objects into native Octave data types, including arrays
of Java primitive data types which is the format of the data returned
by Bio-Formats.  We did not add support for arrays of Java arrays
since we did not require it.  As of version 4.2.0, Octave still lacks
support for it.

In Bio-Formats, we modified the code to use a Java interface that is
both Octave and Matlab compatible.  Matlab has two available syntax
for interfacing with Java but Octave only implements one of them.
This change in Bio-Formats meant that ist package for Matlab and
Octave are effectively the same code, the only difference being on the
actual packaging format.  Finally, we configured the Bio-Formats build
system to prepare an Octave package as part of their own releases so
that it was handled by Octave's package manager.

All of this changes were released with Octave since version 4.0.0 and
Bio-Formats since version 5.1.2

\section{BioPerl}

\section{SCons}
