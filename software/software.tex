\chapter{Developed software}

\section{GNU Octave}

The GNU Octave programming language was heavily used for the analysis
of microscope images throughout this studies as several of the
language and features as well as its community makes it an attractive
choice for this problems.

Biological microscope images have a varying number of dimensions.
Other than the 2 dimensional plane of a standard image, microscope
images will typically include any combination of: $z$ dimension along
the optical axis for a volume image; time dimension for time-lapse
experiments; and wavelength for multi-channels.  Recent microscopy
techniques generate images with an even higher number of dimensions
such as angle, phase, and lifetime.  This varying dimensionality adds
complexity and makes it harder to write generalized subroutines for
microscope image analysis.

GNU Octave is an array programming language, primarily intended for
numerical computations \citep{octave}.  Array programming languages
are high level programming languages where the primary data types and
operators are generalised to multidimensional arrays.  This provides
an abstraction layer that is useful when writing code for an arbitrary
number of dimensions.

In addition to being an array programming language, Octave has several
packages that focus on problems of a specific domain such as control
systems \citep{octave-control}, time-frequency analysis
\citep{octave-ltfat}, or level-set \citep{octave-level-set}.  The
Octave Forge project is a separate project that provides a collection
of such Octave packages and a collaborative environment for their
development.  This includes the Octave Forge image package which
provides several functions for image processing such as geometric
transformations, mathematical morphology, image registration, and
noise reduction.

Octave also has a read–eval–print loop (REPL) or an interactive top
level, like Lisp machines, IPython, and Unix shells, which reduces the
feedback time and allowing for quicker prototyping.

%% No citation for this, only the anecdotal evidence of being the
%% project leader for 6 years.
The project has a large and activity community and being primarily
aimed at numerical computations, the community is mainly composed of
scientists and engineers which provide a large support group of
specialists.

Finally, GNU Octave and Octave Forge packages are free software.  This
allows one to study the existing source, as well as modifying it for
ones purpose and distributing the modified source.  We made extensive
use of this feature and have constantly improved Octave and its
packages to best fit our needs.  Octave is also a superset of the
Matlab language and while Matlab itself is a proprietary language,
code written for it is often free software which we were able to
improve and use.

%% These were actually lsm files but those are actually TIFFs.
While we found the Octave language to be well suited for image
processing, we also identified several problems always related with
the large image size or number of dimensions.  Our microscopy images
were several megabytes in size --- the single cell and half-nuclear
FRAP experiments for histone dynamics generated TIFF files of 449 and
512 megabytes respectively.  These images had a field of view of 300
by 300 pixels and 2500 time frames for the single cell images, and
1024 by 1024 pixels and 466 time points for the half-nuclear FRAP
images, both being acquired by an 8-bit camera.  While these are large
image files, they are not atypical in the field of microscopy.

\subsection{Reading and Writing of Image Files}

%% Number of 88 major image formats taken from GraphicsMagick
%% documentation on February 2017.
Octave uses the GraphicsMagick C++ library for the reading and writing
of image files which provides a common API to a varied collection of
image format specific libraries like libtiff, libpng, and libjpeg.  As
of version 1.3.25, released on September of 2016, it gives support for
over 88 major image formats.  In Octave, this functionality is
provided via the functions \command{imfinfo}, \command{imread}, and
\command{imwrite}.

%% This is not completely exact.  The format is actually a linked list
%% of IFDs and at the end of each IFD is the pointer to the next one.
%% Each IFD includes the pointer for its own page (if any).  So
%% technically, we don't need to read all the pages before, we need to
%% read all the IFDs before.  But I think that will just make things a
%% lot more complicated for nothing.
Microscopy images are often stored in TIFF, a computer file format
with open specifications originally developed for desktop publishing.
Its support for multiple pages means that microscope images with
multiple planes can be stored into a single file, one plane and
wavelength per page.  However, the pages are structured into a singly
linked list so reading a specific page implies scanning all the
previous pages first.  Octave's \command{imread} was originally
designed for reading a single page and required one call to
\command{imread} per page to be read.  This is slow because the
function is called and interpreted multiple times, but it is worse in
the case of TIFF since previously scanned planes are constantly being
rescanned.

One solution is to store the information about scanned pages and pass
such data to consecutive \command{imread} calls.  However, this adds
complexity to the function usage, and breaks the abstraction with TIFF
specific details that are important only for multipage images.
Octave's \command{imread} was instead improved to accept an extra
argument defining the range of images to be read or the string
\command{all}, allowing to read any number of pages with a single call
to \command{imread}.  This suits microscopy images since either all
planes are to be read, or their numbers were known in advance.
However, it does not provide a solution for when individual planes are
read, processed, and discarded sequentially.

This change was part part of a complete rewrite of Octave's image IO
functions with the aim of making it better suited to multidimensional
images, increased performance, and improved Matlab compatibility.

Other new features included support for handling images in floating
point precision and CMYK color model, writing of indexed images, and
appending to existing image files.


* handle rgb2ind and gay2ind for other data types and nd
* new function imformats
* a lot new fields for imfinfo and read exif and GPS



\subsubsection{Bio-Formats}

While GraphicsMagick provides support to read the pixel data of most
microscope image formats, it has no support for reading microscope
image metadata.


\section{BioPerl}

\section{SCons}
